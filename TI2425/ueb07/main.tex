\documentclass[10pt, oneside]{article}
\usepackage[a4paper, total={5.5in, 9in}]{geometry}
\usepackage[ngerman]{babel}
\usepackage{import}

\import{../../.texit/include/}{preamble}

\title{Technische Informatik\\[10pt]\Large{WiSe 2024/25}\\[15pt]\Large{{\"U}bungsblatt 7}}
\author{Volodymyr But\\[10pt]Hochschule Trier}
\date{}

% - - - - - - - - - - - - - - - - - - - - - - - - - - - - - - - - - - - - - - %

\begin{document}

\maketitle
\vspace{25px}

\section{Aufgabe 1}

Erstellen Sie mit Logisim:

\begin{enumerate}[(a)]
    \item ein Multiplexer mit 2 Eing"angen und 1 Steuerleitung. Nutzen Sie dazu ein NOT, zwei AND und ein OR Gatter.

        Siehe circuits/circuit-01-01.circ

    \item einen S-R Latch.

        Siehe circuits/circuit-01-02.circ

    \item einen gated S-R Latch.

        Siehe circuits/circuit-01-03.circ
\end{enumerate}

\section{Aufgabe 2}

Realisieren Sie eine OR Logik: $Y = A + B$

Siehe circuits/circuit-02-01.circ

\section{Aufgabe 3}

\begin{enumerate}[(a)]
    \item Erl"autern Sie den Unterschied zwischen einem D Latch und einem S-R Latch.

        S-R Latch:
        \begin{itemize}[$\rightarrow$]
            \item Zwei Eing"ange: S (Set) und R (Reset)
            \item Kann in einen ung"ultigen Zustand ($\text{S} = 1$ $\text{R} = 1$) geraten.
            \item Speicher Daten basierend auf den Set- und Reset-Signalen.
        \end{itemize}

        D Latch:
        \begin{itemize}[$\rightarrow$]
            \item Ein Eingang f"ur die Daten (D) und ein Steuer-Eingang (E).
            \item Kein ung"ultiger Zustand.
            \item Der Ausgang folgt dem D-Eingang, solange E-Eingang aktiv ist,
                und beh"alt den letzten Zustand, wenn E-Eingang inaktiv ist.
        \end{itemize}

    \item Welchen Vorteil hat ein gated S-R Latch gegen"uber einem S-R Latch?

        Bei einem gated S-R Latch ermöglicht der Enable-Eingang eine präzisere Steuerung darüber, wann
        das Latch auf die Eingänge S und R reagieren soll.
\end{enumerate}

\section{Aufgabe 5}
\begin{enumerate}[(a)]
    \item Zeichnen Sie ein S-R Flip-flop mit 2 NOR und 2 AND Gattern. Die
        Flankenerzeugung k"onnen Sie als Blackbox einzeichnen oder mit 3
        Negierern und einem AND eine aufsteigende Taktflanke anlegen.

        Danach bauen Sie das S-R Flip-flop um zu einem J-K Flip-flop.

        Siehe circuits/circuit-05-01.circ und circuits/circuit-05-02.txt

    \item Welchen Vorteil hat das J-K gegen"uber einem S-R Flip-flop?

        J-K Flip-Flop vermeidet den ungültigen Zustand, der im S-R Flip-Flop
        auftreten kann, und führt stattdessen zu einer umschaltenden (toggle)
        Funktion, wenn beide Eingänge J und K gleichzeitig auf 1 sind
\end{enumerate}

\end{document}
