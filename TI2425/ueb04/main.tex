\documentclass[10pt, oneside]{article}
\usepackage[a4paper, total={5.5in, 9in}]{geometry}
\usepackage[ngerman]{babel}

\usepackage{blindtext}
\usepackage{titlesec}
\usepackage{amsmath}
\usepackage[hidelinks]{hyperref}
\usepackage{parskip}
\usepackage{graphicx}
\usepackage{longtable}
\usepackage[shortlabels]{enumitem}
\usepackage{multirow}
\usepackage{nccmath}
\usepackage{rotating}
\usepackage{makecell}
\usepackage{multicol}
\usepackage{capt-of}
\usepackage{csquotes}
\usepackage{amsfonts}
\usepackage{caption}

\captionsetup[table]{position=bottom}

\titleformat{\section}
    {\normalfont\Large\bfseries}{}{0pt}{}

\let\oldsection\section
\renewcommand{\section}{
  \renewcommand{\theequation}{\thesection.\arabic{equation}}
  \oldsection}
\let\oldsubsection\subsection
\renewcommand{\subsection}{
  \renewcommand{\theequation}{\thesubsection.\arabic{equation}}
  \oldsubsection}

\makeatletter
\renewcommand{\maketitle}{
    \bgroup
    \centering
    \par\LARGE\@title  \\[20pt]
    \par\large\@author \\[10pt]
    \par\large\@date
    \par
    \egroup
}
\makeatother


\title{Technische Informatik\\[10pt]\Large{WiSe 2024/25}\\[15pt]\Large{L{\"o}sungen zum {\"U}bungsblatt 04}}
\author{Volodymyr But\\[10pt]Hochschule Trier}
\date{}

% - - - - - - - - - - - - - - - - - - - - - - - - - - - - - - - - - - - - - - %

\begin{document}
\sloppy

\maketitle
\vspace{25px}

\section{Aufgabe 1}

\begin{enumerate}[(a)]
    \item Auf welche beiden Arten kann die Zahl 0 im 1-er Komplement dargestellt werden?
        \begin{equation*}
            ...1111_{2} \quad\text{oder}\quad ...0000_{2}
        \end{equation*}
    \item Wie wird die 0 im 2-er Komplement dargestellt?
        \begin{equation*}
            ...0000_{2}
        \end{equation*}
\end{enumerate}

\section{Aufgabe 2}

Bestimmen Sie das 1-er Komplement der nachfolgenden Bin"arzahlen:
\begin{enumerate}[(a)]
    \item $101 : \overline{101} = 010$
    \item $110 : \overline{110} = 001$
    \item $1010 : \overline{1010} = 0101$
    \item $11010111 : \overline{11010111} = 00101000$
    \item $1110101 : \overline{1110101} = 0001010$
    \item $00001 : \overline{00001} = 11110$
\end{enumerate}

\section{Aufgabe 3}

Bestimmen Sie das 2-er Komplement der folgenden Bin"arzahlen:
\begin{enumerate}[(a)]
    \item $111 : \overline{00000111} + 1 = 11111001$
    \item $1101 : \overline{00001101} + 1 = 11110011$
    \item $10011 : \overline{00010011} + 1 = 11101101$
    \item $00111101 : \overline{00111101} + 1 = 11000011$
\end{enumerate}

\section{Aufgabe 4}

Wandeln Sie jedes Paar von Dezimalzahlen in ein Paar aus Bin"arzahlen um und
addieren beide Summanden mit Hilfe des 2-er Komplements:
\begin{enumerate}[(a)]
    \item 33 und 15 : $33_{10} + 15_{10} = 00100001_2 + 00001111_2 = 00110000_2 = 48_{10}$
    \item 56 und $-27$ : $56_{10} + -27_{10} = 00111000_2 + 11100101_2 = 00011101_2 = 29_{10}$
    \item $-46$ und 25 : $-46_{10} + 25_{10} = 11010010_2 + 00011001_2 = 11101011_2 = -21_{10}$
    \item $-110$ und $-84$ : $\begin{aligned}[t]-110_{10} + -84_{10} &= 1111111110010010_2 + 1111111110101100_2 =\\&= 1111111100111110_2 = -194_{10}\end{aligned}$
\end{enumerate}

\section{Aufgabe 5}

Wandeln Sie die dezimale Zahl $3,248 \cdot 10^4$ in eine single-precision
bin"are Flie{\ss}kommazahl um.
\begin{equation*}
    3.248_{10} \cdot 10^{4} = 0\ 10001101\ 11111011100000000000000_2
\end{equation*}

\section{Aufgabe 6}

Welchen dezimalen Wert hat die IEEE-754 Flie{\ss}kommazahl?
\begin{align*}
    0\ 01111110\ 10000000000000000000000_2 &= (-1)^0 \cdot (1.1_2) \cdot 2^{(126 - 127)} \\
                                           &= 1.1_2 \cdot 2^{-1} = 0.11_2 = 0.75_{10}
\end{align*}

\section{Aufgabe 7}

Welchen dezimalen Wert hat die IEEE-754 Flie{\ss}kommazahl?
\begin{align*}
    0\ 10000001\ 11000000000000000000000_2 &= (-1)^0 \cdot (1.11_2) \cdot 2^{(129 - 127)} \\
                                           &= 1.11_2 \cdot 2^{2} = 111.0_2 = 7.0_{10}
\end{align*}

\section{Aufgabe 8}

Stellen Sie den dezimalen Wert als IEEE-754 Fließkommazahl dar
\begin{enumerate}[(a)]
    \item $-0.75_{10} = 1\ 01111110\ 10000000000000000000000_2$
    \item $0,375_{10} = 0\ 01111101\ 10000000000000000000000_2$
    \item $98.7_{10} =  0\ 10000101\ 10001010110011001100110_2$
\end{enumerate}

\section{Aufgabe 9}

Die gr"o{\ss}te (normalisierte) IEEE-754 Zahl bei 32-Bit Darstellung ist
\begin{equation*}
    (1 - 2^{-24}) \cdot 2^{128}
\end{equation*}
Erl"autern Sie bitte, wie Sie auf diesen Wert kommen.
\begin{align*}
    0\ 11111110\ 11111111111111111111111 &= ((1 + 2^{-1} + 2^{-2} + ... + 2^{-23}) \cdot 2^{23}) : 2^{23} = \\
                                         &= ((2^{23} + 2^{22} + ... 1) : 2^{23}) \cdot 2^{127} = ((2^{24} - 1) : 2^{23}) \cdot 2^{127} = \\
                                         &= (2^1 - 2^{-23}) \cdot 2^{127} = (2^1 - 2^{-23}) \cdot 2^{127} \cdot \dfrac{2^{1}}{2^{1}} = \\
                                         &= (\dfrac{2^1}{2^1} - \dfrac{2^{-23}}{2^1}) \cdot 2^{127} \cdot 2^1 = \\[5pt]
                                         &= (1 - 2^{-24}) \cdot 2^{128}
\end{align*}


\end{document}
