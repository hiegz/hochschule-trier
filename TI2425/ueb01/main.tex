\documentclass[10pt, oneside]{article}
\usepackage[a4paper, total={5.5in, 9in}]{geometry}
\usepackage[ngerman]{babel}
\usepackage{import}

\import{../../.texit/include}{preamble}

\title{Technische Informatik\\[10pt]\Large{WiSe 2024/25}\\[15pt]\Large{\"Ubungsblatt 1}}
\author{Volodymyr But\\[10pt]Hochschule Trier}
\date{}

% - - - - - - - - - - - - - - - - - - - - - - - - - - - - - - - - - - - - - - %

\begin{document}
\sloppy

\maketitle
\vspace{25px}

\section{Aufgabe 1}

In der Vorlesung wurde die sogenannte \textbf{Von-Neumann-Architektur}
vorgestellt. Erl"autern Sie den so genannten \textbf{Von-Neumann-Flaschenhals}.

Die Von-Neumann-Architektur besteht aus drei Hauptkomponenten: dem Prozessor
(CPU), dem Hauptspeicher und dem Bus. Der Hauptspeicher speichert Daten und
Programme, die vom Prozessor ausgeführt werden. Der Bus verbindet den Prozessor
mit dem Speicher. Das Problem bei dieser Architektur besteht darin, dass sowohl
Befehle als auch Daten nacheinander geladen werden müssen, was zu einem Engpass
in der Verbindung zwischen Prozessor und Speicher führen kann. Die maximale
Datenübertragungsrate wird also zwischen Daten und Befehlen aufgeteilt, was als
Von-Neumann-Flaschenhals bezeichnet wird.

\section{Aufgabe 2}

Welche alternative Architektur vermeidet den Von-Neumann-Flaschenhals?
Skizzieren Sie diese alternative Architektur?

Eine der alternativen Architekturen, die den Von-Neumann-Flaschenhals
vermeiden, ist die Harvard-Architektur. Sie teilt den Hauptspeicher in zwei
Bereiche auf: den Programmspeicher und den Datenspeicher. Zusätzlich werden
zwei separate Busse eingerichtet, die diese Speicher mit der CPU verbinden.
Dadurch wird eine gleichzeitige Übertragung von Daten und Befehlen ermöglicht
und der Von-Neumann-Flaschenhals vermieden.

\vspace{10px}
\begin{figure}[h]
    \centering
    \includegraphics[width=1\textwidth]{./assets/harvard-architektur.png}
    \caption{Harvard-Architektur}
\end{figure}

\section{Aufgabe 3}

Erl"autern Sie die Unterschiede zwischen einer L/S-Architektur, einer
R/M-Architektur und einer R+M-Architektur.

\begin{table}[h]
    \centering
    \begin{tabular}{p{0.3\linewidth}|p{0.3\linewidth}|p{0.3\linewidth}}
        L/S-Architektur & R/M-Architektur & R+M-Architektur \\
        \hline
        teilt die Befehle in zwei Kategorien ein: Speicherzugriff und
        ALU-Operationen. Dabei müssen sowohl Operanden als auch das Ergebnis
        einer Operation in Registern liegen &
        ermöglicht die Durchführung von Operationen im Speicher sowie in
        Registern. Bei diesem Ansatz kann sich einer der Operanden für
        Operationen im Speicher befinden, während der andere in einem Register
        liegt &
        wie bei der R/M-Architektur, mit der Ausnahme, dass zusätzlich alle
        Operanden im Speicher oder in Registern sein können
    \end{tabular}
    \caption{L/S- vs R/M- vs R+M-Architektur}
\end{table}

\section{Aufgabe 4}

In der Vorlesung wurde die Funktionsweise eines Relais erl"autert. Wird der
Schalter im Steuerkreis geschlossen, dann wird der Elektromagnet aktiviert und
schaltet mechanisch den Schalter des Lastkreis.

\begin{enumerate}[(a)]
    \item Modifizieren Sie die Schaltung bitte so, dass eine logische
        UND-Schaltung realisiert ist. Sie ben"otigen dazu zwei Schalter im
        Steuerkreis. Denken Sie z.B. an eine Heckenschere oder eine
        Stanzmaschine, bei denen der Benutzer zwei Schalter dr"ucken muss um
        die Maschine zu bedienen.

        Siehe Abbildung~\ref{fig:und-circuit}

    \item Modifizieren Sie die Schaltung bitte so, dass eine logische
        ODER-Schaltung entsteht. Sie ben"otigen wieder Schalter im Steuerkreis.
        Denken Sie an den Notrufknopf von Patienten im Krankenhaus. Die rote
        Lampe im Schwesternzimmer (Lastkreis) erleuchtet wenn einer der
        Patienten den Schalter dr"uckt.

        Siehe Abbildung~\ref{fig:oder-circuit}

    \item Modifizieren Sie die Schaltung bitte so, dass eine logische
        NOT-Schaltung entsteht. Sie brauchen nur einen Schalter im Steuerkreis.
        Solange der Schalter geschaltet ist, bleibt die Maschine im Lastkreis
        aus. Wird der Schalter im Steuerkreis ge"offnet, l"auft die Maschine.
        Denken Sie z. B. an die Luftdruckbremse beim LKW. Solange Druck auf dem
        Kessel ist, bleibt die Bremsanlage unt"atig (Schalter im Steuerkreis
        geschlossen) und der LKW kann bewegt werden. Ohne Druck ist der
        Schalter offen und die Bremsanlage (Lastkreis) schaltet. Durch
        Bet"atigen des Bremspedals oder durch einen Defekt an der
        Luftdruckanlage w"urde der Schalter im Steuerkreis ge"offnet.

        Siehe Abbildung~\ref{fig:not-circuit}

    \item Jetzt bauen Sie noch bitte eine Schaltung mit zwei Schaltern A und B
        im Steuerkreis, so dass die Maschine im Lastkreis l"auft, solange weder
        A noch B geschlossen sind. Wird einer der beiden, oder werden beide
        geschlossen, wird der Lastkreis ausgeschaltet. Welche logische Funktion
        ist entstanden?

        Siehe Abbildung~\ref{fig:nor-circuit}
\end{enumerate}

\begin{figure}[p]
    \includegraphics[width=1\textwidth]{./assets/und-circuit.png}
    \caption{UND-Schaltung}
    \label{fig:und-circuit}
\end{figure}

\begin{figure}[p]
    \includegraphics[width=1\textwidth]{./assets/oder-circuit.png}
    \caption{ODER-Schaltung}
    \label{fig:oder-circuit}
\end{figure}

\FloatBarrier

 \begin{figure}[p]
    \includegraphics[width=1\textwidth]{./assets/not-circuit.png}
    \caption{NOT-Schaltung}
    \label{fig:not-circuit}
\end{figure}

\begin{figure}[p]
    \includegraphics[width=1\textwidth]{./assets/nor-circuit.png}
    \caption{NOR-Schaltung}
    \label{fig:nor-circuit}
\end{figure}

\FloatBarrier

\section{Aufgabe 5}

Erkl"aren Sie mit eigenen Worten die Bedeutung der Komponenten einer CPU.

\begin{enumerate}[(a)]
    \item \textbf{Steuerwerk} ist eine Einheit der CPU, die den Befehlsdecoder verwendet,
        um eine im Befehlsregister gespeicherte Operation in
        Steuersignale zu entschlüsseln, die den Befehl an andere Einheiten der
        CPU weiterleiten.

    \item \textbf{ALU} ist eine weitere Einheit der CPU, die dazu dient, Berechnungen
        (Addition, Subtraktion, ...) oder logischen Verkn"upfungen (UND, ODER,
        NICHT, ...) durchzuf"uhren.

    \item \textbf{Adresswerk} berechnet Adressen, die zum Abrufen von Daten aus dem
        Hauptspeicher erforderlich sind. Das Adresswerk als eine separate
        Einheit reduziert zus"atzlich die Anzahl der CPU-Zyklen, die für die
        Ausführung verschiedener Maschinenbefehle erforderlich sind, da es
        parallel zum Rest der CPU arbeitet.
\end{enumerate}

\section{Aufgabe 6}

Wozu werden die folgenden Komponenten einer CPU verwendet?

\begin{enumerate}[(a)]
    \item \textbf{Befehlsregister} -- Speichern des aktuell auszuf"uhrenden
            Befehls.
    \item \textbf{Programm-Counter (PC)} -- Speichern der Adresse des n"achsten
        auszuf"uhrenden Befehls.
    \item \textbf{AC-Register} -- temporärer Speicher für die Zwischenergebnisse
        von Berechnungen.
\end{enumerate}

\section{Aufgabe 7}

Es existieren verschiedene Architekturmodelle mit unterschiedlichen
Adressformaten. Schreiben Sie f"ur die nachfolgende Berechnung jeweils ein
Programm unter Verwendung von 3-Adress-, 2-Adress- und 1-Adress-Instruktionen:
\begin{equation*}
    x = (a \cdot b \cdot c) - (d + e)
\end{equation*}

\begin{enumerate}[(a)]
    \itemsep0em
    \item Berechnung unter Verwendung von 1-Adress-Instruktionen:
        \begin{verbatim}
        load a      ; ACC <- a
        mul b       ; ACC <- ACC * b
        mul c       ; ACC <- ACC * c
        sub d       ; ACC <- ACC - d
        sub e       ; ACC <- ACC - e
        store x     ; x <- ACC
        \end{verbatim}

    \vspace{-0.3em}
    \item Berechnung unter Verwendung von 2-Adress-Instruktionen:
        \begin{verbatim}
        mul a, b    ; a <- a * b
        mul a, c    ; a <- a * c
        sub a, d    ; a <- a - d
        sub a, e    ; a <- a - e
        \end{verbatim}

    \vspace{-0.3em}
    \item Berechnung unter Verwendung von 3-Adress-Instruktionen:
        \begin{verbatim}
        mul a, a, b ; a <- a * b
        mul a, a, c ; a <- a * c
        sub a, a, d ; a <- a - d
        sub a, a, e ; a <- a - e
        \end{verbatim}
\end{enumerate}

\end{document}
