\documentclass[10pt, oneside]{article}
\usepackage[a4paper, total={5.5in, 9in}]{geometry}
\usepackage[ngerman]{babel}

\usepackage{blindtext}
\usepackage{titlesec}
\usepackage{amsmath}
\usepackage[hidelinks]{hyperref}
\usepackage{parskip}
\usepackage{graphicx}
\usepackage{longtable}
\usepackage[shortlabels]{enumitem}
\usepackage{multirow}
\usepackage{nccmath}
\usepackage{rotating}
\usepackage{makecell}
\usepackage{multicol}
\usepackage{capt-of}
\usepackage{csquotes}
\usepackage{amsfonts}
\usepackage{caption}

\captionsetup[table]{position=bottom}

\titleformat{\section}
    {\normalfont\Large\bfseries}{}{0pt}{}

\let\oldsection\section
\renewcommand{\section}{
  \renewcommand{\theequation}{\thesection.\arabic{equation}}
  \oldsection}
\let\oldsubsection\subsection
\renewcommand{\subsection}{
  \renewcommand{\theequation}{\thesubsection.\arabic{equation}}
  \oldsubsection}

\makeatletter
\renewcommand{\maketitle}{
    \bgroup
    \centering
    \par\LARGE\@title  \\[20pt]
    \par\large\@author \\[10pt]
    \par\large\@date
    \par
    \egroup
}
\makeatother


\title{Technische Informatik\\[10pt]\Large{WiSe 2024/25}\\[15pt]\Large{L{\"o}sung zum {\"U}bungsblatt 5}}
\author{Volodymyr But\\[10pt]Hochschule Trier}
\date{}

% - - - - - - - - - - - - - - - - - - - - - - - - - - - - - - - - - - - - - - %

\begin{document}

\maketitle
\vspace{25px}

\section{Aufgabe 1}

\begin{enumerate}[(a)]
    \item Welchen dezimalen Wert hat die angegebene IEEE-754 Flie{\ss}kommazahl?
        \begin{equation*}
            1\ 10000010\ 10000000000000000000000
        \end{equation*}
        \begin{enumerate}[1.]
            \item Zeichen: $-\ (1)$
            \item Exponent: $10000010_2 - 127 = 130 - 127 = 3$ (also $2^3$)
            \item $(-1)^1 \cdot 1.1 \cdot 2^3 = -1100_2 = -12$
        \end{enumerate}

    \item Der Wert dieser Zahl wird mit $(-16)_10$ multipliziert. Wie sieht die
        entstandene Zahl in IEEE-754 (32-bit) Notation aus.
        \begin{enumerate}[1.]
            \item $(-12) \cdot (-16) = 192 = 11000000_2$
            \item $11000000 = 1.1 \cdot 2^7$
            \item Exponent: $127 + 7 = 134 = 10000110_2$
            \item Zeichen: $+\ (0)$
        \end{enumerate}
        \begin{equation*}
            0\ 10000110\ 10000000000000000000000
        \end{equation*}
\end{enumerate}

\section{Aufgabe 2}

Die gr"oste positive darstellbare Flie{\ss}kommazahl in IEEE-754 (32-Bit) Notation ist:
\begin{equation*}
    0\ 11111110\ 11111111111111111111111
\end{equation*}
\begin{enumerate}[(a)]
    \item Der Exponent hat dabei den Wert $2^n$. Denken Sie an den Bias-Wert.
        Wie gro{\ss} ist $n$?
        \begin{equation*}
            n = 11111110_2 - 127_{10} = 127
        \end{equation*}
    \item Bestimmen Sie den dezimalen Wert dieser Fließkommazah.
        \begin{align*}
            1.11111111111111111111111 \cdot 2^{127} &=   (1 + 2^{-1} + 2^{-2} + ... + 2^{-23}) \cdot 2^{127} = \\
                                                    &= (((1 + 2^{-1} + 2^{-2} + ... + 2^{-23}) \cdot 2^{23}) : 2^{23}) \cdot 2^{127} = \\
                                                    &=  ((2^{23} + 2^{22} + ... + 1) : 2^{23}) \cdot 2^{127} = ((2^{24} - 1) : 2^{23}) \cdot 2^{127} = \\
                                                    &=   (2^1 - 2^{-23}) \cdot 2^{127} = (2^1 - 2^{-23}) \cdot 2^{127} \cdot \dfrac{2^{1}}{2^{1}} = \\
                                                    &=   (\dfrac{2^1}{2^1} - \dfrac{2^{-23}}{2^1}) \cdot 2^{127} \cdot 2^1 = \\[5pt]
                                                    &=   (1 - 2^{-24}) \cdot 2^{128}
        \end{align*}
\end{enumerate}

\section{Aufgabe 3}

Die kleinste darstellbare positive denormalisierte Flie{\ss}kommazahl ($\neq
0$) in IEEE-754 (32-Bit) Notation ist:
\begin{equation*}
    0\ 00000000\ 00000000000000000000001
\end{equation*}
\begin{enumerate}[(a)]
    \item Wie gro{\ss} ist der Wert des Exponenten als 2-er Potenz? \\
        Bei denormalisierter Darstellung ist der Exponent immer gleich $2^{-126}$.
    \item Wie lautet der (dezimale) Wert der Mantisse?
        \begin{equation*}
            0.00000000000000000000001_2 = 2^{-23}
        \end{equation*}
    \item Weisen Sie nach, dass diese Flie{\ss}kommazahl den Wert $2^{-149}$ hat.
        \begin{align*}
            0.00000000000000000000001 \cdot 2^{-126} &= (0 + 2^{-23}) \cdot 2^{-126} = \\
                                                     &= 2^{-23} \cdot 2^{-126} = \\
                                                     &= 2^{-23 - 126} = 2^{-149}
        \end{align*}
\end{enumerate}

\section{Aufgabe 4}

Die kleinste positive normalisierte Flie{\ss}kommazahl in IEEE-754 (32-Bit) Notation ist.
\begin{equation*}
    0\ 00000001\ 00000000000000000000000
\end{equation*}
\begin{enumerate}[(a)]
    \item Wie gro{\ss} ist der Wert des Exponenten als 2-er Potenz?
        \begin{equation*}
            1 - 127 = -126 \text{, also  } 2^{-126}
        \end{equation*}
    \item Wie lautet der (dezimale) Wert der Mantisse?
        \begin{equation*}
            00000000000000000000000_2 = 0_{10}
        \end{equation*}
    \item Welchen Wert hat diese Flie{\ss}kommazahl? Sie d"urfen den Wert als
        2-er Potenz notieren.
        \begin{equation*}
            1 \cdot 2^{-126}
        \end{equation*}
\end{enumerate}

\section{Aufgabe 5}

Die gr"o{\ss}te positive denormalisierte Fließkommazahl in IEEE-754 (32-Bit)
Notation ist:
\begin{equation*}
    0\ 00000000\ 11111111111111111111111
\end{equation*}
\begin{enumerate}[(a)]
    \item Wie gro{\ss} ist der Wert des Exponenten als 2-er Potenz? \\
        Bei denormalisierter Darstellung ist der Exponent immer gleich $2^{-126}$.
    \item Notieren Sie den Wert der Mantisse in der Form ($1 - 2^k$).
        \begin{align*}
            0.11111111111111111111111_2 &= (2^{-1} + 2^{-2} + ... + 2^{-23}) = \\
                                        &= ((2^{-1} + 2^{-2} + ... + 2^{-23}) \cdot 2^{23}) : 2^{23} = \\
                                        &= (2^{22} + 2^{21} + ... + 1) : 2^{23} = \\
                                        &= (2^23 - 1) : 2^{23} = \\
                                        &= (1 - 2^{-23})
        \end{align*}
    \item Bestimmen Sie nun den dezimalen Wert der Flie{\ss}kommazahl.
        \begin{align*}
            0.11111111111111111111111_2 \cdot 2^{-126} &= (1 - 2^{-23}) \cdot 2^{-126} = \\
                                                       &= (2^{-126} - 2^{-149})
        \end{align*}
\end{enumerate}

\section{Aufgabe 6}

Konvertieren Sie die Dezimalzahl $1.2077 \cdot 10^4$ in eine bin"are,
single-precison Flie{\ss}kommazahl.
\begin{equation*}
    (12077)_{10} = (0010111100101101)_2 = 1.0111100101101_2 \cdot 2^{13}
\end{equation*}
In der IEEE-754 (32-Bit) Notation sieht diese Zahl folgenderma{\ss}en aus:
\begin{equation*}
    0\ 10001100\ 01111001011010000000000
\end{equation*}

\section{Aufgabe 7}

Die angegebene IEEE-754 Darstellung repr"asentiert den dezimalen Wert $(20.00)_{10}$
\begin{equation*}
    0\ 10000011\ 01000000000000000000000
\end{equation*}
Modifizieren Sie diese bitte so, dass der dezimale Wert durch $(-8)_{10}$
dividiert wird.
\begin{equation*}
    0\ 10000000\ 01000000000000000000000
\end{equation*}

\section{Aufgabe 8}

Wie gro{\ss} ist die Differenz (als 2-er Potenz) zwischen der gr"o{\ss}ten
positiven normalisierten Flie{\ss}kommazahl in IEEE-754 (32-Bit) Notation und der
zweitgr"o{\ss}ten Fließkommazahl?
\begin{enumerate}
    \item $\begin{aligned}[t]
            0\ 11111110\ 11111111111111111111111 &= (1 - 2^{-24}) \cdot 2^{128}
        \end{aligned}$
    \item $\begin{aligned}[t]
            0\ 11111110\ 11111111111111111111110 &= (1 + 2^{-1} + ... + 2^{-22}) \cdot 2^{127} \\
                                                 &= (2 - 2^{-22}) \cdot 2^{127} \\
                                                 &= (1 - 2^{-23}) \cdot 2^{128}
        \end{aligned}$
\end{enumerate}
\begin{align*}
    (1 - 2^{-24}) \cdot 2^{128} - (1 - 2^{-23}) \cdot 2^{128} &= (1 - 2^{-24}) - (1 - 2^{-23}) = \\
                                                              &= 2^{-23} - 2^{-24}
\end{align*}

\section{Aufgabe 9}

Wie gro{\ss} ist die Differenz (als 2-er Potenz) zwischen der zweitkleinsten
positiven normalisierten Flie{\ss}kommazahl in IEEE-754 (32-Bit) Notation und
der kleinsten positiven normalisierten?
\begin{enumerate}
    \item $\begin{aligned}[t]
            0\ 00000001\ 00000000000000000000000 &= 1 \cdot 2^{-126}
        \end{aligned}$
    \item $\begin{aligned}[t]
            0\ 00000001\ 00000000000000000000001 &= (1 + 2^{-23}) \cdot 2^{-126}
        \end{aligned}$
\end{enumerate}
\begin{align*}
    (1 + 2^{-23}) \cdot 2^{-126} - 1 \cdot 2^{-126} &= 1 + 2^{-23} - 1 = \\
                                                    &= 2^{-23}
\end{align*}

\section{Aufgabe 10}

Wie gro{\ss} ist die Differenz (als 2-er Potenz) zwischen der gr"o{\ss}ten
positiven denormalisierten Flie{\ss}kommazahl in IEEE-754 (32-Bit) Notation und
der zweitgr"o{\ss}ten denormalisierten Flie{\ss}kommazahl?
\begin{enumerate}
    \item $\begin{aligned}[t]
            0\ 00000000\ 11111111111111111111111 &= (2^{-1} + 2^{-2} + ... + 2^{-23}) \cdot 2^{-126} = \\
                                                 &= (((2^{-1} + 2^{-2} + ... + 2^{-23}) \cdot 2^{23}) : 2^{23}) \cdot 2^{-126} = \\
                                                 &= ((2^{22} + 2^{21} + ... + 1) : 2^{23}) \cdot 2^{-126} = \\
                                                 &= ((2^23 - 1) : 2^{23}) \cdot 2^{-126} = \\
                                                 &= (1 - 2^{-23}) \cdot 2^{-126}
        \end{aligned}$
    \item $\begin{aligned}[t]
            0\ 00000000\ 11111111111111111111110 &= (2^{-1} + 2^{-2} + ... + 2^{-22}) \cdot 2^{-126} = \\
                                                 &= (((2^{-1} + 2^{-2} + ... + 2^{-22}) \cdot 2^{22}) : 2^{22}) \cdot 2^{-126} = \\
                                                 &= ((2^{21} + 2^{20} + ... + 1) : 2^{22}) \cdot 2^{-126} = \\
                                                 &= ((2^22 - 1) : 2^{22}) \cdot 2^{-126} = \\
                                                 &= (1 - 2^{-22}) \cdot 2^{-126}
        \end{aligned}$
\end{enumerate}
\begin{align*}
    (1 - 2^{-23}) \cdot 2^{-126} - (1 - 2^{-22}) \cdot 2^{-126} &= 1 - 2^{-23} - 1 + 2^{-22} = \\
                                                                &= 2^{-22} - 2^{-23}
\end{align*}

\section{Aufgabe 11}

Wie gro{\ss} ist die Differenz (als 2-er Potenz) zwischen der zweitkleinsten
positiven denormalisierten Flie{\ss}kommazahl in IEEE-754 (32-Bit) Notation und
der kleinsten positiven denormalisierten Flie{\ss}kommazahl?
\begin{enumerate}
    \item $\begin{aligned}[t]
            0\ 00000000\ 00000000000000000000001 &= 2^{-23} \cdot 2^{-126}
        \end{aligned}$
    \item $\begin{aligned}[t]
            0\ 00000000\ 00000000000000000000010 &= 2^{-22} \cdot 2^{-126}
        \end{aligned}$
\end{enumerate}
\begin{align*}
     (2^{-22} \cdot 2^{-126}) - (2^{-23} \cdot 2^{-126}) &= 2^{-22} - 2^{-23}
\end{align*}

\pagebreak
\section{Aufgabe 12}

Wie gro{\ss} ist die Differenz (als 2-er Potenz) zwischen der kleinsten
positiven normalisierten Flie{\ss}kommazahl in IEEE-754 (32-Bit) Notation und
der gr"o{\ss}ten positiven denormalisierten Flie{\ss}kommazahl.
\begin{enumerate}
    \item $\begin{aligned}[t]
            0\ 00000001\ 00000000000000000000000 &= 1 \cdot 2^{-126}
        \end{aligned}$
    \item $\begin{aligned}[t]
            0\ 00000000\ 11111111111111111111111 &= (2^{-1} + 2^{-2} + ... + 2^{-23}) \cdot 2^{-126} = \\
                                                 &= (((2^{-1} + 2^{-2} + ... + 2^{-23}) \cdot 2^{23}) : 2^{23}) \cdot 2^{-126} = \\
                                                 &= ((2^{22} + 2^{21} + ... + 1) : 2^{23}) \cdot 2^{-126} = \\
                                                 &= ((2^23 - 1) : 2^{23}) \cdot 2^{-126} = \\
                                                 &= (1 - 2^{-23}) \cdot 2^{-126}
        \end{aligned}$
\end{enumerate}
\begin{align*}
    (1 \cdot 2^{-126}) - (1 - 2^{-23}) \cdot 2^{-126} &= 1 - 1 + 2^{-23} \\
                                                      &= 2^{-23}
\end{align*}

\end{document}
