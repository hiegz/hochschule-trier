\documentclass[10pt, oneside]{article}
\usepackage[a4paper, total={5.5in, 9in}]{geometry}
\usepackage[ngerman]{babel}

\usepackage{blindtext}
\usepackage{titlesec}
\usepackage{amsmath}
\usepackage[hidelinks]{hyperref}
\usepackage{parskip}
\usepackage{graphicx}
\usepackage{longtable}
\usepackage[shortlabels]{enumitem}
\usepackage{multirow}
\usepackage{nccmath}
\usepackage{rotating}
\usepackage{makecell}
\usepackage{multicol}
\usepackage{capt-of}
\usepackage{csquotes}
\usepackage{amsfonts}
\usepackage{caption}

\captionsetup[table]{position=bottom}

\titleformat{\section}
    {\normalfont\Large\bfseries}{}{0pt}{}

\let\oldsection\section
\renewcommand{\section}{
  \renewcommand{\theequation}{\thesection.\arabic{equation}}
  \oldsection}
\let\oldsubsection\subsection
\renewcommand{\subsection}{
  \renewcommand{\theequation}{\thesubsection.\arabic{equation}}
  \oldsubsection}

\makeatletter
\renewcommand{\maketitle}{
    \bgroup
    \centering
    \par\LARGE\@title  \\[20pt]
    \par\large\@author \\[10pt]
    \par\large\@date
    \par
    \egroup
}
\makeatother


\title{Modul Schl{\"u}sselkompetenzen\\[10pt]\Large{WiSe 2024/25}\\[15pt]\Large{{\"U}bungsblatt 2 - Lernen \& Ressourcen}}
\author{Abgabe von Volodymyr But\\[5pt][Matrikel-Nr.: 982324]}
\date{Hochschule Trier}

% - - - - - - - - - - - - - - - - - - - - - - - - - - - - - - - - - - - - - - %

\begin{document}

\maketitle
\vspace{25px}

\section{Aufgabe 1: Reflexionsaufgabe Kognitive Lernstrategien}
\begin{enumerate}[(a)]
    \item Welche Lernstrategien hast Du bisher beim Lernen verwendet?

        \begin{itemize}
            \item Ich organisiere den Stoff, indem ich Zusammenfassungen
                schreibe und Mindmaps erstelle, um die Zusammenhänge zwischen
                den Themen zu sehen.
            \item Um das Gelernte zu festigen, gehe ich regelmäßig durch die wichtigsten
                Punkte, was die Erinnerung verbessert.
            \item Ich versuche, neues Wissen mit bereits Gelerntem zu
                verknüpfen und suche nach praktischen Beispielen, um das
                Verständnis zu vertiefen.
        \end{itemize}

    \item Wie bist Du bisher beim Wiederholen von Lerninhalten vorgegangen?

        \begin{itemize}
            \item Ich wiederhole den Lernstoff in regelmäßigen Abständen. Dies
                hilft mir, den Lernstoff über einen längeren Zeitraum hinweg zu
                festigen und die Erinnerungen zu stärken, ohne mich zu
                überlasten.
            \item Anstatt passiv durch die Materialien zu gehen, versuche ich,
                aktiv das Gelernte abzurufen. Ich stelle mir Fragen zum Thema
                oder erkläre es mir selbst, um zu prüfen, ob ich das Wissen
                wirklich verstanden habe.
            \item Ich mache regelmäßig Übungsaufgaben oder Tests, um das
                Wissen anzuwenden und zu überprüfen, ob ich alles richtig
                verstanden habe.
        \end{itemize}

    \item Wie organisierst / ordnest Du Deine Lernmaterialien bevor du anfängst, zu lernen?

        \begin{itemize}
            \item Ich stelle sicher, dass ich alle relevanten Lernmaterialien
                beisammen habe. Dazu gehören Bücher, Vorlesungsnotizen,
                Online-Ressourcen, Skripte, Übungsaufgaben und eventuell auch
                Aufzeichnungen von Vorlesungen oder Videos.
            \item Bevor ich mit dem Lernen beginne, überlege ich mir, welche
                Themen oder Lektionen ich in der heutigen Sitzung abdecken
                möchte.
            \item Ich schaue mir die Gliederung des Kurses oder die Lernziele
                an, um sicherzustellen, dass ich alle relevanten Themen
                abgedeckt habe und keine wichtigen Inhalte übersehe.
        \end{itemize}

\end{enumerate}

\section{Aufgabe 2: Reflexionsaufgabe Metakognitive Lernstrategien}
\begin{enumerate}[(a)]
    \item Hast Du bisher beim Lernen bereits mit Lernzielen gearbeitet?

        \begin{itemize}
            \item Ich priorisiere meine Ziele nach Dringlichkeit und Relevanz,
                besonders wenn ich mich auf bevorstehende Prüfungen oder
                Deadlines vorbereite.
            \item Ich unterscheide zwischen langfristigen und kurzfristigen
                Lernzielen. Langfristige Ziele umfassen oft das gesamte Thema
                oder den gesamten Kurs. Kurzfristige Ziele sind kleinere,
                spezifische Aufgaben.
        \end{itemize}

    \item Planst Du vor dem Lernen, wie Du Deine Ziele erreichen kannst? Wenn
        ja, wie gehst Du dabei vor?

        \begin{itemize}
            \item Bevor ich mit einer Lernsession beginne, setze ich mir klare
                und spezifische Ziele.
            \item Nachdem ich das Ziel definiert habe, plane ich, wie viel Zeit
                ich für jedes Unterziel oder jede Aktivität aufwenden möchte.
            \item Ich unterteile das übergeordnete Ziel in kleinere,
                handhabbare Schritte.
            \item Ich plane auch immer Pufferzeit ein, um unvorhergesehene
                Probleme oder zusätzliche Wiederholungen zu berücksichtigen.
            \item Bevor ich beginne, stelle ich sicher, dass ich alle
                notwendigen Materialien und Ressourcen zur Verfügung habe.
        \end{itemize}

    \item Überprüfst Du beim Lernen, ob Du Deine Lernziele erreichst? Wenn ja,
        wie?

        \begin{itemize}
            \item Nachdem ich die großen Lernziele in kleinere Schritte
                unterteilt habe, überprüfe ich regelmäßig, ob ich diese
                „Zwischenziele“ erreiche.
            \item Ich frage mich selbst oder verwende Übungsaufgaben, um zu
                sehen, ob ich das Gelernte auch wirklich anwenden kann.
            \item Am Ende jeder Lernsitzung nehme ich mir kurz Zeit, um zu
                reflektieren, was ich erreicht habe.
        \end{itemize}

\end{enumerate}

\section{Aufgabe 3: Reflexionsaufgabe Ressourcenbezogene Lernstrategien – Lernumgebung}

Was brauchst Du, um gut und erfolgreich lernen zu können?

\begin{itemize}
    \item Der wichtigste Faktor ist ein ruhiger und ungestörter Ort, an dem ich
        mich voll und ganz auf das Lernen konzentrieren kann.
    \item Ich brauche alle relevanten Materialien griffbereit, damit ich nicht
        unnötig nach ihnen suchen muss.
    \item Übungsaufgaben und Tests sind für mich unverzichtbar, um das Gelernte
        anzuwenden und zu überprüfen.
    \item Ich brauche einen strukturierten Lernplan, der mir hilft, den
        Überblick zu behalten und sicherzustellen, dass ich alle wichtigen
        Themen abdecke.
    \item Der Zugriff auf das Internet, um ergänzende Materialien wie Videos,
        Online-Vorlesungen oder Fachartikel zu nutzen, ist ebenfalls wichtig.
\end{itemize}

\section{Aufgabe 4: Reflexionsaufgabe Ressourcenbezogene Lernstrategien – Lernherausforderungen}

\begin{enumerate}[(a)]
    \item Was waren bisher Deine größten Herausforderungen beim Lernen?

        \begin{itemize}
            \item Als Nicht-Muttersprachler ist das Verständnis der Fachsprache
                oft eine Herausforderung, besonders bei komplexen oder
                technischen Begriffen, die nicht immer einfach zu übersetzen
                sind.
            \item Manche Informationen, wie zum Beispiel Definitionen, Formeln
                oder bestimmte Details, fielen mir schwer, im Gedächtnis zu
                behalten, besonders wenn sie sich ähnelten oder komplex waren.
            \item Ablenkungen und laute Umgebung haben es mir manchmal schwer
                gemacht, fokussiert zu bleiben, was meinen Lernfluss unterbrach
                und zu ineffizientem Lernen führte.
        \end{itemize}

    \item Was hast Du dabei für Erfahrungen gemacht?

        \begin{itemize}
            \item Am Anfang fühlte sich das Lernen oft wie eine doppelte
                Herausforderung an – ich musste nicht nur den Lernstoff
                verstehen, sondern auch die Fachsprache bewältigen.
            \item Manchmal hatte ich das Gefühl, den Stoff nicht vollständig zu
                verstehen, weil mir die Sprache eine Barriere war. Bestimmte
                Themen wurden schwer nachvollziehbar, vor allem wenn
                komplizierte Erklärungen oder lange Sätze verwendet wurden.
            \item Das Sprechen mit Muttersprachlern oder das Lernen in Gruppen
                war eine besonders wertvolle Erfahrung. Beim Erklären eines
                Themas oder bei Diskussionen verstand ich den Sprachgebrauch
                besser und konnte mir merken, wie bestimmte Begriffe im Kontext
                verwendet wurden.
            \item Bildmaterialien, Diagramme und einfache Erklärungen sind mir
                eine große Hilfe gewesen. Wenn ich eine grafische Darstellung
                oder eine vereinfachte Erklärung finde, fällt mir das
                Verständnis des Themas viel leichter.
        \end{itemize}

    \item Wie konntest Du Dich selbst wieder auf Kurs bringen, um Deine
        Lernziele zu erreichen?

        \begin{itemize}
            \item Wenn ich das Gefühl habe, dass ich von meinen Zielen
                abweiche, zerlege ich den Plan in kleinere Teile Teile und
                setze neue Prioritäten.
            \item Ich habe angefangen, am Ende einer Lernsitzung kurz zu
                reflektieren, was gut gelaufen ist und wo ich vielleicht noch
                Anpassungen machen könnte.
            \item Um den Lernprozess angenehmer zu gestalten, habe ich mir nach
                einer erfolgreichen Lerneinheit eine kleine Belohnung gegönnt,
                wie z.B. eine kleine Auszeit.
        \end{itemize}

\end{enumerate}

\end{document}
