\documentclass[10pt, oneside]{article}
\usepackage[a4paper, total={5.5in, 9in}]{geometry}
\usepackage[ngerman]{babel}

\usepackage{blindtext}
\usepackage{titlesec}
\usepackage{amsmath}
\usepackage[hidelinks]{hyperref}
\usepackage{parskip}
\usepackage{graphicx}
\usepackage{longtable}
\usepackage[shortlabels]{enumitem}
\usepackage{multirow}
\usepackage{nccmath}
\usepackage{rotating}
\usepackage{makecell}
\usepackage{multicol}
\usepackage{capt-of}
\usepackage{csquotes}
\usepackage{amsfonts}
\usepackage{caption}

\captionsetup[table]{position=bottom}

\titleformat{\section}
    {\normalfont\Large\bfseries}{}{0pt}{}

\let\oldsection\section
\renewcommand{\section}{
  \renewcommand{\theequation}{\thesection.\arabic{equation}}
  \oldsection}
\let\oldsubsection\subsection
\renewcommand{\subsection}{
  \renewcommand{\theequation}{\thesubsection.\arabic{equation}}
  \oldsubsection}

\makeatletter
\renewcommand{\maketitle}{
    \bgroup
    \centering
    \par\LARGE\@title  \\[20pt]
    \par\large\@author \\[10pt]
    \par\large\@date
    \par
    \egroup
}
\makeatother


\title{Modul Schl{\"u}sselkompetenzen\\[10pt]\Large{WiSe 2024/25}\\[15pt]\Large{{\"U}bungsblatt 3 - Zeit- \& Selbstmanagement}}
\author{Abgabe von Volodymyr But\\[5pt][Matrikel-Nr.: 982324]\\[10pt]Hochschule Trier}
\date{}

% - - - - - - - - - - - - - - - - - - - - - - - - - - - - - - - - - - - - - - %

\begin{document}

\maketitle
\vspace{25px}

\section{Aufgabe 1. Selbstsch"atzung - Welcher Zeit-Typ bist Du?}

\begin{enumerate}[(a)]
    \item Ordne Dich einem oder mehreren Zeittypen (Mischform), die wir in der Vorlesung besprochen haben,
        zu und begr"unde Deine Wahl mit drei Beispielen aus Deinem Alltag.

        Ich würde mich selbst als eine Mischung aus Perfektionist, Planer und
        Multitasker beschreiben.
        \begin{itemize}
            \item Ich bevorzuge es, im Voraus Zeit zum Lernen einzuplanen und
                festzulegen, was in der kommenden Woche erledigt werden sollte
                und was nicht. Auf diese Weise behalte ich immer einen
                Überblick über meine Ziele und meinen Lernfortschritt.
            \item Ich beschäftige mich während der Woche gerne mit
                verschiedenen Übungen, Projekten und Aktivitäten, so dass ich
                normalerweise mehrere Dinge habe, auf die ich mich
                konzentrieren muss.
            \item Ich kümmere mich immer um kleine Details, wenn ich eine
                Aufgabe erledige, was zwar mehr Zeit in Anspruch nimmt, aber
                keinen großen Unterschied für die Aufgabe selbst macht.
        \end{itemize}

    \item Reflektiere, welche Vorteile Dein Zeit-Typ hat und welche
        Herausforderungen er mit sich bringt.

        Mein Zeitmanagement hilft mir, einen Überblick darüber zu behalten, was
        in der Woche erledigt wird und was nicht. Darüber hinaus habe ich immer
        ein tieferes Verständnis für meine Lernaufgaben und Fortschritte, weil
        ich mir so viel Zeit wie möglich nehme, um mich um die kleinen Details
        zu kümmern. Allerdings kann es auch ziemlich schwierig sein, mit diesem
        Tempo Schritt zu halten, da ich gerne mehrere Dinge im Blick habe.
\end{enumerate}

\section{Aufgabe 2. Reflexionsaufgabe Zeitfresser}

Zeitfresser halten Dich davon ab, effektiv und produktiv zu arbeiten.
Ablenkungen lauern überall. Ehe Du Dich versiehst, hast Du Dich bereits völlig
verzettelt.
\begin{enumerate}[(a)]
    \item Reflektiere, welche Zeitfresser Dich oft beeinflussen. W"ahle zwei
        Zeitfresser aus, die Dich besonders betreffen, und beschreibe typische
        Situationen, in denen diese auftreten.

        \pagebreak
        Ein Zeitfresser, der mich oft beeinflusst, ist die Tendenz, mich in Details zu
        verlieren. Zum Beispiel, wenn ich an einer Aufgabe arbeite, neige ich dazu,
        immer wieder kleine Aspekte zu überprüfen und zu perfektionieren, auch wenn sie
        keinen großen Einfluss auf das Endergebnis haben. Ein typisches Beispiel ist
        das Überarbeiten eines Textes, bei dem ich mich in der Formulierung von Sätzen
        verliere, obwohl der Inhalt bereits klar und gut strukturiert ist. Das kostet
        oft viel mehr Zeit, als ursprünglich geplant.

        Ein weiterer Zeitfresser ist mein Handy. Oft lasse ich mich leicht
        ablenken, wenn ich Benachrichtigungen oder Nachrichten bekomme, und
        dann verbringe ich mehr Zeit damit, auf dem Handy zu surfen oder in
        sozialen Medien zu stöbern, als ursprünglich geplant.

    \item Entwickle jeweils eine Strategie, wie Du diese Zeitfresser in Zukunft
        vermeiden oder eind"ammen kannst.

        Der erste Zeitfresser ist wirklich nicht leicht zu vermeiden. Ich
        schätze, der beste Weg ist, das „gut genug“-Stadium einer Aufgabe zu
        erkennen und einfach mit der nächsten Sache auf der ToDo-Liste
        fortzufahren.

        Den weitere Zeitfresser, das Ablenken durch Handy, kann ich
        reduzieren, indem ich Benachrichtigungen ausschalte oder das Handy
        bewusst in einen anderen Raum lege.
\end{enumerate}

\section{Aufgabe 3. Zeitplanung}

Nenne mindestens drei Vorteile f"ur eine schriftliche Zeitplanung und
erl"autere diese Vorteile.

\begin{itemize}
    \item Eine schriftliche Zeitplanung hilft, den Tag oder die Woche klar zu
        strukturieren. Indem man Aufgaben und Termine notiert, erhält man einen
        klaren Überblick darüber, was erledigt werden muss und wann. Das
        reduziert das Gefühl der Überforderung, weil man nicht mehr alles im
        Kopf behalten muss.
    \item Eine schriftliche Planung ermöglicht es, sich auf die wichtigen
        Aufgaben zu konzentrieren, anstatt ständig zwischen verschiedenen
        Tätigkeiten hin- und herzuspringen. Wenn man genau weiß, was zu tun ist
        und in welcher Reihenfolge, bleibt man fokussiert und vermeidet
        Zeitverlust durch unnötige Entscheidungen. Man hat zudem die
        Möglichkeit, realistische Zeitfenster für jede Aufgabe festzulegen, was
        die Effizienz steigert.
    \item Schriftliche Zeitplanung hilft, Prokrastination zu
        überwinden. Wenn man Aufgaben konkret aufschreibt und Termine festlegt,
        schafft man eine Verpflichtung gegenüber sich selbst. Dies kann
        besonders bei schwierigen oder unangenehmen Aufgaben hilfreich sein, da
        man sie nicht mehr „vergessen“ oder auf später verschieben kann. Der
        Plan gibt klare Fristen vor, die einen motivieren, aktiv zu werden und
        die Arbeit rechtzeitig zu erledigen.
\end{itemize}

\section{Aufgabe 4. Priorit"atensetzung im Studium}

\begin{enumerate}[(a)]
    \item Beschreibe, warum Priorit"aten im Studium wichtig sind, und nenne ein
        konkretes Beispiel.

        Prioritäten im Studium sind wichtig, weil sie helfen, die verfügbaren
        Ressourcen – wie Zeit, Energie und Aufmerksamkeit – gezielt auf die
        wichtigsten Aufgaben und Ziele zu konzentrieren. Ein Studium kann oft
        überwältigend wirken, besonders wenn es viele verschiedene Fächer,
        Projekte und Prüfungen gibt.

        Angenommen, ich habe eine Woche, in der ich gleichzeitig für eine wichtige
        Prüfungswoche lernen muss und eine umfangreiche Hausarbeit abgeben soll. Wenn
        ich keine Prioritäten setze, könnte ich mich dazu verleiten lassen, die
        Hausarbeit zu schreiben, weil es eine eher „greifbare“ Aufgabe
        ist, während ich das Lernen für die Prüfungen aufschiebe, obwohl diese eine
        viel größere Auswirkung auf meine Note haben könnten. Indem ich jedoch die
        Prüfungen als vorrangige Aufgabe einplane, stelle ich sicher, dass ich
        ausreichend Zeit und Fokus auf das Lernen verwende. Ich könnte die Hausarbeit
        in kleinere Abschnitte unterteilen und diese später in der Woche bearbeiten,
        wenn ich die wichtigsten Prüfungsinhalte schon bearbeitet habe.

    \item Erkläre, welche Folgen entstehen können, wenn Prioritäten nicht
        richtig gesetzt werden.

        Wenn man versucht, gleichzeitig alle Aufgaben gleichwertig zu
        behandeln, kann es zu einer Überlastung kommen. Ohne klare Prioritäten
        verliert man leicht den Überblick und fühlt sich von der Vielzahl der
        Aufgaben erdrückt. Oder man verschwendet wertvolle Zeit mit Aufgaben,
        die weniger wichtig sind. Dies kann dazu führen, dass man mehr Zeit für
        weniger produktive Tätigkeiten aufwendet, während die eigentlichen
        Ziele in den Hintergrund rücken.

\end{enumerate}

\section{Aufgabe 5. Vorteile des Selbstmanagements}

Nenne und erl"autere drei konkrete Vorteile des Selbstmanagements f"ur
Studierende. Beziehe Dich dabei auf typische Studienanforderungen.

\begin{itemize}
    \item Ein wichtiger Vorteil des Selbstmanagements ist die Verringerung von
        Stress, insbesondere während intensiver Prüfungsphasen oder bei der
        Bearbeitung von komplexen Gruppenarbeiten. Studierende, die ihre Zeit
        gut organisieren und Aufgaben rechtzeitig priorisieren, können ihre
        Arbeit in kleinere, überschaubare Schritte unterteilen.
    \item Ein weiterer Vorteil des Selbstmanagements ist die Steigerung der
        Effizienz. Studierende, die ihre Aufgaben klar strukturieren und
        Prioritäten setzen, vermeiden Zeitverschwendung und können ihre
        Arbeitsphasen gezielt nutzen. Dadurch bleibt mehr Zeit für andere
        Aktivitäten, wie etwa Freizeit oder Nebenjobs.
    \item Ein ausgewogenes Verhältnis zwischen Studium, Freizeit und sozialen
        Aktivitäten ist für das psychische und physische Wohlbefinden von
        Studierenden entscheidend. Selbstmanagement ermöglicht es Studierenden,
        Zeit für Pausen und Erholung einzuplanen, ohne dass das Studium
        darunter leidet. Dadurch wird eine bessere Work-Life-Balance erreicht,
        was langfristig die Motivation und Lebensqualität erhöht.
\end{itemize}

\end{document}
