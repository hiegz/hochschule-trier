\documentclass[10pt, oneside]{article}
\usepackage[a4paper, total={5.5in, 9in}]{geometry}
\usepackage[ngerman]{babel}

\usepackage{multicol}

\usepackage{blindtext}
\usepackage{titling}
\usepackage{titlesec}
\usepackage{amsmath}
\usepackage[hidelinks]{hyperref}
\usepackage{parskip}
\usepackage{graphicx}

\setlength{\droptitle}{-3cm}

\titleformat{\section}
    {\normalfont\Large\bfseries}{}{0pt}{}

\let\oldsection\section
\renewcommand{\section}{
  \renewcommand{\theequation}{\thesection.\arabic{equation}}
  \oldsection}
\let\oldsubsection\subsection
\renewcommand{\subsection}{
  \renewcommand{\theequation}{\thesubsection.\arabic{equation}}
  \oldsubsection}


\title{Modul Schlüsselkompetenzen\\[5pt]\Large WiSe 2024/25\\[10pt]{\"U}bungsblatt 1 - Studienorganisation}
\author{Abgabe von Volodymyr But\\[5pt][Matrikel-Nr.: 982324]}
\date{}

% - - - - - - - - - - - - - - - - - - - - - - - - - - - - - - - - - - - - - - %

\begin{document}
\sloppy

\maketitle
\vspace{25px}

\section{Aufgabe 1. Reflexionsaufgabe. Unterschiede zwischen Schule und Studium}

Die erste Woche im regulären Lehrbetrieb liegt bereits hinter euch und ihr konntet einen
ersten Einblick ins Studierendenleben und den Studienalltag gewinnen.

Was sind für Dich die wesentlichen Unterschiede zwischen
Schule und Studium? Gibt es aus Deiner Sicht auch Gemeinsamkeiten?

\vspace{10pt}
\hrule
\vspace{5pt}

Eine der ersten Sachen, die einem bereits vor Beginn der
Vorlesungen auffällt, ist die Verantwortung für das eigene Studium und die
Selbstorganisation des Lernprozesses. In der Schule wird meist alles für die
Schüler entschieden. An der Hochschule oder Universität ist man hingegen selbst
dafür verantwortlich, seinen Stundenplan zu erstellen, die passenden
Übungstermine auszuwählen und gegebenenfalls an Vorlesungen und Tutorien
teilzunehmen. Ich finde es besonders angenehm, dass ich selbst entscheiden
kann, zu welcher Zeit ich eine Veranstaltung besuche und ob es nötig ist, an
der Vorlesung teilzunehmen oder ob es ausreicht, die Unterlagen durchzusehen,
wenn ich mit dem jeweiligen Thema bereits vertraut bin. Auf diese Weise habe
ich die volle Kontrolle über meine Zeit und kann entscheiden, wie viel davon
ich für das Studium und andere Aktivitäten aufwenden möchte.

Zudem gibt es im Studium einen weiteren Unterschied: Während in der Schule
regelmäßig Klassenarbeiten, Tests und mündliche Leistungen bewertet werden,
hängt die Note im Studium oft von nur einer oder wenigen Prüfungen am Ende des
Semesters ab. Dadurch ist der Druck in der Prüfungsphase deutlich höher, was
eine kontinuierliche Vorbereitung über das gesamte Semester hinweg erfordert.

Trotz all dieser Unterschiede habe ich auch einige Gemeinsamkeiten zwischen
Schule und Studium bemerkt.

Sowohl in der Schule als auch im Studium ist der Unterricht häufig in einen
theoretischen und einen praktischen Teil unterteilt, wobei die Schüler zunächst
theoretisches Wissen zum Thema erwerben und das Erlernte dann in der Praxis
anwenden. Im Studium entspricht dies den Übungen oder Tutorien, die als
Ergänzung zur Vorlesung dienen.

Auch der Lernfortschritt wird in beiden Bildungssystemen durch Prüfungen oder
andere Leistungsnachweise gemessen. Sowohl in der Schule als auch im Studium
gibt es Tests, Klausuren oder schriftliche Arbeiten, die dazu dienen, das
erlernte Wissen zu überprüfen und zu bewerten. Der entscheidende Unterschied
liegt hierbei eher in der Häufigkeit und der Gewichtung der Prüfungen, aber das
Grundprinzip der Wissenskontrolle bleibt bestehen.

\section{Aufgabe 2. Dos and Don`ts im Studium, im Fachbereich, an der Hochschule}

Was sind aus Deiner Sicht Aspekte, die im Studium, im Fachbereich und an der Hochschule von
Studierenden erwartet werden und was sind Aspekte, die vermieden werden sollten?

Nenne zu jeder der sechs besprochenen Oberkategorien jeweils mindestens zwei
und maximal vier Beispiele für konkrete Dos und/oder Don`ts.

\vspace{10pt}
\hrule

\vspace{5pt}
\bgroup
\large\textbf{Studierfähigkeit \& Lernverhalten}
\egroup

\begin{multicols}{2}

\textbf{Dos:}

\begin{enumerate}[-]
    \item \textbf{Aktives Lernen:} aktiv mit dem Lernstoff auseinandersetzen, z.B.
        durch regelmäßiges Vor- und Nachbereiten von Vorlesungen, Erstellen
        eigener Notizen und das gezielte Bearbeiten von Übungsaufgaben.
    \item \textbf{Eigenständigkeit entwickeln:} selbstständig lernen, Informationen zu
        suchen und sich neues Wissen anzueignen, statt nur auf vorgegebene
        Inhalte zu vertrauen.
\end{enumerate}

\columnbreak

\textbf{Don`ts:}

\begin{enumerate}[-]
    \item \textbf{Passivität:} nur passiv in Vorlesungen zu sitzen und auf das Ende des
        Semesters zu warten, um dann mit dem Lernen zu beginnen.
\end{enumerate}

\end{multicols}

\vspace{2pt}
\hrule
\vspace{5pt}

\bgroup
\large\textbf{Studienorganisation \& Zeitmanagement}
\egroup

\begin{multicols}{2}

\textbf{Dos:}

\begin{enumerate}[-]
    \item \textbf{Frühzeitige Planung:} rechtzeitig zu Kursen und Prüfungen
        anmelden sowie einen Zeitplan für Lern- und Abgabefristen erstellen.
\end{enumerate}

\vfill
\null
\columnbreak

\textbf{Don`ts:}

\begin{enumerate}[-]
    \item \textbf{Unorganisiert sein:} Schlechte Planung führt zu einem
        Verpassen von Fristen oder übermäßigem Stress, wenn zu viele Aufgaben
        gleichzeitig anfallen.
    \item \textbf{Zeitdruck ignorieren:} Den Arbeitsaufwand für Prüfungen und
        Hausarbeiten zu unterschätzen und alles auf den letzten Drücker zu
        erledigen.
\end{enumerate}

\end{multicols}

\vspace{2pt}
\hrule
\vspace{5pt}

\bgroup
\large\textbf{Soziale Kompetenzen}
\egroup

\begin{multicols}{2}

\textbf{Dos:}

\begin{enumerate}[-]
    \item \textbf{Teamarbeit fördern:} In einigen Modulen wird die
        Zusammenarbeit mit Kommilitonen erwartet. Hier ist ein respektvoller
        und konstruktiver Umgang miteinander wichtig.
    \item \textbf{Netzwerke aufbauen:} Der Aufbau von Kontakten und Netzwerken
        mit Kommilitonen und Dozent*innen hilft im Studium und auch darüber
        hinaus.
\end{enumerate}

\columnbreak

\textbf{Don`ts:}

\begin{enumerate}[-]
    \item \textbf{Konflikte ignorieren:} Zwischenmenschliche Konflikte in
        Gruppenarbeiten nicht anzusprechen, kann zu schlechter Zusammenarbeit
        und schlechten Ergebnissen führen.
\end{enumerate}

\end{multicols}

\hrule
\vspace{5pt}

\bgroup
\large\textbf{Digitale Kompetenzen}
\egroup

\begin{multicols}{2}

\textbf{Dos:}

\begin{enumerate}[-]
    \item \textbf{Verwendung digitaler Tools:} Lernprozesse durch Nutzung
        digitaler Werkzeuge zu optimieren.
    \item \textbf{Datensicherheit und Datenschutz beachten:} sensibel mit
        persönlichen und urheberrechtlich geschützten Daten umgehen.
\end{enumerate}

\columnbreak

\textbf{Don`ts:}

\begin{enumerate}[-]
    \item \textbf{Unachtsam sein:} z.B. unsichere Quellen verwenden.
\end{enumerate}

\end{multicols}

\vspace{2pt}
\hrule
\vspace{5pt}

\bgroup
\large\textbf{Ethik \& Integrität}
\egroup

\begin{multicols}{2}

\textbf{Dos:}

\begin{enumerate}[-]
    \item \textbf{Ehrlichkeit und Fairness:} ehrlich arbeiten und nicht
        abschreiben, Plagiat vermeiden.
\end{enumerate}

\vfill
\null
\columnbreak

\textbf{Don`ts:}

\begin{enumerate}[-]
    \item \textbf{Betrug und Plagiate:} abschreiben, Ideen und Inhalte von
        anderen Personen in eigener Arbeit verwenden, ohne den Autor anzugeben.
    \item \textbf{Respektloses Verhalten:} Dozenten oder Kommilitonen
        beleidigen oder diskriminieren.
\end{enumerate}

\end{multicols}

\vspace{2pt}
\hrule
\vspace{5pt}

\bgroup
\large\textbf{Lernumfeld \& Fachbereich \& Hochschule}
\egroup

\begin{multicols}{2}

\textbf{Dos:}

\begin{enumerate}[-]
    \item \textbf{Teilnahme an Angeboten:} die von der Hochschule angebotenen
        Aktivitäten, wie Tutorien oder Seminare, nutzen.
    \item \textbf{Engagement im Fachbereich:} Erfahrungen durch die Teilnahme
        an Veranstaltungen oder Projekten im Fachbereich sammeln
\end{enumerate}

\columnbreak

\textbf{Don`ts:}

\begin{enumerate}[-]
    \item \textbf{Mangelndes Interesse zeigen:} Angebote der Hochschule oder
        des Fachbereichs ignorieren.
\end{enumerate}

\end{multicols}

\pagebreak
\section{Aufgabe 3. Gesamtüberblick Modulbelegung im WiSe 2024/25}

Sicher habt ihr in euren ersten Veranstaltungen bereits festgestellt, dass je
nach Modul unterschiedliche Lern- und Arbeitsmaterialien bereitgestellt /
eingesetzt werden und auch verschiedene Anforderungen an euch als Studierende
bestehen (also so etwas wie wöchentliche Abgabe eines bearbeiteten
Übungsblattes oder Ähnliches)

Um sich auf die Module und die darin enthaltenen Lehrveranstaltungen effizient
und effektiv vorzubereiten und im Blick zu haben, was als Nachbereitung in
diesem Modul sinnvoll ist, braucht es einen Gesamtüberblick über alle eure
Veranstaltungen.

\textit{Erstelle eine eigene Übersicht und befülle diese Übersicht mit den
Informationen zu allen Modulen, die Du in diesem Semester absolvierst und die
Du bisher aus Deinen ersten Lehrveranstaltungen mitnehmen konntest}.

\vspace{10pt}
\hrule

\begin{multicols}{2}

\textbf{Wahrscheinlichkeitstheorie und Statistik}

\begin{enumerate}[-]
    \item Arbeitsmaterialien: Skript, "Ubungsaufgaben
    \item enthaltene Organisationsformen: Vorlesung, "Ubung, Tutorium
    \item geforderte Studienleistung (PV): "Ubungsaufgaben abgeben/vorstellen
\end{enumerate}

\columnbreak

\textbf{Mathematische Grundlagen}

\begin{enumerate}[-]
    \item Arbeitsmaterialien: Skript, "Ubungsaufgaben, Tests auf Papier, SALT-Tests
    \item enthaltene Organisationsformen: Vorlesung, "Ubung, Tutorium
    \item geforderte Studienleistung (PV): "Ubungsaufgaben abgeben/vorstellen, mind. 50\% der Punkte in Tests
\end{enumerate}

\end{multicols}

\vspace{10pt}

\begin{multicols}{2}

\textbf{Schlüsselkompetenzen}

\begin{enumerate}[-]
    \item Arbeitsmaterialien: Folien, "Ubungsaufgaben
    \item enthaltene Organisationsformen: Vorlesung
    \item geforderte Studienleistung (PV): "Ubungsaufgaben abgeben/vorstellen
\end{enumerate}

\columnbreak

\textbf{Technische Informatik}

\begin{enumerate}[-]
    \item Arbeitsmaterialien: Folien, "Ubungsaufgaben, Videoaufzeichnungen
    \item enthaltene Organisationsformen: Vorlesung, "Ubung, Tutorium
    \item geforderte Studienleistung (PV): mind. 60\% der Punkte in Tests
\end{enumerate}

\end{multicols}

\vspace{10pt}

\begin{multicols}{2}

\textbf{Systemadministration}

\begin{enumerate}[-]
    \item Arbeitsmaterialien: Folien, "Ubungsaufgaben
    \item enthaltene Organisationsformen: Vorlesung, "Ubung, Tutorium
    \item geforderte Studienleistung (PV): "Ubungsaufgaben abgeben/vorstellen
\end{enumerate}

\columnbreak

\textbf{Einf"uhrung in die Programmierung}

\begin{enumerate}[-]
    \item Arbeitsmaterialien: Skript, "Ubungsaufgaben, B"ucher
    \item enthaltene Organisationsformen: Vorlesung, "Ubung
    \item geforderte Studienleistung (PV): "Ubungsaufgaben abgeben/vorstellen
\end{enumerate}

\end{multicols}

\end{document}
