\documentclass[10pt, oneside]{article}
\usepackage[a4paper, total={5.5in, 9in}]{geometry}
\usepackage[ngerman]{babel}
\usepackage{import}

\import{../.texit/include}{preamble}

\title{Angewandte Logik\\[15pt]\Large{Hausaufgabenblatt 1-2}\\[10pt]\Large{SoSe 2025}}
\author{Volodymyr But\\[5pt][Matrikel-Nr.: 982324]\\[10pt]Hochschule Trier}
\date{}

% - - - - - - - - - - - - - - - - - - - - - - - - - - - - - - - - - - - - - - %

\begin{document}

\maketitle
\vspace{25px}

\section{Aufgabe 1}

Gegeben sei folgende Formel: $((A \land B) \lor C) \land (\lnot B \lor D)$
\begin{enumerate}[(a)]
    \item Formen Sie die Formel um in KNF

        $((A \land B) \lor C) \land (\lnot B \lor D)$ = $(A \lor C) \land (B \lor C) \land (\lnot B \lor D)$

    \item Formen Sie die Formel um in DNF

        $\begin{aligned}
            ((A \land B) \lor C) \land (\lnot B \lor D) &= ((A \land B) \land (\lnot B \lor D)) \lor (C \land (\lnot B \lor D)) = \\
                                                        &= (A \land B \land \lnot B) \lor (A \land B \land D) \lor (C \land \lnot B) \lor (C \land D)
        \end{aligned}$

    \item In welcher logischen Beziehung stehen die unter a) und b) berechneten Formeln?

        Beide Formen sind logisch äquivalent zur Ursprungsformel und zueinander.
\end{enumerate}

\section{Aufgabe 2}

Begr"unden Sie, warum die folgende Regel des Resolutionskalk"uls korrekt ist.
\[
\frac{
    \begin{array}{c}
        \{L\} \\
        \{\neg L, \neg K, M\}
    \end{array}
}{\{\neg K, M\}}
\]

Aus zwei Klauseln $C_1 = A \cup \{L\}$ und $C_2 = B \cup \{\lnot L\}$ kann man
die Resolvente $C = A \cup B$ ableiten. Dabei ist $L$ ein Literal und $\lnot L$
sein Komplement.

Die Klauseln sind
\begin{itemize}
    \item $\{L\}$,
    \item $\{\lnot L, \lnot K, M\}$.
\end{itemize}
Gem"a{\ss} der Definition werden $L$ und $\lnot L$ gel"oscht und die "ubrigen
Literale vereint. Das ergibt
\begin{equation*}
    \{\lnot K, M\} \ \ \ \square
\end{equation*}

\pagebreak
\section{Aufgabe 3}

Eine Formel ist erf"ullt durch eine Variablenbelegung, wenn in jeder Klausel
\textbf{mindestens ein} Literal wahr ist in dieser Variablenbelegung.

Eine Formel ist nicht erf"ullt durch eine Variablenbelegung, wenn in einer
Klausel \textbf{alle} Literale \textbf{falsch} sind unter der gegebenen
Variablenbelegung.

Eine Signatur ist eine \textbf{nicht leere}, \textbf{endliche} Menge von
\textbf{Bezeichnern}.

Ein Kalk"ul ist vollst"andig, wenn f"ur beliebige Formeln F und G gilt: \\
falls $F \models G$ gilt, dann muss auch $F \vdash G$ gelten.

Ein Kalk"ul besteht aus einer Menge von \textbf{Axiomen} und einer Menge von \textbf{Schlussregeln}.

\end{document}
