\documentclass[10pt, oneside]{article}
\usepackage[a4paper, total={5.5in, 9in}]{geometry}
\usepackage[ngerman]{babel}
\usepackage{import}
\usepackage{caption}

\import{../.texit/include}{preamble}

\title{Angewandte Logik\\[15pt]\Large{Hausaufgabenblatt 1-3}\\[10pt]\Large{SoSe 2025}}
\author{Volodymyr But\\[5pt][Matrikel-Nr.: 982324]\\[10pt]Hochschule Trier}
\date{}


% - - - - - - - - - - - - - - - - - - - - - - - - - - - - - - - - - - - - - - %

\begin{document}

\maketitle
\vspace{25px}

\section{Aufgabe 1}

Für die Aufgabenteile a), b), c) ist die folgende Formel relevant:
\begin{equation*}
    (A \lor \lnot B) \land (A \lor \lnot C) \land (B \lor \lnot E) \land (B \lor \lnot D) \land (D \lor E \lor C)
\end{equation*}
\begin{enumerate}[(a)]
    \item Formen Sie die Formel um in Klauselform
        \begin{equation*}
            \text{P} = \{\{A, \lnot B\}, \{A, \lnot C\}, \{B, \lnot E\}, \{B, \lnot D\}, \{D, E , C\}\}
        \end{equation*}

    \item "Uberpr"ufen Sie mit Hilfe des DPLL-Verfahrens, ob die gegebene
        Formel erf"ullbar oder unerf"ullbar ist. Nehmen Sie weiterhin folgende
        Reihenfolge der zu w"ahlenden Literale an: A, B, C.

        Erf"ullbar (siehe Abbildung~\ref{fig:1.b})

    \item "Uberpr"ufen Sie mit Hilfe des DPLL-Verfahrens, ob die gegebene
        Formel erf"ullbar oder unerf"ullbar ist. Nehmen Sie weiterhin folgende
        Reihenfolge der zu w"ahlenden Literale an: $\lnot\text{C}, \lnot\text{A}, \lnot\text{B}$.

        Erf"ullbar (siehe Abbildung~\ref{fig:1.c})

    \item Was schlie{\ss}en Sie aus der L"ange der in (b) und (c) erstellten Beweise?

        Die unterschiedliche Länge der Beweise in (b) und (c) zeigt, dass die
        Wahl der Literale im DPLL-Algorithmus die Effizienz des Verfahrens
        wesentlich beeinflussen kann.

    \item Geben Sie ein Modell für die oben gegebene Formel an.

        Aus (b): $[\text{A} \rightarrow \top, \text{B} \rightarrow \top, \text{C} \rightarrow \top]$

        Aus (c): $[\text{C} \rightarrow \bot, \text{A} \rightarrow \top, \text{B} \rightarrow \top, \text{D} \rightarrow \top]$

\end{enumerate}

\begin{figure}[p]
    \caption{}
    \label{fig:1.b}
\begin{verbatim}
DPLL(F = {{A, !B}, {A, !C}, {B, !E}, {B, !D}, {D, E, C}}, [])
  1. F enthält keine Klausel der Form {L}.
  2. F != {} und {{}} ist nicht in F enthalten.
  3. simplify(F, L := A) -> {{B, !E}, {B, !D}, {D, E, C}}
  4. DPLL(F = {{B, !E}, {B, !D}, {D, E, C}})
       1. F enthält keine Klausel der Form {L}.
       2. F != {} und {{}} ist nicht in F enthalten.
       3. simplify(F, L := B) = {{D, E, C}}
       4. DPLL(F = {{D, E, C}})
            1. F enthält keine Klausel der Form {L}.
            2. F != {} und {{}} ist nicht in F enthalten.
            3. simplify(F, L := C) = {}
            5. DPLL(F = {})
                 1. F enthält keine Klausel der Form {L}.
                 2. F == {}. Return True
               DPLL(F = {}) == True. Return True.
          DPLL(F = {{D, E, C}}) == True. Return True.
     DPLL(F = {{B, !E}, {B, !D}, {D, E, C}}) == True. Return True.
\end{verbatim}
\end{figure}

\FloatBarrier

\begin{figure}[p]
    \caption{}
    \label{fig:1.c}
\begin{verbatim}
DPLL(F = {{A, !B}, {A, !C}, {B, !E}, {B, !D}, {D, E, C}}, [])
  1. F enthält keine Klausel der Form {L}.
  2. F != {} und {{}} ist nicht in F enthalten.
  3. simplify(F, L := !C) -> {{A, !B}, {B, !E}, {B, !D}, {D, E}}
  4. DPLL(F = {{A, !B}, {B, !E}, {B, !D}, {D, E}})
       1. F enthält keine Klausel der Form {L}.
       2. F != {} und {{}} ist nicht in F enthalten.
       3. simplify(F, L := !A) -> {{!B}, {B, !E}, {B, !D}, {D, E}}
       4. DPLL(F = {{!B}, {B, !E}, {B, !D}, {D, E}})
            1. F enthält keine Klausel der Form {L}.
            2. F != {} und {{}} ist nicht in F enthalten.
            3. simplify(F, L := !B) -> {{!E}, {!D}, {D, E}}
            4. DPLL(F = {{!E}, {!D}, {D, E}})
                 1. F enthält {!E}. simplify(F, L := !E) -> {{!D}, {D}}
                 2. F enthält {!D}. simplify(F, L := !D) -> {{}}
                 3. {{}} ist in F enthalten. Return False.
               DPLL(F = {{!E}, {!D}, {D, E}}) != True.
            5. simplify(F, L := B) -> {{}, {D, E}}
            6. DPLL(F = {{}, {D, E}})
                 1. F enthält keine Klausel der Form {L}.
                 2  {{}} ist in F enthalten. Return False.
               DPLL(F = {{}, {D, E}}) != True. Return False.
          DPLL(F = {{!B}, {B, !E}, {B, !D}, {D, E}}) != True.
       5. simplify(F, L := A) -> {{B, !E}, {B, !D}, {D, E}}
       6. DPLL(F = {{B, !E}, {B, !D}, {D, E}})
            1. F enthält keine Klausel der Form {L}.
            2. F != {} und {{}} ist nicht in F enthalten.
            3. simplify(F, L := !B) -> {{!E}, {!D}, {D, E}}
            4. DPLL(F = {{!E}, {!D}, {D, E}})
                 1. F enthält {!E}. simplify(F, L := !E) -> {{!D}, {D}}
                 2. F enthält {!D}. simplify(F, L := !D) -> {{}}
                 3. {{}} ist in F enthalten. Return False.
               DPLL(F = {{!E}, {!D}, {D, E}}) != True
            5. simplify(F, L := B) -> {{D, E}}
            6. DPLL(F = {{D, E}})
                 1. F enthält keine Klausel der Form {L}.
                 2. F != {} und {{}} ist nicht in F enthalten.
                 3. simplify(F, L := D) -> {}.
                 4. DPLL(F = {})
                      1. F enthält keine Klausel der Form {L}.
                      2. F == {}. Return True.
                    DPLL(F = {}) == True. Return True.
               DPLL(F = {{D, E}}) == True. Return True.
          DPLL(F = {{B, !E}, {B, !D}, {D, E}}) == True. Return True.
     DPLL(F = {{A, !B}, {B, !E}, {B, !D}, {D, E}}) == True. Return True.
\end{verbatim}
\end{figure}

\end{document}
