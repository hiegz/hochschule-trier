\documentclass[10pt, oneside]{article}
\usepackage[a4paper, total={5.5in, 9in}]{geometry}
\usepackage[ngerman]{babel}
\usepackage{import}
\usepackage{caption}

\import{../.texit/include}{preamble}

\title{Angewandte Logik\\[15pt]\Large{Hausaufgabenblatt 2}\\[10pt]\Large{SoSe 2025}}
\author{Volodymyr But\\[5pt][Matrikel-Nr.: 982324]\\[10pt]Hochschule Trier}
\date{}

% - - - - - - - - - - - - - - - - - - - - - - - - - - - - - - - - - - - - - - %

\begin{document}

\maketitle
\vspace{25px}

\section{Aufgabe 1}

Untersuchen Sie die Formel $\forall y \: (\text{R}(h(y, a)) \land b) \lor \exists z \: \text{P}(g(h(y, a), x, h(z, f(y))))$

\begin{enumerate}[(a)]
    \item Welche Pr"adikatssymbole kommen vor?

        $\{\text{R}, \text{P}\}$

    \item Welche Funktionssymbole kommen vor?

        $\{f, g, h\}$

    \item Welche Variablen kommen vor?

        $\{a, b, x, y, z\}$

\end{enumerate}

\section{Aufgabe 2}

\begin{enumerate}[(a)]
    \item Welche Variablen kommen jeweils in den folgenden Formeln gebunden vor?
        \begin{enumerate}[i)]
            \item $\exists w \: \forall x \: \exists y \: \forall z \: \text{P}(c, f(w), x, y, z)$

                $\{w, x, y, z\}$

            \item $\forall x \: \text{P}(x, y) \land \text{Q}(x) = (\forall x \: \text{P}(x, y)) \land \text{Q}(x)$

                $\{x\}$ ($x$ ist gebunden in $\forall x \: \text{P}(x, y)$ und frei in $\text{Q}(x)$)

            \item $\forall x \: (\text{P}(x, y) \land \text{Q}(x))$

                $\{x\}$
        \end{enumerate}

    \item Welche Variablen kommen in den unter (a) gegebenen Formeln frei vor?
        \begin{enumerate}[i)]
            \item $\{c\}$
            \item $\{x, y\}$ ($x$ ist gebunden in $\forall x \: \text{P}(x, y)$ und frei in $\text{Q}(x)$)
            \item $\{y\}$
        \end{enumerate}
\end{enumerate}

\section{Aufgabe 3}

Überprüfen Sie mit Hilfe des DPLL-Verfahrens, ob folgende Formel erfüllbar ist
\begin{equation*}
    (Q \Rightarrow R) \land (R \Rightarrow (P \land Q)) \land (P \Rightarrow (Q \lor R)) \land (\lnot(P \Leftrightarrow Q))
\end{equation*}
Zun"achst stellen wir die angegebene Formel in Klauselnormalform dar:
\begin{align*}
    &\hspace{13pt}(Q \Rightarrow R) \land (R \Rightarrow (P \land Q)) \land (P \Rightarrow (Q \lor R)) \land (\lnot(P \Leftrightarrow Q)) = \\
    &=            (\lnot Q \lor R) \land (\lnot R \lor (P \land Q)) \land (\lnot P \lor Q \lor R) \land (\lnot((\lnot P \lor Q) \land (\lnot Q \lor P))) = \\
    &=            (\lnot Q \lor R) \land (\lnot R \lor P) \land (\lnot R \lor Q) \land (\lnot P \lor Q \lor R) \land (\lnot(\lnot P \lor w) \lor \lnot (\lnot Q \lor P)) = \\
    &=            (\lnot Q \lor R) \land (\lnot R \lor P) \land (\lnot R \lor Q) \land (\lnot P \lor Q \lor R) \land ((P \land \lnot Q) \lor (Q \land \lnot P)) = \\
    &=            (\lnot Q \lor R) \land (\lnot R \lor P) \land (\lnot R \lor Q) \land (\lnot P \lor Q \lor R) \land (P \lor Q) \land (P \lor \lnot P) \land (\lnot Q \lor Q) \land (\lnot Q \lor \lnot P)
\end{align*}
Also
\begin{equation*}
    \{\{\lnot Q, R\}, \{\lnot R, P\}, \{\lnot R, Q\}, \{\lnot P, Q, R\}, \{P, Q\}, \{P, \lnot P\}, \{\lnot Q, Q\}, \{\lnot Q, \lnot P\}\}
\end{equation*}
DPLL:
\begin{enumerate}[1.]
    \item $L := Q$ \\[5pt]
        $F := \{\{R\}, \{\lnot R, P\}, \{P, \lnot P\}, \{\lnot P\}\}$ \\[5pt]
        $F := \{\{P\}, \{P, \lnot P\}, \{\lnot P\}\}$ \\[5pt]
        $F := \{\{\}\}$ \\[5pt]

    \item $L := \lnot Q$ \\[5pt]
        $F := \{\{\lnot R, P\}, \{\lnot R\}, \{\lnot P, R\}, \{P\}, \{P, \lnot P\}\}$ \\[5pt]
        $F := \{\{\lnot R\}, \{R\}\}$ \\[5pt]
        $F := \{\{\}\}$ \\[5pt]
\end{enumerate}
$\Rightarrow F \: \text{ist nicht erf"ullbar}$

\end{document}
