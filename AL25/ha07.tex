\documentclass[10pt, oneside]{article}
\usepackage[a4paper, total={5.5in, 9in}]{geometry}
\usepackage[ngerman]{babel}
\usepackage{import}
\usepackage{caption}

\import{../.texit/include}{preamble}

\title{Angewandte Logik\\[15pt]\Large{Hausaufgabenblatt 5}\\[10pt]\Large{SoSe 2025}}
\author{Volodymyr But\\[5pt][Matrikel-Nr.: 982324]\\[10pt]Hochschule Trier}
\date{}

% - - - - - - - - - - - - - - - - - - - - - - - - - - - - - - - - - - - - - - %

\begin{document}

\maketitle
\vspace{25px}

\section{Aufgabe 1}

\begin{enumerate}[(a)]
    \item
        $\sigma_1(\text{K}1) = \{\lnot\text{zugelassen}(\text{hans}, \text{logik}), \lnot\text{gut\_vorbereitet}(\text{hans}, \text{logik}), \text{besteht\_pruefung}(\text{hans}, \text{logik})\}$ \\[5pt]
        $\sigma_2(\text{K}2) = \{\lnot\text{zugelassen}(\text{hans}, \text{logik}), \lnot\text{gut\_vorbereitet}(\text{hans}, \text{logik})\}$ \\[5pt]
        $\sigma_3(\text{K}1) = \{\lnot\text{zugelassen}(\text{hans}, \text{logik}), \lnot\text{gut\_vorbereitet}(\text{hans}, \text{logik}), \text{besteht\_pruefung}(\text{hans}, \text{logik})\}$ \\[5pt]
        $\sigma_3(\text{K}2) = \{\lnot\text{zugelassen}(\text{hans}, \text{logik}), \lnot\text{gut\_vorbereitet}(\text{hans}, \text{logik})\}$ \\[5pt]

    \item
        $\sigma = \{x \leftarrow f(g(y, a)), z \leftarrow b, v \leftarrow w\}$ \\[5pt]
        $\sigma(P(f(x), g(v, f(f(x))))) = P(f(f(g(y, a))), g(w, f(f(f(g(y, a))))))$
\end{enumerate}

\section{Aufgabe 2}

Sind die folgenden Termpaare unifizierbar? Bestimmen Sie falls möglich den
mgu oder geben Sie einen Grund für die Nichtunifizierbarkeit an.

\begin{enumerate}[(a)]
    \item $f(h(x), x, h(x))$ und $f(z, h(w), h(z))$

        Angenommen, es existiert eine mgu $\sigma$ der beiden Terme. Dann gilt:
        \begin{equation*}
            \sigma(h(x)) = \sigma(z) \text{ und } \sigma(h(x)) = \sigma(h(z))
        \end{equation*}
        Daraus folgt
        \begin{equation*}
            \sigma(z) = \sigma(h(z))
        \end{equation*}
        Dies bedeutet, dass $z$ in seiner eigenen Substitution vorkommt. Somit
        ist sie nicht idempotent und kann keine g"ultige Unifikatorl"osung
        sein.

        Also sind die Terme nicht unifizierbar.

    \item $f(g(x, w), g(w, h(a)), g(h(u), x))$ und $f(y, z, z)$

        $\{ x \leftarrow h(a), y \leftarrow g(h(a), h(u)), w \leftarrow h(u), z \leftarrow g(h(u), h(a))\}$

    \item $f(w, g(w, h(a)), g(h(x), x))$ und $f(h(y), z, z)$

        $\{ x \leftarrow h(a), y \leftarrow h(a), w \leftarrow h(h(a)), z \leftarrow g(h(h(a)), h(a)) \}$

    \item $f(g(x, w), g(w, h(b)), g(w, x))$ und $f(z, z, z)$

        $\{ x \leftarrow h(b), w \leftarrow h(b), z \leftarrow g(h(b), h(b)) \}$
\end{enumerate}


\end{document}
