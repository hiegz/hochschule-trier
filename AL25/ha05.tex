\documentclass[10pt, oneside]{article}
\usepackage[a4paper, total={5.5in, 9in}]{geometry}
\usepackage[ngerman]{babel}
\usepackage{import}
\usepackage{caption}

\import{../.texit/include}{preamble}

\title{Angewandte Logik\\[15pt]\Large{Hausaufgabenblatt 3}\\[10pt]\Large{SoSe 2025}}
\author{Volodymyr But\\[5pt][Matrikel-Nr.: 982324]\\[10pt]Hochschule Trier}
\date{}

% - - - - - - - - - - - - - - - - - - - - - - - - - - - - - - - - - - - - - - %

\begin{document}

\maketitle
\vspace{25px}

\section{Aufgabe 1}

Untersuchen Sie die folgenden Formeln auf Erfüllbarkeit, in dem Sie versuchen
eine passende Interpretation in PL1 anzugeben.
\begin{enumerate}[(a)]
    \item $<(t_1, t_2)$ wobei $t_1$ und $t_2$ Terme sind

        $U = \{0, 1\}$ \\[5pt]
        $<^I :\: <(x, y)^I = x < y$ \\[5pt]
        $t_1^{\beta} = 0$ \\[5pt]
        $t_2^{\beta} = 1$

        Dann gilt $<(t_1, t_2)^I = (0 < 1) = \top$. Also $<(t_1, t_2)$ ist erf"ullbar.

    \item $\forall x \ \exists y <(x, y)$

        $U = \mathbb{N}$ \\[5pt]
        $<^I :\: <(x, y)^I = x < y$

        Da $\mathbb{N}$ unendlich ist, gilt es f"ur $U$, dass
        \begin{equation*}
            \forall x \ (\exists y \:\:\: y = x + 1) \:\:\: \land \:\:\: x < x + 1
        \end{equation*}
        Daraus folgt, dass
        $(\forall x\ \exists y < (x, y))^I = \top$. Also $\forall x\ \exists y < (x, y)$ ist erf"ullbar.

    \item $\forall x \ \exists y >(x, y)$

        $U = \mathbb{Z}$ \\[5pt]
        $>^I :\: >(x, y)^I = x > y$

        Da $\mathbb{Z}$ unendlich ist, gilt es f"ur $U$, dass
        \begin{equation*}
            \forall x \ (\exists y \:\:\: y = x - 1) \:\:\: \land \:\:\: x > x - 1
        \end{equation*}
        Daraus folgt, dass
        $(\forall x\ \exists y > (x, y))^I = \top$. Also $\forall x\ \exists y > (x, y)$ ist erf"ullbar.

    \item $\forall x \ \exists y\ f(x, y) = x$

        $U = \mathbb{Z}$ \\[5pt]
        $f^I : f(x, y) = x \cdot y$

        F"ur $U$ gilt
        \begin{equation*}
            \forall x \ \exists y\  y = e = 1 \land f(x, e) = x \cdot e = x \cdot y = x \cdot 1 = x
        \end{equation*}
        Daraus folgt, dass $(\forall x \ \exists y\ f(x, y) = x)^I = \top$.
        Also $\forall x \ \exists y\ f(x, y) = x$ ist erf"ullbar.

\end{enumerate}

\end{document}
