\section{Aufgabe 62}

Zeigen Sie durch vollst"andige Induktion, dass f"ur $n \in \mathbb{N}$ folgendes gilt:
\begin{equation*}
    \sum_{\nu = 1}^n\nu^3 = \dfrac{1}{4}n^2(n + 1)^2
\end{equation*}
\begin{enumerate}[i)]
    \item \textbf{Induktionsanfang:} Zeigen wir, dass die Aussage f"ur $n = 1$ gilt
        \begin{align*}
            1^3 &= \dfrac{1}{4} \cdot 1^2(1 + 1)^2 \\[5pt]
            1 &= \dfrac{1}{4} \cdot 4 \\[5pt]
            1 &= 1
        \end{align*}
    \item \textbf{Induktionsannahme:} Wir nehmen an, dass die Aussage f"ur ein allgemeines $n_k$
        \begin{align*}
            1^3 + 2^3 + ... + k^3 = \dfrac{1}{4} \cdot k^2(k + 1)^2
        \end{align*}
    \item \textbf{Induktionsschritt:} Zeigen wir, dass unter der
        Induktionsannahme f"ur $n_k$, folgt, dass diese Aussage auch f"ur $n_{k
        + 1}$ gilt
        \begin{align*}
            1^3 + 2^3 + ... + k^3 + (k + 1)^3 &= \dfrac{1}{4} \cdot (k + 1)^2(k + 2)^2 \\[5pt]
            \dfrac{1}{4} \cdot k^2(k + 1)^2 + (k + 1)^3 &= \dfrac{1}{4} \cdot (k + 1)^2(k + 2)^2 \\[5pt]
            (k + 1)^2 \cdot \left(\dfrac{1}{4} \cdot k^2 + (k + 1)\right) &= \dfrac{1}{4} \cdot (k + 1)^2(k + 2)^2 \\[5pt]
            (k + 1)^2 \cdot \left(\dfrac{1}{4} \cdot k^2 + \dfrac{1}{4} \cdot (4k + 4)\right) &= \dfrac{1}{4} \cdot (k + 1)^2(k + 2)^2 \\[5pt]
            \dfrac{1}{4} \cdot (k + 1)^2 (k^2 + 4k + 4) &= \dfrac{1}{4} \cdot (k + 1)^2(k + 2)^2 \\[5pt]
            \dfrac{1}{4} \cdot (k + 1)^2 (k + 2)^2 &= \dfrac{1}{4} \cdot (k + 1)^2(k + 2)^2 \\[5pt]
        \end{align*}
\end{enumerate}
