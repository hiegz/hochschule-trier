\documentclass[10pt, oneside]{article}
\usepackage[a4paper, total={5.5in, 9in}]{geometry}
\usepackage[ngerman]{babel}

\usepackage{blindtext}
\usepackage{titling}
\usepackage{titlesec}
\usepackage{amsmath}
\usepackage[hidelinks]{hyperref}
\usepackage{parskip}
\usepackage{graphicx}

\setlength{\droptitle}{-3cm}

\titleformat{\section}
    {\normalfont\Large\bfseries}{}{0pt}{}

\let\oldsection\section
\renewcommand{\section}{
  \renewcommand{\theequation}{\thesection.\arabic{equation}}
  \oldsection}
\let\oldsubsection\subsection
\renewcommand{\subsection}{
  \renewcommand{\theequation}{\thesubsection.\arabic{equation}}
  \oldsubsection}


\title{Mathematische Grundlagen\\[10pt]\Large{WiSe 2024/25}\\[15pt]\Large{L{\"o}sungen zu den Aufgaben 47, 51 und 60}}
\author{Volodymyr But\\[10pt]Hochschule Trier}
\date{}

% - - - - - - - - - - - - - - - - - - - - - - - - - - - - - - - - - - - - - - %

\begin{document}

\maketitle
\vspace{25px}

\section{Aufgabe 47}
\begin{enumerate}
    \item $[3]_8 + [9]_8 + [-1]_8 = [3]_8 + [1]_8 + [7]_8 = [11]_8 = [3]_8$
    \item $[3]_5 + [1]_5 + [10]_5 = [3]_5 + [1]_5 + [0]_5 = [4]_5$
    \item $[7]_7 \cdot [6]_7 \cdot [-1]_7 = [0]_7 \cdot [6]_7 \cdot [6]_7 = [0]_7$
    \item $([6]_{13} \cdot [-5]_{13}) + [-3]_{13} = ([6]_{13} \cdot [8]_{13}) + [10]_{13} = [58]_{13} = [6]_{13}$
    \item $([8]_{-2} \cdot [6]_{-2}) + [-1]_{-2} = ([0]_{-2} \cdot [0]_{-2}) + [-1]_{-2} = [1]_{-2}$
\end{enumerate}

\section{Aufgabe 51}

\begin{enumerate}
    \item Geben Sie zu jedem Element in $\mathbb{Z}_p \setminus \{[0]_p\}$ das
        inverse Element bez"uglich der Multiplikation (in
        Standardrepr"asentation) an, f"ur $p = 5$.
        \begin{equation*}
            \begin{array}{rclcl}
                \mathbb{Z}_5 \setminus \{[0]_5\} &=& \{1, 2, 3, 4\}&& \\[10pt]
                [1]_5 \cdot [1]_5 &=& [1]_5 &:& [1]_5 \\[5pt]
                [2]_5 \cdot [3]_5 &=& [6]_5 = [1]_5 &:& [3]_5 \\[5pt]
                [3]_5 \cdot [2]_5 &=& [6]_5 = [1]_5 &:& [2]_5 \\[5pt]
                [4]_5 \cdot [4]_5 &=& [16]_5 = [1]_5 &:& [4]_5
            \end{array}
        \end{equation*}
    \item Zeigen Sie, dass $[p - 1]^{-1}_p = [p - 1]_p$ gilt.
        \begin{equation*}
            \begin{array}{rrcl}
                    &[p - 1]_p \cdot [x]_p &=& [1]_p \\[5pt]
                \iff&([p]_p - [1]_p) \cdot [x]_p &=& [1]_p \\[5pt]
                \iff&([0]_p - [1]_p) \cdot [x]_p &=& [1]_p \\[5pt]
                \iff&[-1]_p \cdot [x]_p &=& [1]_p \\[5pt]
                \iff&[x]_p &=& [-1]_p \\[5pt]
                \iff&[x]_p &=& [p - 1]_p \quad \text{\underline{q.e.d.}}
             \end{array}
        \end{equation*}
\end{enumerate}

\section{Aufgabe 60}
\begin{enumerate}[(a)]
    \item $\begin{aligned}[t]
            \prod_{i = 1}^{n} (a_i \cdot b_{n - i + 1}) &= a_1 \cdot b_{n} \cdot a_2 \cdot b_{n - 1} \cdot ... \cdot a_{n} \cdot b_{1} = (a_1 \cdot a_2 \cdot ... \cdot a_n) \cdot (b_1 \cdot b_2 \cdot ... \cdot b_n) = \\
                                                        &= \prod_{i = 1}^n(a_i) \cdot \prod_{i = 1}^2(b_i) \quad \text{\underline{q.e.d.}}
        \end{aligned}$
    \item $\begin{aligned}[t]
            \sum_{i = 0}^n((a_i + b_i)^2) &= (a_1 + b_1)^2 + (a_2 + b_2)^2 + ... + (a_n + b_n)^2 = \\
                                          &= (a_1^2 + 2a_1b_1 + b_1^2) + (a_2^2 + 2a_2b_2 + b_2^2) + ... + (a_n^2 + 2a_nb_n + b_n^2) = \\[5pt]
                                          &= (a_1^2 + a_2^2 + ... + a_n^2) + 2(a_1b_1 + a_2b_2 + ... + a_nb_n) + (b_1^2 + b_2^2 + ... + b_n^2) = \\[5pt]
                                          &= \sum_{i = 1}^n(a_i^2) + 2\sum_{i = 1}^n(a_ib_i) + \sum_{i = 1}^n(b_i^2) = \\[5pt]
                                          &= \sum_{i = 1}^n(a_i^2 + 2a_ib_i + b_i^2) \neq (\sum_{i = 1}^n(a_i) + \sum_{i = 1}^n(b_i))^2
        \end{aligned}$
\end{enumerate}

\end{document}
