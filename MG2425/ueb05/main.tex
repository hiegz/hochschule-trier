\documentclass[10pt, oneside]{article}
\usepackage[a4paper, total={5.5in, 9in}]{geometry}
\usepackage[ngerman]{babel}

\usepackage{blindtext}
\usepackage{titlesec}
\usepackage{amsmath}
\usepackage[hidelinks]{hyperref}
\usepackage{parskip}
\usepackage{graphicx}
\usepackage{longtable}
\usepackage[shortlabels]{enumitem}
\usepackage{multirow}
\usepackage{nccmath}
\usepackage{rotating}
\usepackage{makecell}
\usepackage{multicol}
\usepackage{capt-of}
\usepackage{csquotes}
\usepackage{amsfonts}
\usepackage{caption}

\captionsetup[table]{position=bottom}

\titleformat{\section}
    {\normalfont\Large\bfseries}{}{0pt}{}

\let\oldsection\section
\renewcommand{\section}{
  \renewcommand{\theequation}{\thesection.\arabic{equation}}
  \oldsection}
\let\oldsubsection\subsection
\renewcommand{\subsection}{
  \renewcommand{\theequation}{\thesubsection.\arabic{equation}}
  \oldsubsection}

\makeatletter
\renewcommand{\maketitle}{
    \bgroup
    \centering
    \par\LARGE\@title  \\[20pt]
    \par\large\@author \\[10pt]
    \par\large\@date
    \par
    \egroup
}
\makeatother


\title{Mathematische Grundlagen\\[10pt]\Large{WiSe 2024/25}\\[15pt]\Large{L{\"o}sungen zu den Aufgaben 39, 42, 44 und 45}}
\author{Volodymyr But}
\date{Hochschule Trier}

% - - - - - - - - - - - - - - - - - - - - - - - - - - - - - - - - - - - - - - %

\begin{document}
\sloppy

\maketitle
\vspace{25px}

\section{Aufgabe 39}

Der Graph einer Abbildung $f: X \rightarrow Y$ ist laut Definition eine
Teilmenge von $X \times Y$ und damit eine Relation. Pr"ufen Sie welche
Eigenschaften aus Definition 3.3 (im Skript) auf
\begin{itemize}
    \item den Graph einer beliebigen Abbildung,
    \item den Graph einer injektiven Abbildung,
    \item den Graph einer surjektiven Abbildung
\end{itemize}
zutreffen.
\begin{enumerate}
    \item Sei $f: X \rightarrow Y$ eine Abbildung definiert durch $f(X) :=
        \{x^2 : x \in X\} (\subset Y)$ und $G := \{(x, f(x)) : x \in X\}
        (\subset X \times Y)$ der Graph dieser Abbildung. Diese sind durch die
        folgende Tabelle dargestellt:
        \begin{table*}[h]
            \centering
            \begin{tabular}{c|c|c}
                $X$ & $Y$ & $G$ \\
                \hline
                -2 & 4 & (-2, 4) \\
                -1 & 1 & (-1, 1) \\
                0  & 0 & (0,  0) \\
                1  & 1 & (1,  1) \\
                2  & 4 & (2,  4) \\
                   & 5 &
            \end{tabular}
        \end{table*}\\
        Der Graph $G$ hei{\ss}t also:
        \begin{itemize}
            \item \textbf{linkstotal}, weil gilt: $\forall x \in X, \exists y \in Y : (x, y) \in G$
            \item \textbf{rechtseindeutig}, weil gilt: $\forall x \in X$ und $\forall y,z \in Y : (x, y) \in G \land (x, z) \in G \Rightarrow y = z$
        \end{itemize}
    \item Sei $f: X \rightarrow Y$ eine Abbildung definiert durch $f(X) :=
        \{x + 1 : x \in X\} (\subset Y)$ und $G := \{(x, f(x)) : x \in X\}
        (\subset X \times Y)$ der Graph dieser Abbildung. Diese sind durch die
        folgende Tabelle dargestellt:
        \begin{table*}[h]
            \centering
            \begin{tabular}{c|c|c}
                $X$ & $Y$ & $G$ \\
                \hline
                0  & 1 & (0,  1) \\
                1  & 2 & (1,  2) \\
                2  & 3 & (2,  3) \\
                   & 4 &
            \end{tabular}
        \end{table*}

        \pagebreak
        Der Graph $G$ hei{\ss}t also:
        \begin{itemize}
            \item \textbf{linkstotal}, weil gilt: $\forall x \in X, \exists y \in Y : (x, y) \in G$
            \item \textbf{rechtseindeutig}, weil gilt: $\forall x \in X$ und $\forall y,z \in Y : (x, y) \in G \land (x, z) \in G \Rightarrow y = z$
            \item \textbf{linkseindeutig}, weil gilt: $\forall y \in Y$ und $\forall x,z \in X : (x, y) \in G \land (z, y) \in G \Rightarrow x = z$
        \end{itemize}
    \item Sei $f: X \rightarrow Y$ eine Abbildung definiert durch $f(X) :=
        \{x + 1 : x \in X\} (\subset Y)$ und $G := \{(x, f(x)) : x \in X\}
        (\subset X \times Y)$ der Graph dieser Abbildung. Diese sind durch die
        folgende Tabelle dargestellt:
        \begin{table*}[h]
            \centering
            \begin{tabular}{c|c|c}
                $X$ & $Y$ & $G$ \\
                \hline
                0  & 1 & (0,  1) \\
                1  & 2 & (1,  2) \\
                2  & 3 & (2,  3) \\
            \end{tabular}
        \end{table*} \\
        Der Graph $G$ hei{\ss}t also:
        \begin{itemize}
            \item \textbf{linkstotal}, weil gilt: $\forall x \in X, \exists y \in Y : (x, y) \in G$
            \item \textbf{rechtstotal}, weil gilt: $\forall y \in Y, \exists x \in X: (x, y) \in G$
            \item \textbf{rechtseindeutig}, weil gilt: $\forall x \in X$ und $\forall y,z \in Y : (x, y) \in G \land (x, z) \in G \Rightarrow y = z$
            \item \textbf{linkseindeutig}, weil gilt: $\forall y \in Y$ und $\forall x,z \in X : (x, y) \in G \land (z, y) \in G \Rightarrow x = z$
        \end{itemize}
\end{enumerate}

\section{Aufgabe 42}

Es sei $M := \{(x_1, x_2) \in \mathbb{R}^2 : x_1 \neq 0\}$, und $R \subset M
\times M$ eine Relation definiert durch $R := \{((x_1, x_2), (y_1, y_2)) \in M
\times M : x_2/x_1 = y_2/y_1\}$. Zeigen Sie, dass die Relation $R$ transitiv
ist.

Eine Relation $R$ hei{\ss}t transitiv, wenn
\begin{equation*}
    \forall x,y,z \in M \text{ gilt} : (x, y) \in R \land (y, z) \in R \implies (x, z) \in R
\end{equation*}
Seien $((x_1, x_2), (y_1, y_2)) \in R$ und $((y_1, y_2), (z_1, z_2)) \in R$.
Dann gilt:
\begin{align*}
    &\left(((x_1, x_2), (y_1, y_2)) \in R \Leftrightarrow \dfrac{x_2}{x_1} = \dfrac{y_2}{y_1}\right)
    \land
    \left(((y_1, y_2), (z_1, z_2)) \in R \Leftrightarrow \dfrac{y_2}{y_1} = \dfrac{z_2}{z_1}\right) \\[5pt]
    \iff&
    ((x_1, x_2), (y_1, y_2)) \in R \land ((y_1, y_2), (z_1, z_2)) \in R
    \Leftrightarrow
    \dfrac{x_2}{x_1} = \dfrac{y_2}{y_1} \land \dfrac{y_2}{y_1} = \dfrac{z_2}{z_1}
\end{align*}
Und wenn $\dfrac{x_2}{x_1} = \dfrac{y_2}{y_1} \land \dfrac{y_2}{y_1} =
\dfrac{z_2}{z_1} \Rightarrow \dfrac{x_2}{x_1} = \dfrac{z_2}{z_1}$ gilt, gilt auch
\begin{align*}
    &((x_1, x_2), (y_1, y_2)) \in R \land ((y_1, y_2), (z_1, z_2)) \in R
    \Rightarrow
    \dfrac{x_2}{x_1} = \dfrac{z_2}{z_1} \\[5pt]
    \iff&
    ((x_1, x_2), (y_1, y_2)) \in R \land ((y_1, y_2), (z_1, z_2)) \in R
    \Rightarrow
    ((x_1, x_2), (z_1, z_2)) \in R
\end{align*}
Daher ist die Relation $R$ transitiv, was zu beweisen war.

\pagebreak
\section{Aufgabe 44}

Es sei $M$ eine Menge, $R \subset M \times M$ eine "Aquivalenzrelation und $x,
y \in M$. Zeigen Sie, dass dann entweder $[x]_R = [y]_R$ oder $[x]_R \cap [y]_R
= \emptyset$ gilt.

Zu zeigen ist, dass die Aussage
\begin{align*}
    (R \subset M \times M) \land (x, y \in M) \Rightarrow ([x]_R = [y]_R) \xor ([x]_R \cap [y]_R = \emptyset)
\end{align*}
gilt.

Die Aussage $(R \subset M \times M) \land (x, y \in M)$ ist immer wahr laut der
Aufgabenstellung. Es bleibt also nur noch zu beweisen, dass der rechte Teil
der Implikation auch gilt.
\begin{equation*}
    \begin{array}{rrcrcl}
                         &([x]_R = [y]_R) &\xor&           ([x]_R \cap [y]_R    = \emptyset) &\overset{!}{=}& w \\[15pt]
        \iff             &([x]_R = [y]_R) &\lor&           ([x]_R \cap [y]_R    = \emptyset) &\land& \\[5pt]
             &\land\ \lnot([x]_R = [y]_R) &\lor&      \lnot([x]_R \cap [y]_R    = \emptyset) &\overset{!}{=}& w \\[5pt]
        \iff &\lnot([x]_R \neq [y]_R)     &\lor&           ([x]_R \cap [y]_R    = \emptyset) &\land& \\[5pt]
             &\land\ \lnot([x]_R = [y]_R) &\lor&           ([x]_R \cap [y]_R \neq \emptyset) &\overset{!}{=}& w \\[5pt]
        \iff             &([x]_R \neq [y]_R) &\Rightarrow& ([x]_R \cap [y]_R    = \emptyset) &\land& \\[5pt]
                  &\land\ ([x]_R = [y]_R)    &\Rightarrow& ([x]_R \cap [y]_R \neq \emptyset) &\overset{!}{=}& w
     \end{array}
\end{equation*}
Dies bringt zwei weitere Implikationen:
\begin{enumerate}[i)]
    \item $[x]_R \neq [y]_R \Rightarrow ([x]_R \cap [y]_R = \emptyset)$ : wenn
        "Aquivalenzklassen $[x]_R$ und $[y]_R$ ungleich sind, dann sind sie
        komplett unterschiedlich.

        Nehmen wir an, dass die Aussage $([x]_R \cap [y]_R = \emptyset)$ nicht wahr ist.
        Dann gilt
        \begin{equation*}
            \begin{array}{rc}
                        &[x]_R \cap [y]_R \neq \emptyset \Leftrightarrow (\exists_{z \in M} : z \in [x]_R \land z \in [y]_R)  \\[5pt]
                \implies& (x, z) \in R \land (y, z) \in R \Rightarrow (x, y) \in R
            \end{array}
        \end{equation*}
        Daraus folgt, dass $x$ und $y$ miteinander verbunden sind und geh"oren
        daher zur gleichen "Aquivalenzklassen. Das hei{\ss}t
        \begin{equation*}
            \begin{array}{rlcl}
                      &\phantom{\lnot}[x]_R \cap [y]_R \neq \emptyset &\implies& \phantom{\lnot}[x]_R = [y]_R \\[5pt]
                \iff  &\lnot[x]_R \cap [y]_R =    \emptyset &\implies& \lnot[x]_R \neq [y]_R \\[5pt]
                \iff  &\phantom{\lnot}[x]_R \neq [y]_R                &\implies& \phantom{\lnot}[x]_R \cap [y]_R =    \emptyset \quad \text{\underline{q.e.d.}}
            \end{array}
        \end{equation*}
    \item $[x]_R = [y]_R \Rightarrow ([x]_R \cap [y]_R \neq \emptyset)$ : wenn
        "Aquivalenzklassen $[x]_R$ und $[y]_R$ gleich sind, dann bilden sie
        eine nicht-leere Schnittmenge.

        Die Aussage ist allgemein f"ur alle Mengen richtig und braucht keinen
        formellen Beweis.
\end{enumerate}

Daraus folgt:
\begin{equation*}
    \begin{array}{rrcrcl}
        \iff             &([x]_R \neq [y]_R) &\Rightarrow& ([x]_R \cap [y]_R    = \emptyset) &\land& \\[5pt]
                  &\land\ ([x]_R = [y]_R)    &\Rightarrow& ([x]_R \cap [y]_R \neq \emptyset) &=& w
     \end{array}
\end{equation*}
was zu beweisen war $\square$

\section{Aufgabe 45}

\begin{enumerate}[(a)]
    \item $\begin{aligned}[t]
            M \times M = \{&(a, a), (a, b), (a, c), (a, d), \\
                           &(b, a), (b, b), (b, c), (b, d), \\
                           &(c, a), (c, b), (c, c), (c, d), \\
                           &(d, a), (d, b), (d, c), (d, d)\}
           \end{aligned}$
    \item
        \begin{enumerate}[i)]
            \item ist $R_1$ reflexiv? \textit{Ja.}
            \item ist $R_1$ symmetrisch? \textit{Ja.}
            \item ist $R_1$ transitiv? \textit{Ja.}
            \item Was folgt somit f"ur $R_1$? \textit{$R_1$ ist eine "Aquivalenzrelation.}
        \end{enumerate}
    \item
        \begin{enumerate}[i)]
            \item ist $R_2$ reflexiv? \textit{Nein.}
            \item ist $R_2$ symmetrisch? \textit{Ja.}
            \item ist $R_2$ total? \textit{Ja.}
            \item ist $R_2$ asymmetrisch? \textit{Nein.}
        \end{enumerate}
    \item
        \begin{enumerate}[i)]
            \item ist $R_3$ reflexiv, symmetrisch und transitiv? \textit{Ja.}
        \end{enumerate}
    \item
        \begin{enumerate}[i)]
            \item ist $R_4$ trichotom? \textit{Ja.}
            \item ist $R_4$ asymmetrisch? \textit{Ja.}
        \end{enumerate}
    \item
        \begin{enumerate}[i)]
            \item ist $R_5$ antisymmetrisch? \textit{Ja.}
        \end{enumerate}
    \item
        $\begin{aligned}[t]
            R_6 &:= \{(a, a), (b, b), (c, c), (d, d), (e, e), (f, f), (a, b), (b, c), (a, c)\} \\
            [a]_{R_6} &:= \{a, b, c\} \\
            [b]_{R_6} &:= \{b, c\} \\
            R_7 &:= \{(a, a), (b, b), (c, c), (d, d), (e, e), (f, f)\}
        \end{aligned}$
\end{enumerate}

\end{document}
