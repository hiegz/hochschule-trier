\documentclass[10pt, oneside]{article}
\usepackage[a4paper, total={5.5in, 9in}]{geometry}
\usepackage[ngerman]{babel}

\usepackage{blindtext}
\usepackage{titlesec}
\usepackage{amsmath}
\usepackage[hidelinks]{hyperref}
\usepackage{parskip}
\usepackage{graphicx}
\usepackage{longtable}
\usepackage[shortlabels]{enumitem}
\usepackage{multirow}
\usepackage{nccmath}
\usepackage{rotating}
\usepackage{makecell}
\usepackage{multicol}
\usepackage{capt-of}
\usepackage{csquotes}
\usepackage{amsfonts}
\usepackage{caption}

\captionsetup[table]{position=bottom}

\titleformat{\section}
    {\normalfont\Large\bfseries}{}{0pt}{}

\let\oldsection\section
\renewcommand{\section}{
  \renewcommand{\theequation}{\thesection.\arabic{equation}}
  \oldsection}
\let\oldsubsection\subsection
\renewcommand{\subsection}{
  \renewcommand{\theequation}{\thesubsection.\arabic{equation}}
  \oldsubsection}

\makeatletter
\renewcommand{\maketitle}{
    \bgroup
    \centering
    \par\LARGE\@title  \\[20pt]
    \par\large\@author \\[10pt]
    \par\large\@date
    \par
    \egroup
}
\makeatother


\title{Mathematische Grundlagen\\[10pt]\Large{WiSe 2024/25}\\[15pt]\Large{L{\"o}sungen zu den Aufgaben 33, 35, 36 und 38}}
\author{Volodymyr But}
\date{Hochschule Trier}

% - - - - - - - - - - - - - - - - - - - - - - - - - - - - - - - - - - - - - - %

\begin{document}
\sloppy

\maketitle
\vspace{25px}

\section{Aufgabe 33}

Zeigen Sie mittels Wahrheitstabelle das Folgende:
\begin{enumerate}[a)]
    \item $F_1(A, B) := B \Rightarrow A$ und $F_2(A, B) := \lnot A \Rightarrow
        \lnot B$ sind gleichwertig ($F_1 \Leftrightarrow F_2$)
    \begin{table*}[h]
        \centering
        \begin{minipage}{.29\linewidth}
            \centering
            \begin{tabular}{|c|c|c|}
                \hline
                $A$ & $B$ & $F_1(A, B)$ \\
                \hline
                $w$ & $w$ & $w$ \\
                \hline
                $w$ & $f$ & $w$ \\
                \hline
                $f$ & $w$ & $f$ \\
                \hline
                $f$ & $f$ & $w$ \\
                \hline
            \end{tabular}
        \end{minipage}%
        \begin{minipage}{.29\linewidth}
            \centering
            \begin{tabular}{|c|c|c|}
                \hline
                $A$ & $B$ & $F_2(A, B)$ \\
                \hline
                $w$ & $w$ & $w$ \\
                \hline
                $w$ & $f$ & $w$ \\
                \hline
                $f$ & $w$ & $f$ \\
                \hline
                $f$ & $f$ & $w$ \\
                \hline
            \end{tabular}
        \end{minipage}  \\[15pt]
        \begin{minipage}{1\linewidth}
            \centering
            \begin{tabular}{|c|c|c|}
                \hline
                $F_1(A, B)$ & $F_2(A, B)$ & $F_1(A, B) \Leftrightarrow F_2(A, B)$ \\
                \hline
                $w$ & $w$ & $w$ \\
                \hline
                $w$ & $w$ & $w$ \\
                \hline
                $f$ & $f$ & $w$ \\
                \hline
                $w$ & $w$ & $w$ \\
                \hline
            \end{tabular}
        \end{minipage}
    \end{table*}
    \item $F_3(A, B) := \lnot (A \land B)$ und $F_4(A, B) := \lnot A \lor \lnot
        B$ sind gleichwertig ($F_3 \Leftrightarrow F_4$)
    \begin{table*}[h]
        \centering
        \begin{minipage}{.29\linewidth}
            \centering
            \begin{tabular}{|c|c|c|}
                \hline
                $A$ & $B$ & $F_3(A, B)$ \\
                \hline
                $w$ & $w$ & $f$ \\
                \hline
                $w$ & $f$ & $w$ \\
                \hline
                $f$ & $w$ & $w$ \\
                \hline
                $f$ & $f$ & $w$ \\
                \hline
            \end{tabular}
        \end{minipage}%
        \begin{minipage}{.29\linewidth}
            \centering
            \begin{tabular}{|c|c|c|}
                \hline
                $A$ & $B$ & $F_4(A, B)$ \\
                \hline
                $w$ & $w$ & $f$ \\
                \hline
                $w$ & $f$ & $w$ \\
                \hline
                $f$ & $w$ & $w$ \\
                \hline
                $f$ & $f$ & $w$ \\
                \hline
            \end{tabular}
        \end{minipage}  \\[15pt]
        \begin{minipage}{1\linewidth}
            \centering
            \begin{tabular}{|c|c|c|}
                \hline
                $F_3(A, B)$ & $F_4(A, B)$ & $F_3(A, B) \Leftrightarrow F_4(A, B)$ \\
                \hline
                $f$ & $f$ & $w$ \\
                \hline
                $w$ & $w$ & $w$ \\
                \hline
                $w$ & $w$ & $w$ \\
                \hline
                $w$ & $w$ & $w$ \\
                \hline
            \end{tabular}
        \end{minipage}
    \end{table*}
\end{enumerate}
Somit ist die Gleichwertigkeit der Aussagen bewiesen.

\pagebreak
\section{Aufgabe 35}

Zeigen Sie mittels Wahrheitstabelle, dass $F_1(A, B, C) := A \Rightarrow (B
\lor C)$ und $F_2(A, B, C) := \lnot(A \land \lnot(B \lor C))$ gleichwertig sind
($F_1 \Leftrightarrow F_2$)

\begin{table*}[h]
    \centering
    \begin{minipage}{.4\linewidth}
        \centering
        \begin{tabular}{|c|c|c|c|}
            \hline
            $A$ & $B$ & $C$ & $F_1(A, B, C)$ \\
            \hline
            $w$ & $w$ & $w$ & $w$ \\
            \hline
            $w$ & $w$ & $f$ & $w$ \\
            \hline
            $w$ & $f$ & $w$ & $w$ \\
            \hline
            $f$ & $w$ & $w$ & $w$ \\
            \hline
            $w$ & $f$ & $f$ & $f$ \\
            \hline
            $f$ & $f$ & $w$ & $w$ \\
            \hline
            $f$ & $w$ & $f$ & $w$ \\
            \hline
            $f$ & $f$ & $f$ & $w$ \\
            \hline
        \end{tabular}
    \end{minipage}%
    \begin{minipage}{.4\linewidth}
        \centering
        \begin{tabular}{|c|c|c|c|}
            \hline
            $A$ & $B$ & $C$ & $F_2(A, B, C)$ \\
            \hline
            $w$ & $w$ & $w$ & $w$ \\
            \hline
            $w$ & $w$ & $f$ & $w$ \\
            \hline
            $w$ & $f$ & $w$ & $w$ \\
            \hline
            $f$ & $w$ & $w$ & $w$ \\
            \hline
            $w$ & $f$ & $f$ & $f$ \\
            \hline
            $f$ & $f$ & $w$ & $w$ \\
            \hline
            $f$ & $w$ & $f$ & $w$ \\
            \hline
            $f$ & $f$ & $f$ & $w$ \\
            \hline
        \end{tabular}
    \end{minipage}  \\[15pt]
    \begin{minipage}{1\linewidth}
        \centering
        \begin{tabular}{|c|c|c|}
            \hline
            $F_1(A, B, C)$ & $F_2(A, B, C)$ & $F_1(A, B, C) \Leftrightarrow F_2(A, B, C)$ \\
            \hline
            $w$ & $w$ & $w$ \\
            \hline
            $w$ & $w$ & $w$ \\
            \hline
            $w$ & $w$ & $w$ \\
            \hline
            $w$ & $w$ & $w$ \\
            \hline
            $f$ & $f$ & $w$ \\
            \hline
            $w$ & $w$ & $w$ \\
            \hline
            $w$ & $w$ & $w$ \\
            \hline
            $w$ & $w$ & $w$ \\
            \hline
        \end{tabular}
    \end{minipage}
\end{table*}

\section{Aufgabe 36}

Nehmen Sie an, dass Sie folgende Aussagen durch einen Widerspruchsbeweis
belegen m"ochten.
\begin{enumerate}[a)]
    \item \textit{Es gibt keine gr"o{\ss}te Primzahl}.

        \textbf{Widerspruchsbeweis:}

        Hier ist
        \begin{equation*}
            A = (\text{es gibt keine gr"o{\ss}te Primzahl}) 
            \quad\text{und}\quad 
            \lnot A = (\text{es gibt eine gr"o{\ss}te Primzahl})
        \end{equation*}
        Wir nehmen an, dass $\lnot A = wahr$ gilt. Sei $p$ diese gr"o{\ss}te
        Primzahl und seien $1 < p_1 < p_2 < ... < p_n = p$ alle Primzahlen.
        Dann ist $q = p_1 \cdot p_2 \cdot ... \cdot p_n + 1$ eine Primzahl.
        Wenn also $\lnot A = wahr$ gilt, folgt
        \begin{equation*}
            C = (q \leq p) = wahr.
        \end{equation*}
        Da aber sicher gilt $C = falsch$, haben wir den Widerspruch erzeugt,
        und es muss somit $\lnot A = falsch$ gelten.
    \item \textit{F"ur jedes $a \in [0, \infty)$ gibt es mindestens ein $x_0 \in \mathbb{R}$ sodass $x_0^3 - ax_0 = 0$.}

        \textbf{Widerspruchsbeweis:}

        Hier ist
        \begin{equation*}
            A = (\forall_{a \in [0, \infty)}, \exists_{x_0 \in \mathbb{R}} : x_0^3 - ax_0 = 0)
            \quad\text{und}\quad
            \lnot A = (\exists_{a_0 \in [0, \infty)}, \forall_{x \in \mathbb{R}} : x^3 - a_0x \neq 0)
        \end{equation*}

        \pagebreak
        Wir nehmen an, dass $\lnot A = wahr$ gilt. Sei $a_0 := 0$ und $x_0 := 1$, dann gilt
        \begin{equation*}
            C = (0^3 - 0 \cdot 0 \neq 0) = wahr
        \end{equation*}
        Da aber sicher gilt $C = falsch$, haben wir den Widerspruch erzeugt,
        und es muss somit $\lnot A = falsch$ gelten.
    \item \textit{F"ur jedes Paar $(x_1, x_2)$, $(y_1, y_2) \in \mathbb{R}^2$ gilt:}
        \begin{equation*}
            \sqrt{(x_1 + y_1)^2 + (x_2 + y_2)^2} \leq \sqrt{x_1^2 + x_2^2} + \sqrt{y_1^2 + y_2^2}
        \end{equation*}
        \textbf{Widerspruchsbeweis:}

        Hier ist
        \begin{equation*}
            A = \left(\forall_{(x_1, x_2), (y_1, y_2) \in \mathbb{R}^2} : \sqrt{(x_1 + y_1)^2 + (x_2 + y_2)^2} \leq \sqrt{x_1^2 + x_2^2} + \sqrt{y_1^2 + y_2^2}\right)
        \end{equation*}
        und
        \begin{equation*}
            \lnot A = \left(\exists_{(x_1, x_2), (y_1, y_2) \in \mathbb{R}^2} : \sqrt{(x_1 + y_1)^2 + (x_2 + y_2)^2} > \sqrt{x_1^2 + x_2^2} + \sqrt{y_1^2 + y_2^2}\right)
        \end{equation*}
        Wir nehmen an, dass $\lnot A = wahr$ gilt. Dann folgt aus der obigen Aussage:
        \begin{equation*}
            \begin{array}{rcl}
                C &=& (\sqrt{(x_1 + y_1)^2 + (x_2 + y_2)^2} > \sqrt{x_1^2 + x_2^2} + \sqrt{y_1^2 + y_2^2}) = wahr \\
                  &\vdots& \\
                C &=& ((x_1y_2 + x_2y_1)^2 < 0) = wahr
            \end{array}
        \end{equation*}
        Da aber sicher gilt $C = falsch$, haben wir den Widerspruch erzeugt,
        und es muss somit $\lnot A = falsch$ gelten.
\end{enumerate}

\section{Aufgabe 38}

In den unteren Folgerungen sind jeweils ein oder mehrere Fehler unterlaufen.
Korrigieren Sie jeweils die rechte Seite so, dass eine g"ultige Folgerung
entsteht.

\begin{enumerate}[(a)]
    \item Mit $f : X \rightarrow Y$ eine Abbildung:

        \quad $f$ ist surjektiv $\Rightarrow$ f"ur jedes $y \in Y$ gibt es ein $x \in X$ mit $f(x) = y$
    \item Mit $f : X \rightarrow Y$ und $g: Y \rightarrow Z$ zwei Abbildungen:

        \quad $f$ und $g$ sind injektiv $\Rightarrow$ $g \circ f$ ist injektiv.
    \item Mit $A, B$ zwei Aussagenvariablen:

        \quad $(A \Leftrightarrow B) \Rightarrow ((A \Rightarrow B) \land (\lnot B \Rightarrow \lnot A)$
    \item Mit $M$ eine Menge:

        \quad $N \subset M \Rightarrow N \in \mathcal{P}(M)$
\end{enumerate}

\end{document}
