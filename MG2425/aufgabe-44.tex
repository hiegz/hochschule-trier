\section{Aufgabe 44}

Es sei $M$ eine Menge, $R \subset M \times M$ eine "Aquivalenzrelation und $x,
y \in M$. Zeigen Sie, dass dann entweder $[x]_R = [y]_R$ oder $[x]_R \cap [y]_R
= \emptyset$ gilt.

Zu zeigen ist, dass die Aussage
\begin{align*}
    (R \subset M \times M) \land (x, y \in M) \Rightarrow ([x]_R = [y]_R) \xor ([x]_R \cap [y]_R = \emptyset)
\end{align*}
gilt.

Die Aussage $(R \subset M \times M) \land (x, y \in M)$ ist immer wahr laut der
Aufgabenstellung. Es bleibt also nur noch zu beweisen, dass der rechte Teil
der Implikation auch gilt.
\begin{equation*}
    \begin{array}{rrcrcl}
                         &([x]_R = [y]_R) &\xor&           ([x]_R \cap [y]_R    = \emptyset) &\overset{!}{=}& w \\[15pt]
        \iff             &([x]_R = [y]_R) &\lor&           ([x]_R \cap [y]_R    = \emptyset) &\land& \\[5pt]
             &\land\ \lnot([x]_R = [y]_R) &\lor&      \lnot([x]_R \cap [y]_R    = \emptyset) &\overset{!}{=}& w \\[5pt]
        \iff &\lnot([x]_R \neq [y]_R)     &\lor&           ([x]_R \cap [y]_R    = \emptyset) &\land& \\[5pt]
             &\land\ \lnot([x]_R = [y]_R) &\lor&           ([x]_R \cap [y]_R \neq \emptyset) &\overset{!}{=}& w \\[5pt]
        \iff             &([x]_R \neq [y]_R) &\Rightarrow& ([x]_R \cap [y]_R    = \emptyset) &\land& \\[5pt]
                  &\land\ ([x]_R = [y]_R)    &\Rightarrow& ([x]_R \cap [y]_R \neq \emptyset) &\overset{!}{=}& w
     \end{array}
\end{equation*}
Dies bringt zwei weitere Implikationen:
\begin{enumerate}[i)]
    \item $[x]_R \neq [y]_R \Rightarrow ([x]_R \cap [y]_R = \emptyset)$ : wenn
        "Aquivalenzklassen $[x]_R$ und $[y]_R$ ungleich sind, dann sind sie
        komplett unterschiedlich.

        Nehmen wir an, dass die Aussage $([x]_R \cap [y]_R = \emptyset)$ nicht wahr ist.
        Dann gilt
        \begin{equation*}
            \begin{array}{rc}
                        &[x]_R \cap [y]_R \neq \emptyset \Leftrightarrow (\exists_{z \in M} : z \in [x]_R \land z \in [y]_R)  \\[5pt]
                \implies& (x, z) \in R \land (y, z) \in R \Rightarrow (x, y) \in R
            \end{array}
        \end{equation*}
        Daraus folgt, dass $x$ und $y$ miteinander verbunden sind und geh"oren
        daher zur gleichen "Aquivalenzklassen. Das hei{\ss}t
        \begin{equation*}
            \begin{array}{rlcl}
                      &\phantom{\lnot}[x]_R \cap [y]_R \neq \emptyset &\implies& \phantom{\lnot}[x]_R = [y]_R \\[5pt]
                \iff  &\lnot[x]_R \cap [y]_R =    \emptyset &\implies& \lnot[x]_R \neq [y]_R \\[5pt]
                \iff  &\phantom{\lnot}[x]_R \neq [y]_R                &\implies& \phantom{\lnot}[x]_R \cap [y]_R =    \emptyset \quad \text{\underline{q.e.d.}}
            \end{array}
        \end{equation*}
    \item $[x]_R = [y]_R \Rightarrow ([x]_R \cap [y]_R \neq \emptyset)$ : wenn
        "Aquivalenzklassen $[x]_R$ und $[y]_R$ gleich sind, dann bilden sie
        eine nicht-leere Schnittmenge.

        Die Aussage ist allgemein f"ur alle Mengen richtig und braucht keinen
        formellen Beweis.
\end{enumerate}

Daraus folgt:
\begin{equation*}
    \begin{array}{rrcrcl}
        \iff             &([x]_R \neq [y]_R) &\Rightarrow& ([x]_R \cap [y]_R    = \emptyset) &\land& \\[5pt]
                  &\land\ ([x]_R = [y]_R)    &\Rightarrow& ([x]_R \cap [y]_R \neq \emptyset) &=& w
     \end{array}
\end{equation*}
was zu beweisen war $\square$
