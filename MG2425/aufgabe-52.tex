\section{Aufgabe 52}

Welche der folgenden Beispiele ist eine Gruppe:

\begin{enumerate}[(a)]
    \item $(\mathbb{Z}, \cdot)$, wobei $\cdot$ die "ubliche Multiplikation bezeichne.
        \begin{todolist}
            \item[\done] $\forall x, y, z \in \mathbb{Z} : (x \cdot y) \cdot z = x \cdot (y \cdot z)$
            \item[\done] $\forall x \in \mathbb{Z}, \exists e \in \mathbb{Z} : e \cdot x = x \ (e = 1)$
            \item[\wontfix] $\forall x \in \mathbb{Z}, \exists x^{-1} \in \mathbb{Z} : x^{-1} \cdot x = e \ (x^{-1} = \dfrac{1}{x}, \dfrac{1}{x} \notin \mathbb{Z}, e = 1)$
        \end{todolist}

    \item $(M, +)$ mit $M$ die Menge der geraden ganzen Zahlen und $+$ der "ublichen Addition.
        \begin{todolist}
            \item[\done] $\forall x, y, z \in M : (x + y) + z = x + (y + z)$
            \item[\done] $\forall x \in M, \exists e \in M : e + x = x \ (e = 0)$
            \item[\done] $\forall x \in M, \exists x^{-1} \in M : x^{-1} + x = e \ (x^{-1} = -x, e = 0)$
        \end{todolist}

    \item $(\mathbb{R}^n, +)$, wobei $+$ die "ubliche Addition von Tupeln.
        \begin{todolist}
            \item[\done] $\forall x, y, z \in \mathbb{R}^n : (x + y) + z = x + (y + z)$
            \item[\done] $\forall x \in \mathbb{R}^n, \exists e \in \mathbb{R}^n : e + x = x \ (e = (0_1,...,0_n))$
            \item[\done] $\forall x \in \mathbb{R}^n, \exists x^{-1} \in \mathbb{R}^n : x^{-1} + x = e \ (x^{-1} = -x, e = (0_1,...,0_n))$
        \end{todolist}
\end{enumerate}
