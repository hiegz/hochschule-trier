\section{Aufgabe 45}

\begin{enumerate}[(a)]
    \item $\begin{aligned}[t]
            M \times M = \{&(a, a), (a, b), (a, c), (a, d), \\
                           &(b, a), (b, b), (b, c), (b, d), \\
                           &(c, a), (c, b), (c, c), (c, d), \\
                           &(d, a), (d, b), (d, c), (d, d)\}
           \end{aligned}$
    \item
        \begin{enumerate}[i)]
            \item ist $R_1$ reflexiv? \textit{Ja.}
            \item ist $R_1$ symmetrisch? \textit{Ja.}
            \item ist $R_1$ transitiv? \textit{Ja.}
            \item Was folgt somit f"ur $R_1$? \textit{$R_1$ ist eine "Aquivalenzrelation.}
        \end{enumerate}
    \item
        \begin{enumerate}[i)]
            \item ist $R_2$ reflexiv? \textit{Nein.}
            \item ist $R_2$ symmetrisch? \textit{Ja.}
            \item ist $R_2$ total? \textit{Ja.}
            \item ist $R_2$ asymmetrisch? \textit{Nein.}
        \end{enumerate}
    \item
        \begin{enumerate}[i)]
            \item ist $R_3$ reflexiv, symmetrisch und transitiv? \textit{Ja.}
        \end{enumerate}
    \item
        \begin{enumerate}[i)]
            \item ist $R_4$ trichotom? \textit{Ja.}
            \item ist $R_4$ asymmetrisch? \textit{Ja.}
        \end{enumerate}
    \item
        \begin{enumerate}[i)]
            \item ist $R_5$ antisymmetrisch? \textit{Ja.}
        \end{enumerate}
    \item
        $\begin{aligned}[t]
            R_6 &:= \{(a, a), (b, b), (c, c), (d, d), (e, e), (f, f), (a, b), (b, c), (a, c)\} \\
            [a]_{R_6} &:= \{a, b, c\} \\
            [b]_{R_6} &:= \{b, c\} \\
            R_7 &:= \{(a, a), (b, b), (c, c), (d, d), (e, e), (f, f)\}
        \end{aligned}$
\end{enumerate}
