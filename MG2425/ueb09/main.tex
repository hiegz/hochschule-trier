\documentclass[10pt, oneside]{article}
\usepackage[a4paper, total={5.5in, 9in}]{geometry}
\usepackage[ngerman]{babel}

\usepackage{blindtext}
\usepackage{titlesec}
\usepackage{amsmath}
\usepackage[hidelinks]{hyperref}
\usepackage{parskip}
\usepackage{graphicx}
\usepackage{longtable}
\usepackage[shortlabels]{enumitem}
\usepackage{multirow}
\usepackage{nccmath}
\usepackage{rotating}
\usepackage{makecell}
\usepackage{multicol}
\usepackage{capt-of}
\usepackage{csquotes}
\usepackage{amsfonts}
\usepackage{caption}

\captionsetup[table]{position=bottom}

\titleformat{\section}
    {\normalfont\Large\bfseries}{}{0pt}{}

\let\oldsection\section
\renewcommand{\section}{
  \renewcommand{\theequation}{\thesection.\arabic{equation}}
  \oldsection}
\let\oldsubsection\subsection
\renewcommand{\subsection}{
  \renewcommand{\theequation}{\thesubsection.\arabic{equation}}
  \oldsubsection}

\makeatletter
\renewcommand{\maketitle}{
    \bgroup
    \centering
    \par\LARGE\@title  \\[20pt]
    \par\large\@author \\[10pt]
    \par\large\@date
    \par
    \egroup
}
\makeatother


\title{Mathematische Grundlagen\\[10pt]\Large{WiSe 2024/25}\\[15pt]\Large{L{\"o}sung zu der Aufgabe 72}}
\author{Volodymyr But\\[10pt]Hochschule Trier}
\date{}

% - - - - - - - - - - - - - - - - - - - - - - - - - - - - - - - - - - - - - - %

\begin{document}

\maketitle
\vspace{25px}

\section{Aufgabe 72}

\begin{enumerate}[(a)]
    \item In der Mensa stehen 6 Men"us zur Auswahl und Sie gehen mit 6 weiteren
        Kommilitonen zum Essen, von denen jeder genau ein Men"u nimmt. Wie
        viele m"ogliche Men"uauswahlen k"onnen dabei getroffenen werden? Es
        soll dabei alleine die Auswahl, und nicht die Zuordnung zu den
        einzelnen Personen unterschieden werden.
        \begin{align*}
            \text{Ur}_6^6(mZ, oR) &= \binom{6 + 6 - 1}{6} = \binom{11}{6} = \dfrac{11!}{6!\cdot5!} = 462
        \end{align*}
    \item Zeigen Sie, dass f"ur $n, \nu \in \mathbb{N}$, $n \geq \nu$ gilt
        \begin{align*}
            \binom{n + 1}{\nu} &= \binom{n}{\nu - 1} + \binom{n}{\nu} \\[5pt]
                               &= \dfrac{n!}{(\nu - 1)!(n - \nu + 1)!} + \dfrac{n!}{\nu!(n - \nu)!} \\[5pt]
                               &= \dfrac{n!\nu + n!(n - \nu + 1))}{\nu!(n - \nu + 1)!} \\[5pt]
                               &= \dfrac{n!(n + 1)}{\nu!(n - \nu + 1)!} \\[5pt]
                               &= \binom{n + 1}{\nu}
        \end{align*}
\end{enumerate}

\end{document}
