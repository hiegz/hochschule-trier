\section{Aufgabe 35}
\setcounter{section}{35}

Ein neues Online-Lernprogramm soll getestet werden, der laut Anbieter dazu
f"uhren, dass 70\% der Sch"uler eine Verbesserung in ihren Testergebnissen
erzielen. Eine Stichprobe von $n = 40$ Sch"ulern wird zuf"allig ausgew"ahlt und
mit dem Programm geschult. Gegeben sei, dass in der Stichprobe 32 von 40
Sch"ulern eine Verbesserung ihrer Testergebnisse erzielen.

Bestimmen Sie bei einem Signifikanzniveau von $\alpha = 0.05$ f"ur jede
Teilfrage, ob die jeweilige Nullhypothese $H_0$ abgelehnt wird.

\begin{enumerate}[(a)]
    \item Wir nehmen an, dass das Programm tats"achlich eine Erfolgsquote von
        70\% besitzt.
        \begin{enumerate}[i)]
            \item Nullhypothese $H_0$ : $p = 0.7$
            \item Alternativhypothese $H_1$ : $p \neq 0.7$
        \end{enumerate}
        Wir w"ahlen dann das gr"o{\ss}tm"ogliche $k_0, k_1 \in \{0,...,40\}$,
        sodass
        \begin{equation*}
            P(X \leq k_0) + P(X \geq n - k_1) \leq \alpha
        \end{equation*}
        Es ergibt sich damit
        \begin{equation*}
            A := \{0,...,21\} \cup \{34,...,40\} \quad \text{Ablehnungsbereich}
        \end{equation*}
        Daraus folgt mit $T(X) = 32$
        \begin{equation*}
            T(X) \notin A \implies H_0 \text{ wird nicht abgelehnt.}
        \end{equation*}
    \item Das Ziel ist auszuschlie{\ss}en, dass die Erfolgsquote unter 60\% liegt.
        \begin{enumerate}[i)]
            \item Nullhypothese $H_0$ : $p \leq 0.6$
            \item Alternativhypothese $H_1$ : $p > 0.6$
        \end{enumerate}
        Wir w"ahlen dann das gr"o{\ss}tm"ogliche $k_0 \in \{0,...,40\}$,
        sodass
        \begin{equation*}
            P(X \geq n - k_0) \leq \alpha
        \end{equation*}
        Es ergibt sich damit
        \begin{equation*}
            A := \{30,...,40\} \quad \text{Ablehnungsbereich}
        \end{equation*}
        Daraus folgt mit $T(X) = 32$
        \begin{equation*}
            T(X) \in A \implies H_0 \text{ wird abgelehnt.}
        \end{equation*}
    \item Das Ziel ist zu testen, ob die Erfolgsquote m"oglicherweise unter 90\%
        liegt.
        \begin{enumerate}[i)]
            \item Nullhypothese $H_0$ : $p \geq 0.9$
            \item Alternativhypothese $H_1$ : $p < 0.9$
        \end{enumerate}
        Wir w"ahlen dann das gr"o{\ss}tm"ogliche $k_0 \in \{0,...,40\}$,
        sodass
        \begin{equation*}
            P(X \leq k_0) \leq \alpha
        \end{equation*}
        Es ergibt sich damit
        \begin{equation*}
            A := \{0,...,32\} \quad \text{Ablehnungsbereich}
        \end{equation*}
        Daraus folgt mit $T(X) = 32$
        \begin{equation*}
            T(X) \in A \implies H_0 \text{ wird abgelehnt.}
        \end{equation*}
\end{enumerate}
