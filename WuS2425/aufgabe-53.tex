\section{Aufgabe 53}
\setcounter{section}{53}

Gegeben ist eine Urne, die vier Kugeln enth"alt. Die Kugeln sind mit den Zahlen
1, 2, 3, 4 nummeriert. Es werden zwei Kugeln gleichzeitig und zuf"allig aus der
Urne gezogen, und anschlie{\ss}end werden die Nummern der beiden gezogenen
Kugeln addiert.
\begin{enumerate}
    \item Bestimmen Sie die m"oglichen Werte, die die Summe der gezogenen
        Kugelnummern annehmen kann, und berechnen Sie die Wahrscheinlichkeit
        f"ur jede dieser Summen.

        Sei $X : \Omega \rightarrow \mathbb{R}$ die Zufallsvariable, die die
        Summe der beiden gezogenen Kugelnummern annehmt, und $(\Omega, P)$ ein
        W-Raum, dann gilt:
        \begin{equation*}
            |\Omega| = \binom{4}{2} = 6 \quad\text{und}\quad P = \dfrac{1}{6}
        \end{equation*}
        \begin{table*}[h]
            \centering
            \begin{tabular}{c|c}
                $x_i$ & $P(X = x_i)$ \\
                \hline
                3 & 0.16 \\
                4 & 0.16 \\
                5 & 0.33 \\
                6 & 0.16 \\
                7 & 0.16 \\
            \end{tabular}
        \end{table*}
    \item Berechnen Sie den Erwartungswert der Summe der gezogenen Kugelnummern.
        \begin{equation*}
            E(X) = \dfrac{1}{6} \cdot (3 + 4 + 6 + 7) + \dfrac{1}{3} \cdot 5 = 5
        \end{equation*}
\end{enumerate}
