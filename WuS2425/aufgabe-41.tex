\section{Aufgabe 41}
\setcounter{section}{41}

In einer Studie wird untersucht, ob die Pr"aferenz f"ur eine bestimmte
Freizeitaktivit"at (un)abh"angig vom Wohnort ist. Dazu wurden 300 Personen
befragt. Die Antworten wurden in einer Kontingenztabellen zusammengefasst:

\begin{table}[h]
    \centering
    \begin{tabular}{|c|c|c|c|c|}
        \hline
        Wohnort  & Aktivit"at 1 & Aktivit"at 2 & Aktivit"at 3 & Summe \\
        Stadt    & 50           & 30           & 20           & 100 \\
        Vorstadt & 40           & 50           & 30           & 120 \\
        Land     & 20           & 30           & 30           & 80 \\
        Summe    & 110          & 110          & 80           & 300 \\ \hline
    \end{tabular}
    \caption{Antworten der Umfrageteilnehmer}
\end{table}

\begin{enumerate}
    \item Formulieren Sie die Nullhypothese $H_0$ und die Alternativhypothese
        $H_1$ f"ur einen $\chi^2$-Unabh"angigkeitstest.
        \begin{itemize}[-]
            \item Nullhypothese $H_0$: $X \perp Y$
            \item Alternativhypothese $H_1$: $X \not\perp Y$
        \end{itemize}
    \item Berechnen Sie den $\chi^2$-Wert und bestimmen Sie die Freiheitsgrade des Tests.

        Aus den Antworten der Umfrageteilnehmer lassen sich die erwarteten
        Werte ableiten. Die Ergebnisse sind in der folgenden Tabelle
        zusammengefasst:

        \begin{table}[h]
            \centering
            \begin{tabular}{|c|c|c|c|}
                \hline
                Wohnort  & Aktivit"at 1 & Aktivit"at 2 & Aktivit"at 3 \\
                Stadt    & 36.7         & 36.7         & 26.7         \\
                Vorstadt & 44           & 44           & 32           \\
                Land     & 29.3         & 29.3         & 21.3         \\ \hline
            \end{tabular}
            \caption{Erwartungswerte $E_{j,k} := n\hat{p}_{j1}\hat{p}_{2k}$}
        \end{table}

        Der entsprechende $\chi^2$-Wert ist
        \begin{equation*}
            \begin{array}[t]{rccccccl}
                15.55 &=&\dfrac{(50 - 36.7)^2}{36.7} &+& \dfrac{(30 - 36.7)^2}{36.7} &+& \dfrac{(20 - 26.7)^2}{26.7} &+ \\[10pt]
                      &+& \dfrac{(40 - 44)^2}{44}     &+& \dfrac{(50 - 44)^2}{44}     &+& \dfrac{(30 - 32)^2}{32}     &+ \\[10pt]
                      &+& \dfrac{(20 - 29.3)^2}{29.3} &+& \dfrac{(30 - 29.3)^2}{29.3} &+& \dfrac{(30 - 21.3)^2}{21.3}
               \end{array} \\[10pt]
        \end{equation*}
        Die Freiheitsgrade des Tests ist
        \begin{equation*}
            4 = (3 - 1)(3 - 1)
        \end{equation*}
    \item F"uhren Sie den Test auf einem Signifikanzniveau von $\alpha = 0.05$
        durch, und ziehen Sie eine Schlussfolgerung.

        Ein Signifikanzniveau $\alpha = 0.05$ und Freiheitsgrade von $4$
        entsprechen einem Grenzwert von $9.488$. Da $9.488 < 15.55$ gilt, wird
        die Hypothese, dass der Ort und Aktivit"aten unabh"angig sind,
        abgelehnt.
\end{enumerate}
