\documentclass[10pt, oneside]{article}
\usepackage[a4paper, total={5.5in, 9in}]{geometry}
\usepackage[ngerman]{babel}

\usepackage{blindtext}
\usepackage{titling}
\usepackage{titlesec}
\usepackage{amsmath}
\usepackage[hidelinks]{hyperref}
\usepackage{parskip}
\usepackage{graphicx}

\setlength{\droptitle}{-3cm}

\titleformat{\section}
    {\normalfont\Large\bfseries}{}{0pt}{}

\let\oldsection\section
\renewcommand{\section}{
  \renewcommand{\theequation}{\thesection.\arabic{equation}}
  \oldsection}
\let\oldsubsection\subsection
\renewcommand{\subsection}{
  \renewcommand{\theequation}{\thesubsection.\arabic{equation}}
  \oldsubsection}


\title{Wahrscheinlichkeitstheorie und Statistik\\[5pt]\Large WiSe 2024/25\\[15pt]\Large L{\"o}sungen zu den Aufgaben 19, 20, 23 und 26}
\author{Volodymyr But}
\date{Hochschule Trier}

% - - - - - - - - - - - - - - - - - - - - - - - - - - - - - - - - - - - - - - %

\begin{document}
\sloppy

\maketitle
\vspace{25pt}

\section{Aufgabe 19}

Es sei $(\Omega, P)$ ein W-Raum und $A, B \subset \Omega$ sowie $A_1,...,A_n
\subset \Omega$ Ereignisse.

\begin{enumerate}[(a)]
    \item Geben Sie folgendes Ereignis in der Mengenschreibweise an: Es treten
        $A$ und mindestens eines der $A_1,...,A_n$ ein.
        \begin{equation*}
            A \times \left(\bigcup_{i = 1}^nA_i\right)
        \end{equation*}
    \item Geben Sie folgendes Ereignis in der Mengenschreibweise: Es tritt genau
        eines der Ereignisse A, B ein
        \begin{equation*}
            (A \setminus B) \cup (B \setminus A)
        \end{equation*}
    \item Beschreiben Sie das Folgende mathematisch mittels $P$: Es ist
        wahrscheinlicher, dass keines der $A_1,...,A_n$ eintritt, als das B
        eintritt.
        \begin{equation*}
            P\left(\Omega \setminus \left(\bigcup_{i = 1}^nA_i\right)\right) > P(B)
        \end{equation*}
\end{enumerate}

\section{Aufgabe 20}

In einer Kiste mit Schrauben seien zu je gleichen Teilen Schrauben der St"arke
4 und 5 (mm) und zu je gleichen Teilen Schrauben der L"angen 30, 40 und 50
(mm). Es werde nun zuf"allig eine Schraube aus der Kiste genommen und sowohl
zwischen St"arke als auch der L"ange der Schraube unterschieden.

\begin{enumerate}[(a)]
    \item Geben Sie einen geeigneten W-Raum an, der dieses Experiment modelliert.
        \begin{equation*}
            \Omega = \{4, 5\} \times \{30, 40, 50\} = \{(4, 30), (4, 40), (4, 50), (5, 30), (5, 40), (5, 50)\}
        \end{equation*}
    \item Sie ben"otigen eine Schraube der St"arke 4 (mm) und von maximaler L"ange 40 (mm).
        Mit welcher Wahrscheinlichkeit ziehen Sie eine passende Schraube?
        \begin{equation*}
            A = \{(4, 30), (4, 40)\} \quad\quad
            P(A) = \dfrac{|A|}{|\Omega|} = \dfrac{2}{6} = \dfrac{1}{3} = 0.33
        \end{equation*}
\end{enumerate}

\section{Aufgabe 23}

In einer Urne liegen 90 rote Kugeln und 10 wei{\ss}e Kugeln. In einem
Experiment werde 6 mal eine Kugel mit Zur"ucklegen gezogen.

\begin{enumerate}[(a)]
    \item \label{itm:23-a} Definieren Sie einen Wahrscheinlichkeitsraum, der dieses Experiment modelliert.
        \begin{equation*}
            \begin{array}{rcl}
                &|\text{Ur}_k^n(mR, mZ)|& = n^k \\[5pt]
                |\Omega| =\ &|\text{Ur}_6^{100}(mR, mZ)|& = 100^6 = 10^{12}
            \end{array}
        \end{equation*}
    % \item Definieren Sie auf ihrem W-Raum aus Teil \ref{itm:23-a} eine
    %     Zufallsvariable $X$, die die Anzahl der wei{\ss}en Kugeln in diesem
    %     Experiment z"ahlt, und berechnen Sie damit $P(X = 2)$, $P(X > 4)$ und
    %     $P(X \leq 2)$.
\end{enumerate}

\section{Aufgabe 26}

Beweisen Sie die Aussage $P(A \cup B) = P(A) + P(B) - P(A \cap B)$.
\begin{equation*}
    \begin{aligned}
        A \cup B &= (A \setminus B) \cup (B \setminus A) \cup (A \cap B) \\[5pt]
        P(A \setminus B) &= P(A) - P(A \cap B) \\[5pt]
        P(B \setminus A) &= P(B) - P(A \cap B) \\[5pt]
        P(A \cup B) &= P(A \setminus B) + P(B \setminus A) + P(A \cap B) = \\[5pt]
                    &= (P(A) - P(A \cap B)) + (P(B) - P(A \cap B)) + P(A \cap B) = \\[5pt]
                    &= P(A) + P(B) + P(A \cap B) - P(A \cap B) - P(A \cap B) = \\[5pt]
                    &= P(A) + P(B) - P(A \cap B) \quad \text{\underline{q.e.d.}}
    \end{aligned}
\end{equation*}

\end{document}
