\documentclass[10pt, oneside]{article}
\usepackage[a4paper, total={5.5in, 9in}]{geometry}
\usepackage[ngerman]{babel}

\usepackage{blindtext}
\usepackage{titlesec}
\usepackage{amsmath}
\usepackage[hidelinks]{hyperref}
\usepackage{parskip}
\usepackage{graphicx}
\usepackage{longtable}
\usepackage[shortlabels]{enumitem}
\usepackage{multirow}
\usepackage{nccmath}
\usepackage{rotating}
\usepackage{makecell}
\usepackage{multicol}
\usepackage{capt-of}
\usepackage{csquotes}
\usepackage{amsfonts}
\usepackage{caption}

\captionsetup[table]{position=bottom}

\titleformat{\section}
    {\normalfont\Large\bfseries}{}{0pt}{}

\let\oldsection\section
\renewcommand{\section}{
  \renewcommand{\theequation}{\thesection.\arabic{equation}}
  \oldsection}
\let\oldsubsection\subsection
\renewcommand{\subsection}{
  \renewcommand{\theequation}{\thesubsection.\arabic{equation}}
  \oldsubsection}

\makeatletter
\renewcommand{\maketitle}{
    \bgroup
    \centering
    \par\LARGE\@title  \\[20pt]
    \par\large\@author \\[10pt]
    \par\large\@date
    \par
    \egroup
}
\makeatother


\title{Wahrscheinlichkeitstheorie und Statistik\\[10pt]\Large{WiSe 2024/25}\\[15pt]\Large{L{\"o}sung zu der Aufgabe 59}}
\author{Volodymyr But\\[10pt]Hochschule Trier}
\date{}

% - - - - - - - - - - - - - - - - - - - - - - - - - - - - - - - - - - - - - - %

\begin{document}

\maketitle
\vspace{25px}

\section{Aufgabe 59}

Siehe Aufgabensammlung.
\begin{enumerate}[i)]
    \item
        $\begin{aligned}[t]
            P(A_1|B) &= \dfrac{P(B|A_1)P(A_1)}{P(B|A_1)P(A_1) + P(B|A_2)P(A_2) + P(B|A_3)P(A_3)} = \\
                     &= \dfrac{1 \cdot (1/3)}{1 \cdot (1/3) + (1/2) \cdot (1/3) + 0 \cdot (1/3)} = \dfrac{(1 / 3)}{(1/2)} = \dfrac{2}{3}
        \end{aligned}$
    \item
        $\begin{aligned}[t]
            P(A_2|B) &= \dfrac{P(B|A_2)P(A_2)}{P(B|A_1)P(A_1) + P(B|A_2)P(A_2) + P(B|A_3)P(A_3)} = \\
                     &= \dfrac{(1/2) \cdot (1/3)}{1 \cdot (1/3) + (1/2) \cdot (1/3) + 0 \cdot (1/3)} = \dfrac{(1/6)}{(1/2)} = \dfrac{1}{3}
        \end{aligned}$
\end{enumerate}

\end{document}
