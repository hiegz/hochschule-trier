\section{Aufgabe 20}
\setcounter{section}{20}

In einer Kiste mit Schrauben seien zu je gleichen Teilen Schrauben der St"arke
4 und 5 (mm) und zu je gleichen Teilen Schrauben der L"angen 30, 40 und 50
(mm). Es werde nun zuf"allig eine Schraube aus der Kiste genommen und sowohl
zwischen St"arke als auch der L"ange der Schraube unterschieden.

\begin{enumerate}[(a)]
    \item Geben Sie einen geeigneten W-Raum an, der dieses Experiment modelliert.
        \begin{equation*}
            \Omega = \{4, 5\} \times \{30, 40, 50\} = \{(4, 30), (4, 40), (4, 50), (5, 30), (5, 40), (5, 50)\}
        \end{equation*}
    \item Sie ben"otigen eine Schraube der St"arke 4 (mm) und von maximaler L"ange 40 (mm).
        Mit welcher Wahrscheinlichkeit ziehen Sie eine passende Schraube?
        \begin{equation*}
            A = \{(4, 30), (4, 40)\} \quad\quad
            P(A) = \dfrac{|A|}{|\Omega|} = \dfrac{2}{6} = \dfrac{1}{3} = 0.33
        \end{equation*}
\end{enumerate}

