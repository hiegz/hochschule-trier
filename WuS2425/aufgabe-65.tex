\section{Aufgabe 65}
\setcounter{section}{65}

In einem Casino sitzen zwei Spieler am Roulettetisch (m"ogliche Zahlen 0-36),
und es beschreibe $X$ den Gewinn von Spieler 1 und $Y$ den Gewinn von Spieler
2.
\begin{enumerate}[(a)]
    \item Spieler 1 setzt 10 Euro auf die Zahlen 1-18 und hat damit die Chance seinen
        Einsatz zu verdoppeln. Spieler 2 setzt 10 Euro auf die Zahlen 19-36 und hat
        damit auch die Chance seinen Einsatz zu verdoppeln. Bestimmen Sie $\text{C}(X, Y)$
        \begin{equation*}
            \begin{array}{lclclcrcl}
                 P(X = \phantom{-} 10 &\land& Y = \phantom{-} 10) &=& P(\{1,...,18\}  &\cap& \{19,...,36\}) &=& 0 \\[5pt]
                 P(X = \phantom{-} 10 &\land& Y =            -10) &=& P(\{1,...,18\}  &\cap& \{1,...,18\})  &=& 1/2 \\[5pt]
                 P(X =            -10 &\land& Y = \phantom{-} 10) &=& P(\{19,...,36\} &\cap& \{19,...,36\}) &=& 1/2 \\[5pt]
                 P(X =            -10 &\land& Y =            -10) &=& P(\{19,...,36\} &\cap& \{1,...,18\})  &=& 0
            \end{array}
        \end{equation*}
        Daraus ergibt sich eine Tabelle:

        \begin{table}[h]
            \centering
            \renewcommand{\arraystretch}{1.5}
            \begin{tabular}{c|c|c}
                $X \setminus Y$ & -10  & 10   \\ \hline
                -10             & 0    & 1/2  \\
                 10             & 1/2  & 0    \\
            \end{tabular}
            \caption{}
        \end{table}

        Die anschließende Lösung ist analog zur Aufgabe~\ref{sec:aufgabe-64}

    \item Spieler 1 setzt 10 Euro auf die Zahlen 1-12 und hat damit die Chance
        seinen Einsatz zu verdreifachen. Spieler 2 setzt 10 Euro auf die Zahlen
        1-18 und hat damit die Chance seinen Einsatz zu verdoppeln. Bestimmen
        Sie $C(X, Y)$
        \begin{equation*}
            \begin{array}{lclclcrcl}
                 P(X = \phantom{-} 20 &\land& Y = \phantom{-} 10) &=& P(\{1,...,12\}  &\cap& \{1,...,18\})  &=& 1/3 \\[5pt]
                 P(X = \phantom{-} 20 &\land& Y =            -10) &=& P(\{1,...,12\}  &\cap& \{19,...,36\}) &=& 0 \\[5pt]
                 P(X =            -10 &\land& Y = \phantom{-} 10) &=& P(\{13,...,36\} &\cap& \{1,...,18\})  &=& 1/6 \\[5pt]
                 P(X =            -10 &\land& Y =            -10) &=& P(\{13,...,36\} &\cap& \{19,...,36\})  &=& 1/2
            \end{array}
        \end{equation*}
        Daraus ergibt sich eine Tabelle:

        \begin{table}[h]
            \centering
            \renewcommand{\arraystretch}{1.5}
            \begin{tabular}{c|c|c}
                $X \setminus Y$ & -10  & 10   \\ \hline
                -10             & 1/2  & 1/6  \\
                 20             & 0    & 1/3  \\
            \end{tabular}
            \caption{}
        \end{table}

        Die anschließende Lösung ist analog zur Aufgabe~\ref{sec:aufgabe-64}
\end{enumerate}
