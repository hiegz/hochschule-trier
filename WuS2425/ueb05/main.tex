\documentclass[10pt, oneside]{article}
\usepackage[a4paper, total={5.5in, 9in}]{geometry}
\usepackage[ngerman]{babel}

\usepackage{blindtext}
\usepackage{titlesec}
\usepackage{amsmath}
\usepackage[hidelinks]{hyperref}
\usepackage{parskip}
\usepackage{graphicx}
\usepackage{longtable}
\usepackage[shortlabels]{enumitem}
\usepackage{multirow}
\usepackage{nccmath}
\usepackage{rotating}
\usepackage{makecell}
\usepackage{multicol}
\usepackage{capt-of}
\usepackage{csquotes}
\usepackage{amsfonts}
\usepackage{caption}

\captionsetup[table]{position=bottom}

\titleformat{\section}
    {\normalfont\Large\bfseries}{}{0pt}{}

\let\oldsection\section
\renewcommand{\section}{
  \renewcommand{\theequation}{\thesection.\arabic{equation}}
  \oldsection}
\let\oldsubsection\subsection
\renewcommand{\subsection}{
  \renewcommand{\theequation}{\thesubsection.\arabic{equation}}
  \oldsubsection}

\makeatletter
\renewcommand{\maketitle}{
    \bgroup
    \centering
    \par\LARGE\@title  \\[20pt]
    \par\large\@author \\[10pt]
    \par\large\@date
    \par
    \egroup
}
\makeatother


\title{Wahrscheinlichkeitstheorie und Statistik\\[10pt]\Large{WiSe 2024/25}\\[15pt]\Large{L{\"o}sungen zu den Aufgaben 52 und 53}}
\author{Volodymyr But}
\date{Hochschule Trier}

% - - - - - - - - - - - - - - - - - - - - - - - - - - - - - - - - - - - - - - %

\begin{document}
\sloppy

\maketitle
\vspace{25px}

\section{Aufgabe 52}

\begin{enumerate}
    \item Gegeben sei die Zufallsvariable $X$ mit den Werten und zugeh"origen Wahrscheinlichkeiten:
        \begin{table*}[h]
            \centering
            \begin{tabular}{c|c}
                $x_i$ & $P(X = x_i)$ \\
                \hline
                1 & 0.1 \\
                2 & 0.15 \\
                3 & 0.2 \\
                4 & 0.25 \\
                5 & 0.2 \\
                6 & 0.1
            \end{tabular}
        \end{table*} \\
        Berechnen Sie den Erwartungswert $E(X)$ von $X$.
        \begin{align*}
            E(X) := \sum_{i = 1}^n x_iP(X = x_i) &= 1 \cdot 0.1 + 2 \cdot 0.15 + 3 \cdot 0.2 + 4 \cdot 0.25 + 5 \cdot 0.2 + 6 \cdot 0.1 = \\
                                                 &= 3.6
        \end{align*}
    \item Eine Zufallsvariable $Y$ beschreibt die Anzahl der erfolgreichen
        Versuche in einer Stichprobe. Die m"oglichen Werte von $Y$ und deren
        Wahrscheinlichkeit sind:
        \begin{table*}[h]
            \centering
            \begin{tabular}{c|c}
                $x_i$ & $P(X = x_i)$ \\
                \hline
                0 & 0.05 \\
                1 & 0.15 \\
                2 & 0.25 \\
                3 & 0.3 \\
                4 & 0.15 \\
                5 & 0.1
            \end{tabular}
        \end{table*} \\
        Berechnen Sie den Erwartungswert $E(Y)$ von $Y$.
        \begin{align*}
            E(Y) := \sum_{i = 1}^n y_iP(Y = y_i) &= 0 \cdot 0.05 + 1 \cdot 0.15 + 2 \cdot 0.25 + 3 \cdot 0.3 + 4 \cdot 0.15 + 5 \cdot 0.1 = \\
                                                 &= 2.65
        \end{align*}
\end{enumerate}

\section{Aufgabe 53}

Gegeben ist eine Urne, die vier Kugeln enth"alt. Die Kugeln sind mit den Zahlen
1, 2, 3, 4 nummeriert. Es werden zwei Kugeln gleichzeitig und zuf"allig aus der
Urne gezogen, und anschlie{\ss}end werden die Nummern der beiden gezogenen
Kugeln addiert.
\begin{enumerate}
    \item Bestimmen Sie die m"oglichen Werte, die die Summe der gezogenen
        Kugelnummern annehmen kann, und berechnen Sie die Wahrscheinlichkeit
        f"ur jede dieser Summen.

        Sei $X : \Omega \rightarrow \mathbb{R}$ die Zufallsvariable, die die
        Summe der beiden gezogenen Kugelnummern annehmt, und $(\Omega, P)$ ein
        W-Raum, dann gilt:
        \begin{equation*}
            |\Omega| = \binom{4}{2} = 6 \quad\text{und}\quad P = \dfrac{1}{6}
        \end{equation*}
        \begin{table*}[h]
            \centering
            \begin{tabular}{c|c}
                $x_i$ & $P(X = x_i)$ \\
                \hline
                3 & 0.16 \\
                4 & 0.16 \\
                5 & 0.33 \\
                6 & 0.16 \\
                7 & 0.16 \\
            \end{tabular}
        \end{table*}
    \item Berechnen Sie den Erwartungswert der Summe der gezogenen Kugelnummern.
        \begin{equation*}
            E(X) = \dfrac{1}{6} \cdot (3 + 4 + 6 + 7) + \dfrac{1}{3} \cdot 5 = 5
        \end{equation*}
\end{enumerate}

\end{document}
