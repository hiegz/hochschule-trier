\documentclass[10pt, oneside]{article}
\usepackage[a4paper, total={5.5in, 9in}]{geometry}
\usepackage[ngerman]{babel}

\usepackage{blindtext}
\usepackage{titling}
\usepackage{titlesec}
\usepackage{amsmath}
\usepackage[hidelinks]{hyperref}
\usepackage{parskip}
\usepackage{graphicx}

\setlength{\droptitle}{-3cm}

\titleformat{\section}
    {\normalfont\Large\bfseries}{}{0pt}{}

\let\oldsection\section
\renewcommand{\section}{
  \renewcommand{\theequation}{\thesection.\arabic{equation}}
  \oldsection}
\let\oldsubsection\subsection
\renewcommand{\subsection}{
  \renewcommand{\theequation}{\thesubsection.\arabic{equation}}
  \oldsubsection}


\title{Wahrscheinlichkeitstheorie und Statistik\\[10pt]\Large{WiSe 2024/25}\\[15pt]\Large{L{\"o}sungen zu den Aufgaben 35, 36, 55 und 56}}
\author{Volodymyr But\\[10pt]Hochschule Trier}
\date{}

% - - - - - - - - - - - - - - - - - - - - - - - - - - - - - - - - - - - - - - %

\begin{document}
\sloppy

\maketitle
\vspace{25px}

\section{Aufgabe 35}

Ein neues Online-Lernprogramm soll getestet werden, der laut Anbieter dazu
f"uhren, dass 70\% der Sch"uler eine Verbesserung in ihren Testergebnissen
erzielen. Eine Stichprobe von $n = 40$ Sch"ulern wird zuf"allig ausgew"ahlt und
mit dem Programm geschult. Gegeben sei, dass in der Stichprobe 32 von 40
Sch"ulern eine Verbesserung ihrer Testergebnisse erzielen.

Bestimmen Sie bei einem Signifikanzniveau von $\alpha = 0.05$ f"ur jede
Teilfrage, ob die jeweilige Nullhypothese $H_0$ abgelehnt wird.

\begin{enumerate}[(a)]
    \item Wir nehmen an, dass das Programm tats"achlich eine Erfolgsquote von
        70\% besitzt.
        \begin{enumerate}[i)]
            \item Nullhypothese $H_0$ : $p = 0.7$
            \item Alternativhypothese $H_1$ : $p \neq 0.7$
        \end{enumerate}
        Wir w"ahlen dann das gr"o{\ss}tm"ogliche $k_0, k_1 \in \{0,...,40\}$,
        sodass
        \begin{equation*}
            P(X \leq k_0) + P(X \geq n - k_1) \leq \alpha
        \end{equation*}
        Es ergibt sich damit
        \begin{equation*}
            A := \{0,...,21\} \cup \{34,...,40\} \quad \text{Ablehnungsbereich}
        \end{equation*}
        Daraus folgt mit $T(X) = 32$
        \begin{equation*}
            T(X) \notin A \implies H_0 \text{ wird nicht abgelehnt.}
        \end{equation*}
    \item Das Ziel ist auszuschlie{\ss}en, dass die Erfolgsquote unter 60\% liegt.
        \begin{enumerate}[i)]
            \item Nullhypothese $H_0$ : $p \leq 0.6$
            \item Alternativhypothese $H_1$ : $p > 0.6$
        \end{enumerate}
        Wir w"ahlen dann das gr"o{\ss}tm"ogliche $k_0 \in \{0,...,40\}$,
        sodass
        \begin{equation*}
            P(X \geq n - k_0) \leq \alpha
        \end{equation*}
        Es ergibt sich damit
        \begin{equation*}
            A := \{30,...,40\} \quad \text{Ablehnungsbereich}
        \end{equation*}
        Daraus folgt mit $T(X) = 32$
        \begin{equation*}
            T(X) \in A \implies H_0 \text{ wird abgelehnt.}
        \end{equation*}
    \item Das Ziel ist zu testen, ob die Erfolgsquote m"oglicherweise unter 90\%
        liegt.
        \begin{enumerate}[i)]
            \item Nullhypothese $H_0$ : $p \geq 0.9$
            \item Alternativhypothese $H_1$ : $p < 0.9$
        \end{enumerate}
        Wir w"ahlen dann das gr"o{\ss}tm"ogliche $k_0 \in \{0,...,40\}$,
        sodass
        \begin{equation*}
            P(X \leq k_0) \leq \alpha
        \end{equation*}
        Es ergibt sich damit
        \begin{equation*}
            A := \{0,...,32\} \quad \text{Ablehnungsbereich}
        \end{equation*}
        Daraus folgt mit $T(X) = 32$
        \begin{equation*}
            T(X) \in A \implies H_0 \text{ wird abgelehnt.}
        \end{equation*}
\end{enumerate}

% \section{Aufgabe 36}
% 
% In einer Produktionslinie wird eine Ausschussrate von 10\% vermutet, und der
% Hersteller m"ochte sicherstellen, dass diese Rate nicht "uberschritten wird.
% Ein Test wird durchgef"uhrt, um die Nullhypothese $H_0 : p = 0.05$ zu testen.
% Festgelegt wird ein Signifikanzniveau von $\alpha = 0.05$. Bestimmen Sie den
% kleinsten Stichprobenumfang $n \in \{10,20,...,100\}$ damit der Test gegen die
% Alternativhypothese $p = 0.2$ mit Testst"arke $90\%$ durchgef"uhrt werden kann.
% 
% --

\section{Aufgabe 55}

\begin{enumerate}
    \item
        \begin{enumerate}[(a)]
            \item $\text{E}(X) = \sum_{i = 1}^n x_iP(X = x_i) = 3.8$
            \item $\text{Var}(X) = \text{E}(X - \text{E}(X))^2 = 3.404 (1.96)$
        \end{enumerate}

    \item
        \begin{enumerate}[(a)]
            \item $\text{E}(X) = \sum_{i = 1}^n x_iP(X = x_i) = 2$
            \item $\text{Var}(X) = \text{E}(X - \text{E}(X))^2 = 2.59 (2.6)$
        \end{enumerate}
\end{enumerate}

% \section{Aufgabe 56}
% 
% In einer Urne seien 20 wei{\ss}e und 5 rote Kugeln. Sie d"urfen f"ur den
% Einsatz von 20 Euro einmal ziehen. Falls Sie eine rote Kugel ziehen erhalten
% Sie 300 Euro.
% \begin{enumerate}[(a)]
%     \item Es beschreibe $X$ Ihren Gewinn bzw. Verlust in diesem Spiel.
%         Berechnen Sie $\text{E}(X)$ und $\text{Var}(X)$.
%     \item Nach Ihnen d"urfen zwei weitere Spieler eine Kugel ziehen, wobei die
%         gezogenen Kugeln nicht zur"uckgelegt werden. Falls Sie gewinnen und
%         au{\ss}erdem einer oder zwei dieses Spiler gewinnen, m"ussen Sie den
%         Gewinn teilen bzw. dritteln. Es beschreibe $X$ Ihren Gewinn bzw. Verlust.
%         Berechnen Sie $\text{E}(X)$ und $\text{Var}(X)$ in dieser neuen Situation.
% \end{enumerate}

\end{document}
