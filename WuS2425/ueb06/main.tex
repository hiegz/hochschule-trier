\documentclass[10pt, oneside]{article}
\usepackage[a4paper, total={5.5in, 9in}]{geometry}
\usepackage[ngerman]{babel}

\usepackage{blindtext}
\usepackage{titlesec}
\usepackage{amsmath}
\usepackage[hidelinks]{hyperref}
\usepackage{parskip}
\usepackage{graphicx}
\usepackage{longtable}
\usepackage[shortlabels]{enumitem}
\usepackage{multirow}
\usepackage{nccmath}
\usepackage{rotating}
\usepackage{makecell}
\usepackage{multicol}
\usepackage{capt-of}
\usepackage{csquotes}
\usepackage{amsfonts}
\usepackage{caption}

\captionsetup[table]{position=bottom}

\titleformat{\section}
    {\normalfont\Large\bfseries}{}{0pt}{}

\let\oldsection\section
\renewcommand{\section}{
  \renewcommand{\theequation}{\thesection.\arabic{equation}}
  \oldsection}
\let\oldsubsection\subsection
\renewcommand{\subsection}{
  \renewcommand{\theequation}{\thesubsection.\arabic{equation}}
  \oldsubsection}

\makeatletter
\renewcommand{\maketitle}{
    \bgroup
    \centering
    \par\LARGE\@title  \\[20pt]
    \par\large\@author \\[10pt]
    \par\large\@date
    \par
    \egroup
}
\makeatother


\title{Wahrscheinlichkeitstheorie und Statistik\\[10pt]\Large{WiSe 2024/25}\\[15pt]\Large{L{\"o}sungen zu den Aufgaben 35 und 55}}
\author{Volodymyr But\\[10pt]Hochschule Trier}
\date{}

% - - - - - - - - - - - - - - - - - - - - - - - - - - - - - - - - - - - - - - %

\begin{document}
\sloppy

\maketitle
\vspace{25px}

\section{Aufgabe 35}

Ein neues Online-Lernprogramm soll getestet werden, der laut Anbieter dazu
f"uhren, dass 70\% der Sch"uler eine Verbesserung in ihren Testergebnissen
erzielen. Eine Stichprobe von $n = 40$ Sch"ulern wird zuf"allig ausgew"ahlt und
mit dem Programm geschult. Gegeben sei, dass in der Stichprobe 32 von 40
Sch"ulern eine Verbesserung ihrer Testergebnisse erzielen.

Bestimmen Sie bei einem Signifikanzniveau von $\alpha = 0.05$ f"ur jede
Teilfrage, ob die jeweilige Nullhypothese $H_0$ abgelehnt wird.

\begin{enumerate}[(a)]
    \item Wir nehmen an, dass das Programm tats"achlich eine Erfolgsquote von
        70\% besitzt.
        \begin{enumerate}[i)]
            \item Nullhypothese $H_0$ : $p = 0.7$
            \item Alternativhypothese $H_1$ : $p \neq 0.7$
        \end{enumerate}
        Wir w"ahlen dann das gr"o{\ss}tm"ogliche $k_0, k_1 \in \{0,...,40\}$,
        sodass
        \begin{equation*}
            P(X \leq k_0) + P(X \geq n - k_1) \leq \alpha
        \end{equation*}
        Es ergibt sich damit
        \begin{equation*}
            A := \{0,...,21\} \cup \{34,...,40\} \quad \text{Ablehnungsbereich}
        \end{equation*}
        Daraus folgt mit $T(X) = 32$
        \begin{equation*}
            T(X) \notin A \implies H_0 \text{ wird nicht abgelehnt.}
        \end{equation*}
    \item Das Ziel ist auszuschlie{\ss}en, dass die Erfolgsquote unter 60\% liegt.
        \begin{enumerate}[i)]
            \item Nullhypothese $H_0$ : $p \leq 0.6$
            \item Alternativhypothese $H_1$ : $p > 0.6$
        \end{enumerate}
        Wir w"ahlen dann das gr"o{\ss}tm"ogliche $k_0 \in \{0,...,40\}$,
        sodass
        \begin{equation*}
            P(X \geq n - k_0) \leq \alpha
        \end{equation*}
        Es ergibt sich damit
        \begin{equation*}
            A := \{30,...,40\} \quad \text{Ablehnungsbereich}
        \end{equation*}
        Daraus folgt mit $T(X) = 32$
        \begin{equation*}
            T(X) \in A \implies H_0 \text{ wird abgelehnt.}
        \end{equation*}
    \item Das Ziel ist zu testen, ob die Erfolgsquote m"oglicherweise unter 90\%
        liegt.
        \begin{enumerate}[i)]
            \item Nullhypothese $H_0$ : $p \geq 0.9$
            \item Alternativhypothese $H_1$ : $p < 0.9$
        \end{enumerate}
        Wir w"ahlen dann das gr"o{\ss}tm"ogliche $k_0 \in \{0,...,40\}$,
        sodass
        \begin{equation*}
            P(X \leq k_0) \leq \alpha
        \end{equation*}
        Es ergibt sich damit
        \begin{equation*}
            A := \{0,...,32\} \quad \text{Ablehnungsbereich}
        \end{equation*}
        Daraus folgt mit $T(X) = 32$
        \begin{equation*}
            T(X) \in A \implies H_0 \text{ wird abgelehnt.}
        \end{equation*}
\end{enumerate}

\section{Aufgabe 55}

\begin{enumerate}
    \item $\sum P(X = x_I) > 1 \Rightarrow \emptyset$
        % \begin{enumerate}[(a)]
        %     \item $\text{E}(X) = \sum_{i = 1}^n x_iP(X = x_i) = 3.8$
        %     \item $\text{Var}(X) = \text{E}(X - \text{E}(X))^2 = 3.404 (1.96)$
        % \end{enumerate}

    \item
        \begin{enumerate}[(a)]
            \item $\text{E}(X) = \sum_{i = 1}^n x_iP(X = x_i) = 2$
            \item $\text{Var}(X) = \text{E}(X - \text{E}(X))^2 = 2.59$
        \end{enumerate}
\end{enumerate}

\end{document}
