\section{Aufgabe 16}
\setcounter{section}{16}

In einer Lotterie werden 6 nummerierte Kugeln aus 49 Kugeln
ungeordnet gezogen. Wie viele Ziehung gibt es, welche die Zahlen
1, 2, 3 enthalten.

Unter der Annahme, dass die Reihenfolge der Ziehungen unerheblich
ist, lässt sich feststellen, dass sich das Ergebnis, bei dem die
ersten drei gezogenen Kugeln die Zahlen 1, 2 und 3 enthalten,
nicht von dem Ergebnis unterscheidet, bei dem diese Kugeln in
einer anderen Reihenfolge gezogen wurden. Es sollte also gen"ugen,
die Anzahl der Möglichkeiten zu berechnen, drei weitere Kugeln zu
ziehen, wenn man davon ausgeht, dass die drei Zielkugeln bereits
gezogen wurden.

\begin{equation*}
    \begin{aligned}
        |\text{Ur}^n_k(oR, oZ)| &= |\text{Fa}^n_k(nuD, oM)| = \\[5pt]
                                &= \binom{n}{k} = \binom{46}{3} = \\[5pt]
                                &= \dfrac{46!}{3!\cdot43!} = \dfrac{46 \cdot 45 \cdot 44}{3 \cdot 2} = 15180
    \end{aligned}
\end{equation*}
