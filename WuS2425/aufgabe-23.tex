\section{Aufgabe 23}
\setcounter{section}{23}

In einer Urne liegen 90 rote Kugeln und 10 wei{\ss}e Kugeln. In einem
Experiment werde 6 mal eine Kugel mit Zur"ucklegen gezogen.

\begin{enumerate}[(a)]
    \item Definieren Sie einen Wahrscheinlichkeitsraum, der dieses Experiment modelliert.
        \begin{equation*}
            |\Omega| = |\text{Ur}_6^{100}(mR, mZ)| = 100^6 = 10^{12}
        \end{equation*}
    \item Definieren Sie auf ihrem W-Raum aus Teil (a) eine
        Zufallsvariable $X$, die die Anzahl der wei{\ss}en Kugeln in diesem
        Experiment z"ahlt, und berechnen Sie damit $P(X = 2)$, $P(X > 4)$ und
        $P(X \leq 2)$.

        Sei $X$ die Zufallsvariable, die die Anzahl der wei{\ss}en Kugeln z"ahlt. Dann gilt:
        \begin{equation*}
            P(X = k) = \left(\dfrac{1}{10}\right)^k
        \end{equation*}
\end{enumerate}
