\section{Aufgabe 59}
\setcounter{section}{59}

In einer Urne seien 20 wei{\ss}e und 5 rote Kugeln. Sie d"urfen f"ur den
Einsatz von 20 Euro einmal ziehen. Falls Sie eine rote Kugel ziehen erhalten
Sie 300 Euro.
\begin{enumerate}[(a)]
    \item Es beschreibe $X$ Ihren Gewinn bzw. Verlust in diesem Spiel.
        Berechnen Sie $\text{E}(X)$ und $\text{Var}(X)$.
        \begin{align*}
            E(X)          &= \dfrac{4}{5} \cdot (-20) + \dfrac{1}{5} \cdot 280 = -16 + 56 = 40 \\[5pt]
            \text{Var}(X) &= \dfrac{4}{5} \cdot (-20 - 40)^2 + \dfrac{1}{5} \cdot (280 - 40)^2 = 2880 + 11520 = 14400
        \end{align*}
    \item Nach Ihnen d"urfen zwei weitere Spieler eine Kugel ziehen, wobei die
        gezogenen Kugeln nicht zur"uckgelegt werden. Falls Sie gewinnen und
        au{\ss}erdem einer oder zwei dieses Spiler gewinnen, m"ussen Sie den
        Gewinn teilen bzw. dritteln. Es beschreibe $X$ Ihren Gewinn bzw. Verlust.
        Berechnen Sie $\text{E}(X)$ und $\text{Var}(X)$ in dieser neuen Situation.

        \begin{align*}
            E(X)          &= \dfrac{4}{5} \cdot (-20) + \dfrac{19}{138} \cdot 280 + \dfrac{4}{69} \cdot 130 + \dfrac{1}{230} \cdot 80 = 30.4 \\[5pt]
            \text{Var}(X) &= \dfrac{4}{5} \cdot (-20 - 30.4)^2 + \dfrac{19}{138} \cdot (280 - 30.4)^2\ + \\[5pt]
                          &+ \dfrac{4}{69} \cdot (130 - 30.4)^2 + \dfrac{1}{230} \cdot (80 - 30.4)^2 = 11195.5
        \end{align*}
\end{enumerate}
