\section{Aufgabe 22}
\setcounter{section}{22}

In einer Urne liegen 10 nummerierte Kugeln. In einem Experiment werden zwei
Kugeln ohne Zurücklegen gezogen (ungeordnete Stichproben).
\begin{enumerate}[(a)]
    \item Geben Sie eine geeigneten Laplaceschen W-Raum an, der dieses
        Experiment modelliert.

        Wählt man ohne Zurücklegen und ohne Berücksichtigung der Reihenfolge
        aus 10 Kugeln 2 aus, ergibt sich der Wahrscheinlichkeitsraum:
        \begin{equation*}
            |\Omega| = \binom{10}{2} = 45 \quad\text{und}\quad p(\omega) = \dfrac{1}{|\Omega|} = \dfrac{1}{45} = 0.02 \quad (\forall_{\omega \in \Omega})
        \end{equation*}

    \item Berechnen Sie die Wahrscheinlichkeit, dass die Größere der gezogenen
        Zahlen kleiner 6 ist. Definieren Sie dazu eine geeignete
        Zufallsvariable und wenden Sie die Notation aus 6.10 bzw. 6.11 aus dem
        Skript an.

        Definieren wir die Zufallsvariable $X: \Omega \rightarrow \{2,...,10\}$ durch
        \begin{equation*}
            X(\Omega) := \{\max(\omega_1, \omega_2) : (\omega_1, \omega_2) \in \Omega\},
        \end{equation*}
        bezeichnet $\omega \in X(\Omega)$ die Gr"o{\ss}ere der beiden gezogenen
        Zahlen. Dann entspricht der Wert von $P(X < 6)$ der Wahrscheinlichkeit,
        dass sie kleiner als 6 ist. Ist $A := \{2,...,5\}$, dann gilt
        gem"a{\ss} der Definition 3.10 und 3.11 aus dem Skript
        \begin{equation*}
            P(X < 6) := P(X^{-1}(A))
        \end{equation*}
        wobei $X^{-1}(A) := \{(\omega_1, \omega_2) \in \Omega : \max(\omega_1, \omega_2) < 6\}$.

        Aus der Definition von $X^{-1}(A)$ folgt, dass
        \begin{equation*}
            \forall_{(\omega_1, \omega_2) \in X^{-1}(A)} : 1 \leq \min(\omega_1, \omega_2) < \max(\omega_1, \omega_2) < 6
        \end{equation*}
        Das bedeutet also, dass sowohl $\omega_1$ als auch $\omega_2$ mit $(\omega_1, \omega_2) \in X^{-1}(A)$ aus der Menge [1, 6)
        ausgew"ahlt werden. Daher entsteht der Ereignisraum $P$ definiert durch:
        \begin{equation*}
            X^{-1}(A) = P\text{,} \quad P := \text{Ur}_2^5(oR, oZ)\text{,} \quad |P| = \binom{5}{2} = 10
        \end{equation*}
        Und da der W-Raum $(\Omega, p(\omega))$ gleichverteilt ist, gilt somit
        \begin{equation*}
            P(X < 6) = P(X^{-1}(A)) = \dfrac{|X^{-1}(A)|}{|\Omega|} = \dfrac{|P|}{|\Omega|} = \dfrac{10}{45} = \dfrac{2}{9} \approx 0.22
        \end{equation*}

    \item Es bezeichne nun $(\Omega,P)$ Ihren W-Raum aus Teil (a) und es sei
        $X:\Omega\to\{0,1\}$ für $(k_1,k_2)\in\Omega$ definiert durch
        $X((k_1,k_2))=1$, falls entweder $k_1$ oder $k_2$ gleich 3 sind, und
        $X((k_1,k_2))=0$ andernfalls. Bestimmen Sie die Zähldichte zu $P^X.$

        In der Aufgabestellung ist die Zufallsvariable $X: \Omega \rightarrow \{0, 1\}$ definiert durch:
        \begin{align*}
            X(\{(k_1, k_2)\}) &:= \begin{cases}
                1\text{,}& \text{ falls } k_1 = 3 \lor k_2 = 3\text{,} \\
                0\text{,}& \text{ sonst, }
            \end{cases}  \\
            \intertext{oder anders ausgedr"uckt als}
            X(\{(k_1, k_2)\}) &:= \begin{cases}
                1\text{,}& \text{ falls } (k_1, k_2) \in M \lor (k_2, k_1) \in M\text{,} \\
                0\text{,}& \text{ sonst, }
            \end{cases}
        \end{align*}
        wobei $M$ den Ereignisraum bezeichnet, der alle Ziehungen umfasst, in
        denen eine von den beiden gezogenen Kugeln mit der Nummer 3 nummeriert
        ist. Dies l"asst sich wie folgt darstellen:
        \begin{equation*}
            M := \{3\} \times (\{1,...,10\} \setminus \{3\})\text{,} \quad |M| = 9 \quad \text{mit} \quad P(M) = \dfrac{|M|}{|\Omega|} = \dfrac{9}{45} = 0.2
        \end{equation*}
        Daraus ergibt sich:
        \begin{equation*}
            \begin{array}{rccl}
                &P^X(\{a\}) &=& \begin{cases}
                    P(M)\text{,}& \text{ falls } a = 1, \\
                    1 - P(M)\text{,}& \text{ falls } a = 0
                \end{cases} \\[20pt]
                p_x(a) := &P^X(\{a\}) &\implies& p_x(0) = 0.8, \quad p_x(1) = 0.2
            \end{array}
        \end{equation*}
\end{enumerate}
