\documentclass[10pt, oneside]{article}
\usepackage[a4paper, total={5.5in, 9in}]{geometry}
\usepackage[ngerman]{babel}

\usepackage{blindtext}
\usepackage{titlesec}
\usepackage{amsmath}
\usepackage[hidelinks]{hyperref}
\usepackage{parskip}
\usepackage{graphicx}
\usepackage{longtable}
\usepackage[shortlabels]{enumitem}
\usepackage{multirow}
\usepackage{nccmath}
\usepackage{rotating}
\usepackage{makecell}
\usepackage{multicol}
\usepackage{capt-of}
\usepackage{csquotes}
\usepackage{amsfonts}
\usepackage{caption}

\captionsetup[table]{position=bottom}

\titleformat{\section}
    {\normalfont\Large\bfseries}{}{0pt}{}

\let\oldsection\section
\renewcommand{\section}{
  \renewcommand{\theequation}{\thesection.\arabic{equation}}
  \oldsection}
\let\oldsubsection\subsection
\renewcommand{\subsection}{
  \renewcommand{\theequation}{\thesubsection.\arabic{equation}}
  \oldsubsection}

\makeatletter
\renewcommand{\maketitle}{
    \bgroup
    \centering
    \par\LARGE\@title  \\[20pt]
    \par\large\@author \\[10pt]
    \par\large\@date
    \par
    \egroup
}
\makeatother


\title{Wahrscheinlichkeitstheorie und Statistik\\[10pt]\Large WiSe 2024/25\\[15pt]\Large L{\"o}sungen zu den Aufgaben 37 und 39}
\author{Volodymyr But\\[10pt]Hochschule Trier}
\date{}

% - - - - - - - - - - - - - - - - - - - - - - - - - - - - - - - - - - - - - - %

\begin{document}

\maketitle
\vspace{25px}

\section{Aufgabe 37}

Ein Medikamenthersteller behauptet, dass sein Produkt eine Wirksamkeit von 80\%
besitzt. Um dies zu "uberpr"ufen, wird ein Test durchgef"uhrt, um die
Nullhypothese $H_0 : p = 0.8$ zu testen. Festgelegt wird ein Signifikanzniveau
von $\alpha$ = 0.01. Bestimmen Sie den kleinsten Stichprobenumfang $n \in \{10,
20, ..., 100\}$, damit der Test gegen die Alternativhypothese $p = 0.6$ mit
einer Testst"arke von 95\% durchgef"uhrt werden kann.

$\quad\rightarrow n = 94$ $(100)$

\section{Aufgabe 39}

Betrachten Sie einen Binomialtest zur Hypothese $H_0 : p = 0.4$ und
Signifikanzniveau $\alpha = 0.1$. Bestimmen Sie Testst"arke f"ur folgende
alternative Hypothesen und Stichprobenumf"ange $n$:

\begin{enumerate}
    \item $H_1 : p = 0.5$ und $n = 50$
        $\rightarrow$ 33\%
    \item $H_1 : p = 0.6$ und $n = 40$
        $\rightarrow$ 79\%
    \item $H_1 : p = 0.7$ und $n = 30$
        $\rightarrow$ 95\%
\end{enumerate}

% \section{Aufgabe 42}
%
% Finden Sie ein Beispiel eines W-Raums $(\Omega, P)$ und $A, B, C \subset
% \Omega$ mit $P(B \cap C) > 0$, so dass $A$ und $C$ stochastisch unabh"angig
% sind und $$P(A|B \cap C) \neq P(A|B)$$ gilt.

% \section{Aufgabe 56}
%
% In einer Urne seien 20 wei{\ss}e und 5 rote Kugeln. Sie d"urfen f"ur den
% Einsatz von 20 Euro einmal ziehen. Falls Sie eine rote Kugel ziehen erhalten
% Sie 300 Euro.
% \begin{enumerate}[(a)]
%     \item Es beschreibe $X$ Ihren Gewinn bzw. Verlust in diesem Spiel.
%         Berechnen Sie $\text{E}(X)$ und $\text{Var}(X)$.
%     \item Nach Ihnen d"urfen zwei weitere Spieler eine Kugel ziehen, wobei die
%         gezogenen Kugeln nicht zur"uckgelegt werden. Falls Sie gewinnen und
%         au{\ss}erdem einer oder zwei dieses Spiler gewinnen, m"ussen Sie den
%         Gewinn teilen bzw. dritteln. Es beschreibe $X$ Ihren Gewinn bzw. Verlust.
%         Berechnen Sie $\text{E}(X)$ und $\text{Var}(X)$ in dieser neuen Situation.
% \end{enumerate}

\end{document}
