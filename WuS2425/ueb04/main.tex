\documentclass[10pt, oneside]{article}
\usepackage[a4paper, total={5.5in, 9in}]{geometry}
\usepackage[ngerman]{babel}

\usepackage{blindtext}
\usepackage{titling}
\usepackage{titlesec}
\usepackage{amsmath}
\usepackage[hidelinks]{hyperref}
\usepackage{parskip}
\usepackage{graphicx}

\setlength{\droptitle}{-3cm}

\titleformat{\section}
    {\normalfont\Large\bfseries}{}{0pt}{}

\let\oldsection\section
\renewcommand{\section}{
  \renewcommand{\theequation}{\thesection.\arabic{equation}}
  \oldsection}
\let\oldsubsection\subsection
\renewcommand{\subsection}{
  \renewcommand{\theequation}{\thesubsection.\arabic{equation}}
  \oldsubsection}


\title{Wahrscheinlichkeitstheorie und Statistik\\[10pt]\Large WiSe 2024/25\\[15pt]\Large L{\"o}sungen zu den Aufgaben 21, 22 und 30}
\author{Volodymyr But}
\date{Hochschule Trier}

% - - - - - - - - - - - - - - - - - - - - - - - - - - - - - - - - - - - - - - %

\begin{document}
\sloppy

\maketitle
\vspace{25px}

\section{Aufgabe 21}
\setcounter{section}{21}
In einer Lotterie werden 3 Kugeln aus 100 nummerierten Kugeln ohne Zurücklegen
gezogen (ungeordnete Stichproben). Ein Tipp eines Teilnehmers der Lotterie
bezeichne das Auswählen dreier unterschiedlicher Zahlen aus $\{1,...,100\}$
(vor der Ziehung).
\begin{enumerate}[(a)]
    \item Geben Sie einen geeigneten Laplaceschen W-Raum an, der dieses
        Experiment modelliert.
    \item Nehmen Sie an, dass Sie drei (unterschiedliche) Tipps abgegeben haben.
        Berechnen Sie mit dem W-Raum aus Teil (a) die Wahrscheinlichkeit, dass
        einer Ihrer Tipps mit der Ziehung übereinstimmt.
    \item Nehmen Sie an, dass Sie einen Tipp abgegeben haben. Berechnen Sie mit dem
        W-Raum aus Teil (a) die Wahrscheinlichkeit, dass mindestens zwei der
        Zahlen Ihres Tipps mit der Ziehung übereinstimmen.
\end{enumerate}
\textbf{L"osung.}
\begin{enumerate}[(a)]
    \item Wählt man aus 100 Kugeln 3 aus, ohne sie zurückzulegen und ohne auf
        ihre Reihenfolge zu achten, dann gilt:
        \begin{equation*}
            |\Omega| = \binom{100}{3} = 161700 \quad\text{und}\quad p(\omega) = \dfrac{1}{|\Omega|} = \dfrac{1}{161700} = 6.18 \cdot 10^{-6} \quad (\forall_{\omega \in \Omega})
        \end{equation*}
    \item Sei $M \subset \Omega$ die Menge der drei unterschiedlichen Tipps, dann gilt:
        \begin{equation*}
            |M| = 3 \quad\text{und}\quad P(M) = \dfrac{|M|}{|\Omega|} = \dfrac{3}{161700} \approx 1.85 \cdot 10^{-5}
        \end{equation*}
        wobei $P(M)$ die Wahrscheinlichkeit ist, dass einer dieser Tipps mit der Ziehung "ubereinstimmt.
    \pagebreak
    \item Sei $X \sim \text{HV}(100, 3, 3)$ eine Zufallsvariable, die die
        Anzahl der Zahlen in dem Tipp, die mit der Ziehung "ubereinstimmen.
        Dann gilt:
        \begin{equation}
            P(X = k) = \dfrac{\binom{K}{k}\binom{N - K}{n - k}}{\binom{N}{n}}
        \end{equation}
        wo
        \begin{enumerate}[-]
            \item $N$ die Gesamtanzahl der Kugeln ist,
            \item $K$ die Gesamtanzahl der Erfolge ist,
            \item $n$ die Anzahl der Ziehungen ist,
            \item $k$ die Anzahl der beobachteten Erfolge ist.
        \end{enumerate}
        \vspace{5pt}
        Daher folgt:
        \begin{equation*}
            \begin{array}{rcccl}
                P(X \geq 2) &=& P(X = 2) + P(X = 3) &=& \\[7.5pt]
                            &=& \dfrac{\binom{3}{2}\binom{100 - 3}{3 - 2}}{\binom{100}{3}} + \dfrac{\binom{3}{3}\binom{100 - 3}{3 - 3}}{\binom{100}{3}} &\approx& \\[15pt]
                      &\approx& 0.0017996 + 6.18 \cdot 10^{-6} &\approx& 0.0018058
            \end{array}
        \end{equation*}
\end{enumerate}

\section{Aufgabe 22}
In einer Urne liegen 10 nummerierte Kugeln. In einem Experiment werden zwei
Kugeln ohne Zurücklegen gezogen (ungeordnete Stichproben).
\begin{enumerate}[(a)]
    \item Geben Sie eine geeigneten Laplaceschen W-Raum an, der dieses
        Experiment modelliert.
    \item Berechnen Sie die Wahrscheinlichkeit, dass die Größere der gezogenen
        Zahlen kleiner 6 ist. Definieren Sie dazu eine geeignete
        Zufallsvariable und wenden Sie die Notation aus 6.10 bzw. 6.11 aus dem
        Skript an.
    \item Es bezeichne nun $(\Omega,P)$ Ihren W-Raum aus Teil (a) und es sei
        $X:\Omega\to\{0,1\}$ für $(k_1,k_2)\in\Omega$ definiert durch
        $X((k_1,k_2))=1$, falls entweder $k_1$ oder $k_2$ gleich 3 sind, und
        $X((k_1,k_2))=0$ andernfalls. Bestimmen Sie die Zähldichte zu $P^X.$
\end{enumerate}
\textbf{L"osung.}
\begin{enumerate}[(a)]
    \item Wählt man ohne Zurücklegen und ohne Berücksichtigung der Reihenfolge
        aus 10 Kugeln 2 aus, ergibt sich der Wahrscheinlichkeitsraum:
        \begin{equation*}
            |\Omega| = \binom{10}{2} = 45 \quad\text{und}\quad p(\omega) = \dfrac{1}{|\Omega|} = \dfrac{1}{45} = 0.02 \quad (\forall_{\omega \in \Omega})
        \end{equation*}
    \item Definieren wir die Zufallsvariable $X: \Omega \rightarrow \{2,...,10\}$ durch
        \begin{equation*}
            X(\Omega) := \{\max(\omega_1, \omega_2) : (\omega_1, \omega_2) \in \Omega\},
        \end{equation*}
        bezeichnet $\omega \in X(\Omega)$ die Gr"o{\ss}ere der beiden gezogenen
        Zahlen. Dann entspricht der Wert von $P(X < 6)$ der Wahrscheinlichkeit,
        dass sie kleiner als 6 ist. Ist $A := \{2,...,5\}$, dann gilt
        gem"a{\ss} der Definition 3.10 und 3.11 aus dem Skript
        \begin{equation*}
            P(X < 6) := P(X^{-1}(A))
        \end{equation*}
        wobei $X^{-1}(A) := \{(\omega_1, \omega_2) \in \Omega : \max(\omega_1, \omega_2) < 6\}$.

        Aus der Definition von $X^{-1}(A)$ folgt, dass
        \begin{equation*}
            \forall_{(\omega_1, \omega_2) \in X^{-1}(A)} : 1 \leq \min(\omega_1, \omega_2) < \max(\omega_1, \omega_2) < 6
        \end{equation*}
        Das bedeutet also, dass sowohl $\omega_1$ als auch $\omega_2$ mit $(\omega_1, \omega_2) \in X^{-1}(A)$ aus der Menge [1, 6)
        ausgew"ahlt werden. Daher entsteht der Ereignisraum $P$ definiert durch:
        \begin{equation*}
            X^{-1}(A) = P\text{,} \quad P := \text{Ur}_2^5(oR, oZ)\text{,} \quad |P| = \binom{5}{2} = 10
        \end{equation*}
        Und da der W-Raum $(\Omega, p(\omega))$ gleichverteilt ist, gilt somit
        \begin{equation*}
            P(X < 6) = P(X^{-1}(A)) = \dfrac{|X^{-1}(A)|}{|\Omega|} = \dfrac{|P|}{|\Omega|} = \dfrac{10}{45} = \dfrac{2}{9} \approx 0.22
        \end{equation*}
    \item % W"ahlen wir zwei Kugel derart, dass eine von denen mit 3 nummeriert
          % ist, ergibt sich ein Ereignisraum $M$:
          % \begin{equation*}
          %     M := \{3\} \times (\{1,...,10\} \setminus \{3\})\text{,} \quad |M| = 9 \quad \text{mit} \quad P(M) = \dfrac{|M|}{|\Omega|} = \dfrac{9}{45} = 0.2
          % \end{equation*}
        In der Aufgabestellung ist die Zufallsvariable $X: \Omega \rightarrow \{0, 1\}$ definiert durch:
        \begin{align*}
            X(\{(k_1, k_2)\}) &:= \begin{cases}
                1\text{,}& \text{ falls } k_1 = 3 \lor k_2 = 3\text{,} \\
                0\text{,}& \text{ sonst, }
            \end{cases}  \\
            \intertext{oder anders ausgedr"uckt als}
            X(\{(k_1, k_2)\}) &:= \begin{cases}
                1\text{,}& \text{ falls } (k_1, k_2) \in M \lor (k_2, k_1) \in M\text{,} \\
                0\text{,}& \text{ sonst, }
            \end{cases}
        \end{align*}
        wobei $M$ den Ereignisraum bezeichnet, der alle Ziehungen umfasst, in
        denen eine von den beiden gezogenen Kugeln mit der Nummer 3 nummeriert
        ist. Dies l"asst sich wie folgt darstellen:
        \begin{equation*}
            M := \{3\} \times (\{1,...,10\} \setminus \{3\})\text{,} \quad |M| = 9 \quad \text{mit} \quad P(M) = \dfrac{|M|}{|\Omega|} = \dfrac{9}{45} = 0.2
        \end{equation*}
        Daraus ergibt sich:
        \begin{equation*}
            \begin{array}{rccl}
                &P^X(\{a\}) &=& \begin{cases}
                    P(M)\text{,}& \text{ falls } a = 1, \\
                    1 - P(M)\text{,}& \text{ falls } a = 0
                \end{cases} \\[20pt]
                p_x(a) := &P^X(\{a\}) &\implies& p_x(0) = 0.8, \quad p_x(1) = 0.2
            \end{array}
        \end{equation*}
\end{enumerate}

\pagebreak
\section{Aufgabe 30}

\begin{enumerate}[(a)]
    \item Es sei $X\sim\mathcal{B}(n,p)$. Zeigen Sie mit Hilfe des binomischen
        Satz (Math. Grundlagen), dass gilt $$\sum_{k=0}^nP(X=k)=1.$$
    \item Es sei wie in Teil (a) $X\sim\mathcal{B}(n,p)$ und $k\in\{0,...,n\}$
        fest. Für welchen Wert $p$ ist $P(X=k)$ maximal? (Hinweis: setzen
        Sie $f(p):=P(X=k)$ und berechnen
        Sie das Maximum von $f$).
\end{enumerate}
\textbf{L"osung.}
\begin{enumerate}[(a)]
    \item Binomischer Satz lautet: \textit{F"ur alle $a, b \in \mathbb{R}$ und alle $n \in \mathbb{N}_0$ gilt}
        \begin{equation}
            (a + b)^n = \sum_{\nu = 0}^n\binom{n}{\nu}a^{\nu}b^{n - \nu}
        \end{equation}
        Zu zeigen ist, dass
        \begin{equation*}
            \sum_{k=0}^nP(X=k)=1
        \end{equation*}
        Aus der Definition von der Binomialverteilung gilt:
        \begin{equation*}
            P(X = k) = \binom{n}{k}p^k(1 - p)^{n - k}
        \end{equation*}
        Daraus folgt:
        \begin{align*}
            \sum_{k=0}^nP(X=k) &= \binom{n}{0}p^0(1 - p)^{n} + \binom{n}{1}p^1(1 - p)^{n - 1} + ... + \binom{n}{k}p^k(1 - p)^{n - k} = \\[10pt]
                               &= \sum_{k = 0}^n\binom{n}{k}p^k(1 - p)^{n - k} = (p + (1 - p))^n = 1^n = 1 \quad \text{\underline{q.e.d.}}
        \end{align*}
\end{enumerate}
\end{document}
