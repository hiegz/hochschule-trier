\section{Aufgabe 1}

Stellen Sie sich vor, Sie sind Teil eines Forschungsteams, das eine Studie
"uber das Essverhalten in Großst"adten durchf"uhrt. Ihre Untersuchungseinheiten
sind Einwohner einer Großstadt. F"ur jede Untersuchungseinheit werden folgende
Merkmale erfasst.

\begin{enumerate}[-]
    \item Alter (in Jahren)
    \item Berufsgruppe (z.B Lehrer, Ingenieur, Student, usw.)
    \item Monatliches Einkommen (in Euro)
    \item Durchschnittliche Ausgaben f"ur Lebensmittel pro Woche (in Euro)
    \item Bevorzugte Art von Restaurant
    \item H"aufigkeit des Restaurantbesuchs pro Monat
    \item Bewertung der eigenen Kochf"ahigkeiten (auf eine Skala von 1 bis 5,
        wobei 1 sehr schlecht und 5 ausgezeichnet bedeutet)
\end{enumerate}

Klassifizieren Sie jedes dieser Merkmale als quantitativ oder qualitativ.
Unterteilen Sie weiter die qualitativen Merkmale in nominale und ordinale
Merkmale.

\bgroup
\def\arraystretch{1.5}
\begin{table}[h]
    \centering
    \begin{tabular}{p{0.33\linewidth}|p{0.33\linewidth}|p{0.33\linewidth}}
        \hfil\multirow{2}{*}{\centering quantitativ} & \multicolumn{2}{c}{qualitativ} \\
        \cline{2-3}
        & \hfil nominale & \hfil ordinale \\ \hline
        Alter & Berufsgruppe & Bevorzugte Art von Restaurant \\ \cline{1-1} \cline{3-3}
        Monatliches Einkommen & & Bewertung der Kochf"ahigkeiten \\ \cline{1-1}
        Durchschnittliche Ausgaben pro Woche & & \\ \cline{1-1}
        H"aufigkeit des Restaurantbesuchs pro Monat & &
    \end{tabular}
\end{table}
\egroup
