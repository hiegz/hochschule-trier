\documentclass[10pt, oneside]{article}
\usepackage[a4paper, total={5.5in, 9in}]{geometry}
\usepackage[ngerman]{babel}

\usepackage{blindtext}
\usepackage{titlesec}
\usepackage{amsmath}
\usepackage[hidelinks]{hyperref}
\usepackage{parskip}
\usepackage{graphicx}
\usepackage{longtable}
\usepackage[shortlabels]{enumitem}
\usepackage{multirow}
\usepackage{nccmath}
\usepackage{rotating}
\usepackage{makecell}
\usepackage{multicol}
\usepackage{capt-of}
\usepackage{csquotes}
\usepackage{amsfonts}
\usepackage{caption}

\captionsetup[table]{position=bottom}

\titleformat{\section}
    {\normalfont\Large\bfseries}{}{0pt}{}

\let\oldsection\section
\renewcommand{\section}{
  \renewcommand{\theequation}{\thesection.\arabic{equation}}
  \oldsection}
\let\oldsubsection\subsection
\renewcommand{\subsection}{
  \renewcommand{\theequation}{\thesubsection.\arabic{equation}}
  \oldsubsection}

\makeatletter
\renewcommand{\maketitle}{
    \bgroup
    \centering
    \par\LARGE\@title  \\[20pt]
    \par\large\@author \\[10pt]
    \par\large\@date
    \par
    \egroup
}
\makeatother


\title{Wahrscheinlichkeitstheorie und Statistik\\[10pt]\Large{WiSe 2024/25}\\[15pt]\Large{L{\"o}sungen zu den Aufgaben 45, 50 und 51}}
\author{\ }
\date{}

% - - - - - - - - - - - - - - - - - - - - - - - - - - - - - - - - - - - - - - %

\begin{document}

\maketitle
\vspace{25px}

\section{Aufgabe 45}

Es sei $\Omega = \{0, 1\}^3$ mit $P(\{\omega\}) = \dfrac{1}{8}$ $(\omega \in \Omega)$ unser W-Raum. Dann gilt
\begin{equation*}
    \Omega = \{(0, 0, 0), (0, 0, 1), (0, 1, 0), (0, 1, 1), (1, 0, 0), (1, 0, 1), (1, 1, 0), (1, 1, 1)\}
\end{equation*}
Daraus ergibt sich das Folgende:
\begin{enumerate}[i)]
    \item $P(X = 4 | Y = f_1) = \dfrac{6}{8} = \dfrac{3}{4}$
    \item $P(X = 4 | Y = f_2) = \dfrac{2}{8} = \dfrac{1}{4}$
    \item $P(X = 4 | Y = f_3) = \dfrac{3}{8}$
\end{enumerate}
Gem"a{\ss} der Bayes-Formel (6.49 im Skript) gilt
\begin{equation*}
    P(Y = f_l | X = 4) = \dfrac{P(X = 4 | Y = f_l)P(Y = f_l)}{\sum_{j = 1}^nP(X = 4 | Y = f_j)P(Y = f_j)}
\end{equation*}
Daraus folgt
\begin{enumerate}[i)]
    \item $\begin{aligned}[t]
            P(Y = f_1 | X = 4) &= \dfrac{P(X = 4 | Y = f_1)P(Y = f_1)}{\sum_{j = 1}^3P(X = 4 | Y = f_j)P(Y = f_j)}  \\[5pt]
                               &= \dfrac{6/8 \cdot 1/3}{6/8 \cdot 1/3 + 2/8 \cdot 1/3 + 3/8 \cdot 1/3} = \dfrac{6}{11}
        \end{aligned}$ \\[5pt]
    \item $\begin{aligned}[t]
            P(Y = f_2 | X = 4) &= \dfrac{P(X = 4 | Y = f_2)P(Y = f_2)}{\sum_{j = 1}^3P(X = 4 | Y = f_j)P(Y = f_j)}  \\[5pt]
                               &= \dfrac{2/8 \cdot 1/3}{6/8 \cdot 1/3 + 2/8 \cdot 1/3 + 3/8 \cdot 1/3} = \dfrac{2}{11}
        \end{aligned}$ \\[5pt]
    \item $\begin{aligned}[t]
            P(Y = f_3 | X = 4) &= \dfrac{P(X = 4 | Y = f_3)P(Y = f_3)}{\sum_{j = 1}^3P(X = 4 | Y = f_j)P(Y = f_j)}  \\[5pt]
                               &= \dfrac{3/8 \cdot 1/3}{6/8 \cdot 1/3 + 2/8 \cdot 1/3 + 3/8 \cdot 1/3} = \dfrac{3}{11}
        \end{aligned}$
\end{enumerate}

\section{Aufgabe 50}

\begin{enumerate}[(a)]
    \item $P(X \leq k) = \sum_{i = 0}^kp(1 - p)^i = p\sum_{i = 0}^k(1 - p)^i = p \cdot \dfrac{1 - (1 - p)^{k + 1}}{1 - (1 - p)} = 1 - (1 - p)^{k + 1}$
    \item $f(p) = p(1 - p)^2 = p^3 - 2p^2 + p\text{;  } f'(p) = 3p^2 - 4p + 1\text{;  } f''(p) = 6p - 4 \\[5pt]
            f'(p) = 0 \iff p = 1 \lor p = 1/3a \\[5pt]
            f''(1/3) < 0 \iff p = 1/3$
\end{enumerate}

\section{Aufgabe 51}

\begin{enumerate}[(a)]
    \item $P(1 \leq X \leq 5) = \sum_{k = 1}^5P(X = k)$
    \item $P(X + Y \leq 3) = P(X = 0 \land Y \leq 3) + P(X = 1 \land Y \leq 2) + P(X = 2 \land Y \leq 1)$
    \item $P(X > 4 \land Y < 2) = P(X > 4) \cdot P(Y < 2) = (1 - P(X \leq 4)) \cdot P(Y < 2)$
\end{enumerate}

\end{document}
