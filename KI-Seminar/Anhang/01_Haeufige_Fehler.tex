\chapter{Häufige Fehler in Seminar- und Abschlussarbeiten}

An dieser Stelle wird ein Überblick darüber gegeben, welche Fehler in
schriftlichen wissenschaftlichen Arbeiten häufig auftreten. Die
genannten Punkte können zur Reflektion und zur Prüfung in allen
Phasen des Verfassens einer Seminar- oder Abschlussarbeit
herangezogen werden.

\section{Inhalt}

\emph{Erfassung der Aufgabenstellung}
\begin{itemize}
    \item Das Thema wird nicht vollständig erfasst, einzelne Themenbestandteile werden nicht berücksichtigt, die Beziehungen zwischen den Themenbestandteilen werden nicht herausgearbeitet.
    \item Es wird ein Thema bearbeitet, welches sich nicht aus der Aufgabenstellung erschließt.
\end{itemize}

\emph{Aufbau der Arbeit und Inhalt}
\begin{itemize}
    \item Technische Gliederungsfehler
    \item Der Fluss der Gliederung ist nicht ersichtlich.
    \item Die Gliederungsteile bauen nicht aufeinander auf, sie stehen vielmehr isoliert nebeneinander.
    \item Die Überschriften sind nicht aussagekräftig.
    \item Eine Überschrift deckt sich vollständig mit dem Thema der Arbeit.
\end{itemize}

\emph{Anmerkungen zum Inhalt}
\begin{itemize}
    \item Problemstellung und Zielsetzung der Arbeit werden nur oberflächlich angerissen.
    \item Allgemeinplätze: „Globalisierung und zunehmende Dynamik ...“
\end{itemize}

\newpage
\emph{Eigenständigkeit der erbrachten Leistung}
\begin{itemize}
    \item Mangelnde Eigenständigkeit der erbrachten Leistung
    \item Zu viele direkte Zitate, mangelnde Reflektion in eigenen Worten
    \item Abbildungen und Tabellen werden 1:1 übernommen ohne kontextbezogene Eigenleistung und Darstellung.
\end{itemize}

