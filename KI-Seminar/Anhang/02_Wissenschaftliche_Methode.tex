\chapter{Wissenschaftliche Methode}

\emph{Begriffsbildung, Definition, Abgrenzung}
\begin{itemize}
    \item Begriffe werden nicht eingeführt, gar nicht oder aber erst später definiert.
    \item Falsche oder ungenaue Begriffsbildungen und -verwendungen
    \item Begriffe werden nicht überschneidungsfrei voneinander abgegrenzt.
    \item Gleiche Tatbestände werden mit unterschiedlichen Begriffen belegt.
    \item Mit der verwendeten Quelle wechselt die Notation, d. h. Begriffe werden mit unterschiedlichen Symbolen belegt.
\end{itemize}

\emph{Gedankenführung und Aufbau}
\begin{itemize}
    \item Die Kapitelüberschriften behandeln nicht oder nicht exakt das Thema der Arbeit.
    \item Sämtliche Überschriften einer tieferen Gliederungsebene behandeln nicht oder nicht exakt das Thema der Überschrift auf der nächst höheren Gliederungsebene.
    \item Es erfolgt die Untergliederung eines Absatzes, wobei nur ein Unterpunkt ausgewiesen wird: z. B. Absatz 3 soll untergliedert werden. Es wird nur ein Absatz 3.1 aufgeführt, 3.2, 3.3 usw. fehlen.
    \item Die Ausführungen innerhalb eines Absatzes betreffen nicht die Überschrift des Absatzes.
\item Der Fluss der Gedanken muss auf jeder Ebene deutlich sein; dies gilt sowohl für die gesamte Gliederung als auch für einzelne Absätze (auch die müssen strukturiert sein und einem „roten Faden“ folgen).
    \item Oft fehlen die Überleitungen zwischen einzelnen Kapiteln, der Gang der Arbeit wird nicht inhaltlich begründet (sowohl in der Einleitung als auch in den folgenden Ausführungen).
\end{itemize}

\newpage
\emph{Ergebnisbildung}
\begin{itemize}
    \item Angesichts einer in der Einleitung beschriebenen Problemstellung fehlt ein Diskussionsergebnis bzw. es ist unklar.
    \item Das Ergebnis hat wenig mit der in der Einleitung aufgeworfenen Problemstellung sowie den Untersuchungszielen zu tun.
    \item Das Ergebnis wird nicht eindeutig formuliert, der Kandidat „drückt“ sich um eine Stellungnahme.
\end{itemize}
