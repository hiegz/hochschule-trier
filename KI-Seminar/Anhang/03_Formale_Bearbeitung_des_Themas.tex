\chapter{Formale Bearbeitung des Themas}

\emph{Zitierweise und Literatur}
\begin{itemize}
    \item Besteht das Thema in der reinen Zusammenfassung der Literatur, sind die wichtigsten Quellen in ausreichender Anzahl zu verwenden.
    \item Sinngemäß entlehnte Literaturstellen werden gar nicht oder mit einem falschen Literaturverweis gekennzeichnet.
    \item Wörtlich wiedergegebene Textstellen werden nicht als solche gekennzeichnet.
    \item Die Literaturverweise (z. B. in Fußnoten) sind uneinheitlich, einmal mit ausgeschriebenem, das andere Mal ohne oder mit abgekürztem Vornamen. Das eine Mal wird ein Kurzverweis gewählt (z. B. Kloock (1990), S. 515), das andere Mal ein ausführlicher Verweis (z. B. Kloock, J.: Betriebswirtschaftliches Rechnungswesen, 2. Auflage, Köln 1997).
    \item Einige Autoren werden mit Titeln (Dr., Prof.) genannt, andere nicht. Generell: Keine Titelangabe.
    \item Aufschwemmen des Literaturverzeichnisses: Aufführung von Literaturquellen im Literaturverzeichnis, auf die nicht in der Arbeit verwiesen wurde.
    \item Nicht alphabetisch geordnetes Literaturverzeichnis.
    \item Zitieren zeigt u. a. wissenschaftliches Arbeiten auf. Insofern ist eine Auseinandersetzung mit der Literatur wichtig. Die Betonung liegt dabei auf Auseinandersetzung. Es bringt also nichts, 50 Quellen auf 20 Seiten zu zitieren, ohne dass ein inhaltlicher Diskurs mit diesen Quellen erfolgt. Nur Quellen allein (ohne Sinn und Verstand) machen nicht glücklich.
\end{itemize}

\emph{Orthographie und Interpunktion}
\begin{itemize}
    \item Unbefriedigende Orthographie und Interpunktion
\end{itemize}

\emph{Satzbau, Ausdruck, Stil}
\begin{itemize}
    \item Lange, verschachtelte und daher unverständliche Sätze
    \item Unvollständige Sätze
    \item Kein eigenständiger, homogener und flüssiger Stil, da die Sätze aus Originalquellen nur an wenigen Stellen mit dem Thesaurus in MS-Word bearbeitet wurden.
    \item Eitle Formulierungen wie „natürlich“, „schnell ersichtlich“, „selbstverständlich“ usw.
    \item Fehlende Struktur deutet sich in der Wortwahl an (daneben, ferner, weiterhin, ...).
    \item Wissenschaftliche Klarheit wird mit komplizierter Ausdrucksweise verwechselt.
    \item Übertriebener Gebrauch von Fremdworten.
\end{itemize}

\emph{Äußere Form}
\begin{itemize}
    \item
    \item Abbildungen weisen eine unproportionale Größe auf.
    \item Flatterrand und Blocksatz wechseln einander ab.
    \item Der Schriftgrad der Gliederung ist unterproportional zum Rest der Arbeit.
    \item Schlechte Druck- bzw. Kopierqualität.
\end{itemize}

\emph{Sonstiges}
\begin{itemize}
    \item Abbildungen weisen keine Textanbindungen auf und stehen unkommentiert im Raum.
    \item Der Anhang wird nicht als Anhang verwendet. Teile der Arbeit werden aufgenommen, die für das Textverständnis unbedingt erforderlich sind. Eine zu lange Arbeit wird nicht zweckentsprechend gekürzt, sondern – obwohl zusammengehörend – in Anhang und Hauptteil zerlegt.
    \item Nicht nachvollziehbare Absatzsetzung (z. B. wird nach jedem Satz ein Absatz gesetzt).
    \item Fehlende eigenständige Beispiele.
    \item Unvollständige oder fehlende Verzeichnisse (z. B. Abkürzungs-, Symbolverzeichnis).
\end{itemize}
