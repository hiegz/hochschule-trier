\chapter{Konzeptionelle Grundlagen}

In einem Grundlagenkapitel können Sie in Ihrer Arbeit die dem Hauptthema\slash der Forschungsfrage zugrundeliegenden und zur Auseinandersetzung benötigten Konzepte\slash Methoden\slash Modelle vorstellen und Begrifflichkeiten definieren, falls relevant.

\section{Allgemeine inhaltliche Aspekte des wissenschaftlichen Arbeitens}
Grundsätzlich können fünf Typen von \textbf{Forschungsfragen} ausgemacht werden (Überschneidungen sind möglich):
\begin{enumerate}
    \item \emph{Beschreibung eines Sachverhalts}, z.\,B.: \enquote{Entwicklung des Marktanteils japanischer Autohersteller auf dem US-amerikanischen Automarkt von 2001 bis 2010}.
    \item \emph{Erklärung eines Sachverhalts}, z.\,B.: \enquote{Gründe für den gestiegenen Marktanteil japanischer Autohersteller auf dem US-amerikanischen Automarkt}.
    \item \emph{Gestaltung eines Sachverhalts}, z.\,B.: \enquote{Prognose des Marktanteils japanischer Autohersteller auf dem US-amerikanischen Automarkt für das Jahr 2020}.
    \item \emph{Prognose eines Sachverhalts}, z.\,B.: \enquote{Gestaltung einer Marketingstrategie für General Motors zur Abwehr japanischer Autohersteller auf dem US-amerikanischen Automarkt}.
    \item \emph{Kritik\slash Bewertung eines Sachverhalts}, z.\,B.: \enquote{Negative Auswirkungen des gestiegenen Marktanteils japanischer Autohersteller auf dem US-amerikanischen Automarkt auf den US-Arbeitsmarkt}.\footcite[Vgl. zur Herangehensweise bspw.][S. 2f]{Bansch.2009}
\end{enumerate}
Grundsätzlich möglich:
\begin{itemize}
    \item \emph{theoriegeleitete Untersuchung}
    \item \emph{empirische Untersuchung}
    \item \emph{praktische Problemlösung}
    \item \emph{Bewertung vorhandener Lösungen}
\end{itemize}
Da Wissenschaft den Zweck hat, \enquote{wahre} Erkenntnisse zu gewinnen, unterliegt eine schriftliche wissenschaftliche Arbeit in Hinblick auf verschiedene Kriterien hohen Ansprüchen:
\begin{itemize}
    \item \emph{Objektivität} (sachliches, vorurteilsfreies, neutrales Arbeiten),
    \item \emph{Validität} (genaues Arbeiten auf Basis messbarer Verfahren),
    \item \emph{Verlässlichkeit} (methodisch angemessenes, exaktes Arbeiten),
    \item \emph{Fairness} (ehrliches, redliches, respektvolles Arbeiten),
    \item \emph{Überprüfbarkeit} (verifizierbares, nachvollziehbares Arbeiten) und
    \item \emph{Eigenständigkeit} (selbstständiges Arbeiten ohne fremde Hilfe).\footcite[Vgl. bspw.][S. 13-48]{Balzert.2017}
\end{itemize}

Grundsätzlich gilt es, in Ihrer Arbeit
\begin{itemize}
    \item auf der Grundlage wissenschaftlicher Erkenntnisse bzw. des Standes der wissenschaftlichen Diskussion in einem bestimmten Fachgebiet und
    \item in systematischer Auseinandersetzung mit den wissenschaftlichen Auffassungen von Anderen,
    \item sich eigene Gedanken zu machen,
    \item selbständig und ohne fremde Hilfe zu neuen Erkenntnissen zu gelangen,
    \item diese systematisch auf ihre objektive Geltung hin zu prüfen,
    \item die eigenen Gedanken und Auffassungen anderer in einer allgemein verständlichen Form darzustellen und
    \item damit die Praxis und den Erkenntnisprozess ein (kleines) Stück weiter zu bringen.\footnote{\cite[Vgl.][S. 9]{Sesink.2007} oder auch \cite[S. 71-85]{Kruse.2005}}
\end{itemize}

\section{Konkrete Elemente einer wissenschaftlichen Arbeit}
Die im Folgenden genannten Elemente einer wissenschaftlichen Arbeit können Ihnen ganz konkret als Gerüst\slash Basis bei der inhaltlichen Auseinandersetzung mit Ihrer Forschungsfrage dienen.
Klären Sie und berücksichtigen Sie beim Gang Ihrer Arbeit \textbf{immer}
\begin{enumerate}
    \item Zentrale Frage- bzw. Problemstellung (+ Ist-Situation)
    \item Ziele der Arbeit
    \item Rahmenbedingungen / Prämissen
    \item relevante Theorien / Modelle
    \item relevante praktische Alternativen
    \item Ergebnisbildung
\end{enumerate}

\section{Sechs Ebenen des wissenschaftlichen Erkenntnisprozesses}
Seien Sie sich darüber im Klaren, dass Sie bei der inhaltlichen Auseinandersetzung mit Ihrer Forschungsfrage sauber und stringent auf den folgenden Ebenen arbeiten sollen. Bis zu den Ebenen fünf und sechs dringen Sie nicht notwendigerweise in Ihrer Arbeit vor:

\begin{tabular}{@{\rownumber.\quad}p{0.2\textwidth}p{0.8\textwidth}}
    Definition            & Worüber sprechen wir?\\
    Klassifikation        & Was lässt sich unterscheiden? \newline
        Worüber sprechen wir nicht?\\
    Deskription           & Was läuft ab, was verändert sich?\\
    Theorie\slash Modell  & a. Woran liegen die Veränderungen? (Ursache $\rightarrow$ Wirkung)\newline
        b. Was ist für die Zukunft zu erwarten?\\
    Technologie         & Wie setzen wir es in der Praxis um?\\
    Philosophie         & a. Werturteile \newline
    b. technologische Ziel-Mittel-Aussagen (Wenn ..., dann ...)\\
\end{tabular}


\section{Formale Regeln Richtig Zitieren}
Zitieren ist gut und wichtig: Prüfen, nutzen und verarbeiten Sie Erkenntnisse Anderer zu Ihrem Forschungsgegenstand aus geeigneten Publikationen. Beachten Sie dabei, dass die aktuelle wissenschaftliche Diskussion nahezu ausschließlich in Fachzeitschriften stattfindet. Orientieren Sie sich an den ebenfalls vom Fachbereich Wirtschaft zur Verfügung gestellten Leitfäden für die wissenschaftliche Recherche und dem zugehörigen Video:
\href{https://www.hochschule-trier.de/hauptcampus/wirtschaft/studium/beratung-service/wissenschaftliches-arbeiten}{\nolinkurl{https://www.hochschule-trier.de/hauptcampus/wirtschaft/studium/beratung-service/wissenschaftliches-arbeiten}}.

Es folgt ein Beispiel für ein indirektes Zitat. Zitate - egal ob direkt oder indirekt - sind stets kenntlich zu machen. Auf direkte Zitate sollte weitestgehend verzichtet werden. Zeigen Sie, dass Sie in der Lage sind, Zusammenhänge in eigenen Worten wiederzugeben. Beispieltext mit indirektem Zitat und Beleg in Fußnote:

\begin{small}
\begin{minipage}[t]{0.45\textwidth}
\texttt{Eine mögliche Kennzahl zur Bestimmung der Vorteilhaftigkeit von Investitionen in Aktien stellt deren Rendite als Vergleichskennziffer dar. Unsichere zukünftige Renditen haben dabei die Eigenschaften von Zufallsvariablen.} \verb|\footcite[Vgl.][S. 314]{Schmidt.2006}|
\end{minipage}
\begin{minipage}{0.1\textwidth}
\centering
$\rightarrow$
\end{minipage}
\renewcommand{\thempfootnote}{\arabic{mpfootnote}}
\begin{minipage}[t]{0.45\textwidth}
Eine mögliche Kennzahl zur Bestimmung der Vorteilhaftigkeit von Investitionen in Aktien stellt deren Rendite als Vergleichskennziffer dar. Unsichere zukünftige Renditen haben dabei die Eigenschaften von Zufallsvariablen.\footcite[Vgl.][S. 314]{Schmidt.2006}
\end{minipage}
\end{small}

Es handelt sich bei den Belegen in dieser Vorlage um eine Kurzform in Fußnoten. Die Details zu den Quellen sind dann dem Literaturverzeichnis zu entnehmen. Alternative Belegeinbindungen sind möglich, bspw. im Fließtext nach der alphanumerischen Methode: \\\verb|\cite[vgl.][S. 314]{Schmidt.2006}| $\rightarrow$ \cite[vgl.][S. 314]{Schmidt.2006}. Bei größeren Werken empfiehlt sich die Seitenangabe, ansonsten kann diese weggelassen werden: \\ \verb|\cite[vgl.][]{Kruse.2005}| $\rightarrow$ \cite[vgl.][]{Kruse.2005}. Eine weitere mögliche Einbindung wäre bspw.: \verb|Nach \cite{Kruse.2005}| ist ... $\rightarrow$ Nach \cite{Kruse.2005} ist ... .

\section{Formale Regeln - Tabellen, Abbildungen, Formeln und Verweise}
Es folgt ein Beispiel für eine selbsterstellte \textbf{Tabelle}. Tabellen sind in einem Tabellenverzeichnis aufzunehmen und \text{immer zu kommentieren}. Dazu sei folgendes Beispiel gegeben:

\begin{lstlisting}[
    language=TeX,
    morekeywords={*,begin,end,caption,label,textbf,hline,colhead},
    caption={Beispielcode für Tabelle \ref{tab:portfolio}. In Zeile 3 werden die Anzahl der Spalten und deren Ausrichtungen festgelegt. Zur Wahl stehen für jede Spalte \texttt{l, c, r} für jeweils links, mittig und rechts. Damit in der Kopfzeile die Zahlen 1-4 dennoch mittig platziert werden, wird das Makro \texttt{colhead} verwendet. Die Kommentierung erfolgt mit dem Befehl \texttt{caption\{...\}}. In Zeile 1 wird mittels \texttt{[htb]} die Reihenfolge der gewünschten Positionierung der Tabelle festgelegt (\texttt{h}=here, \texttt{t}=top, \texttt{b}=bottom).}
]
\begin{table}[htb]
    \centering
    \begin{tabular}{lrrrr}
    \textbf{Umweltzustand} & \colhead{c}{1} & \colhead{c}{2}
    & \colhead{c}{3} & \colhead{c}{4} \\
    \hline
    \textbf{Eintrittswahrscheinlichkeit} & 0,2  & 0,3  & 0,1  & 0,4  \\
    \textbf{Erwartete Rendite Aktie 1}   & 0,11 & 0,16 & 0,12 & 0,05 \\
    \textbf{Erwartete Rendite Aktie 2}   & 0,02 & 0,22 & 0,15 & 0,18 \\
    \end{tabular}
    \caption{Anwendungsdaten zur Portfoliotheorie}
    \label{tab:portfolio}
\end{table}
\end{lstlisting}

Die Eintragung in das Tabellenverzeichnis erfolgt bei Befolgung des Schemas automatisch. Das Ergebnis ist in Tabelle \ref{tab:portfolio} zu sehen und kann im Text bspw. mit \newline\verb|Tabelle \ref{tab:portfolio}| (entsprechend des Eintrags in \texttt{label\{...\}}) referenziert werden.\footnote{Der Doppelpunkt in der Bezeichnung dient dabei lediglich der Befolgung der Namenskonvention nach dem Schema \texttt{<Typ>:<Bezeichnung>}.}

\begin{table}[htb]
    \centering
    \begin{tabular}{lrrrr}
    \textbf{Umweltzustand} & \colhead{c}{1} & \colhead{c}{2} & \colhead{c}{3} & \colhead{c}{4} \\
    \hline
    \textbf{Eintrittswahrscheinlichkeit} & 0,2          & 0,3           & 0,1           & 0,4           \\
    \textbf{Erwartete Rendite Aktie 1}   & 0,11         & 0,16          & 0,12          & 0,05          \\
    \textbf{Erwartete Rendite Aktie 2}   & 0,02         & 0,22          & 0,15          & 0,18          \\
    \end{tabular}
    \caption{Anwendungsdaten zur Portfoliotheorie}
    \label{tab:portfolio}
\end{table}

\newpage
Es folgt ein Beispiel für eine \textbf{Abbildung}:

\begin{lstlisting}[
    language=TeX,
    morekeywords={*,begin,end,caption,label,textbf,hline,colhead,includegraphics,footnotemark,footnotetext},
    caption={Beispielcode für Abbildung \ref{fig:rendite-risiko}. In diesem Fall wird für die \texttt{caption} ein optionaler Parameter mitverwendet. Dieser wird für die Eintragung in das Abbildungsverzeichnis verwendet. \emph{Hinweis}: Fußnoten sind in Captions nicht ohne Weiteres möglich, daher der Work-around mit \texttt{footnotemark} und \texttt{footnotetext}.}
]
\footnotetext{Eigene Darstellung.}
\end{lstlisting}
\footnotetext{Eigene Darstellung.}

Abbildungen sind i. d. R. selbst zu erstellen und werden nicht aus Vorlagen eingescannt und eingefügt. Zeigen Sie, dass Sie wie bei indirekten Zitaten in der Lage sind, wesentliche Inhalte herauszuarbeiten und in eigener Art darzustellen. Auch auf Abbildungen ist im Text stets Bezug zu nehmen, sie stehen ebenfalls \textbf{nie unkommentiert}. Wie bei Tabellen können Abbildungen im Text bspw. mit \verb|Abbildung \ref{fig:rendite-risiko}| verwendet werden.

Es folgt ein Beispiel für eine \textbf{Formel}. Formel werden durchnummeriert, verwendete verwendete Symbole werden erklärt und sind \textbf{durchgehend einheitlich}. Die gängige Darstellungsweise einer Formel im linken Bereich mit rechtsbündiger Nummerierung wurde hier wie folgt umgesetzt:

\begin{lstlisting}[
    language=TeX,
    morekeywords={*,begin,end,caption,label,textbf,hline,colhead,includegraphics,footnotemark,text,frac},
    caption={Beispielcode für Formel\ref{eq:rendite}}
]
\begin{equation}
    r_i=\frac{\text{D}_{i,t+1}
    + (\text{BK}_{i,t+1}-\text{DK}_{i,t})}{\text{BK}_{i,t}}
    \label{eq:rendite}
\end{equation}
\end{lstlisting}

\begin{equation}
    r_i=\frac{\text{D}_{i,t+1} +
    (\text{BK}_{i,t+1}-\text{DK}_{i,t})}{\text{BK}_{i,t}}
    \label{eq:rendite}
\end{equation}
Dabei ist $r_i$ die Rendite der Aktie $i$ mit $\text{BK}_{i,t}$ als Börsenkurs der Aktie $i$ zum Zeitpunkt $t$ und $\text{D}_{i,t}$ als Dividendenzahlung der Aktie $i$ zum Zeitpunkt $t$. Die Formel kann (und sollte!) im Text referenziert werden. Dies erfolgt wie bei Tabell- und Abbildungsverweise, bspw.: \\ \verb|In Formel \ref{eq:rendite} ...| $\rightarrow$ In Formel \ref{eq:rendite} ... .
