\chapter{Schlussbetrachtung}

Diese Seminararbeit hatte zum Ziel, den Minimax-Algorithmus und seine
praktischen Einsatzmöglichkeiten in der adversarialen Suche zu erläutern. Dazu
wurden die theoretischen Grundlagen, Optimierungsverfahren und typische
Probleme in der Praxis beschrieben. Es kann festgehalten werden, dass Minimax
eine solide Basis für die Entscheidungsfindung in deterministischen
Nullsummenspielen darstellt und in Verbindung mit Optimierungen wie
Alpha-Beta-Pruning in modernen Expertensystemen wie Stockfish erfolgreich
angewendet wird. Gleichzeitig zeigt sich, dass die Genauigkeit und Effizienz
der Suche durch Faktoren wie die Bewertungsfunktion, die Zugreihenfolge und die
Größe des Aktionsraums begrenzt sind. Für sehr komplexe Spiele oder Szenarien
mit unüberschaubarem Aktionsraum sind daher alternative Strategien notwendig.
