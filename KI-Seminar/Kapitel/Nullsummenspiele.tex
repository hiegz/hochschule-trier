\chapter{Nullsummenspiele}

Im Kontext der adversarialen Suche werden in der Fachliteratur überwiegend
sogenannte \textbf{Nullsummenspiele} betrachtet. Dabei handelt es sich um
Entscheidungsszenarien, in denen der Vorteil eines Akteurs gleichzeitig einen
gleich großen Nachteil für den Gegenspieler bedeutet. Das Grundprinzip besteht
darin, dass der Gesamtnutzen aller Akteure für jedes mögliche Ergebnis konstant
bleibt, was zu einem direkten Widerspruch der Interessen führt.  Die optimale
Strategie bei solchen Spielen ergibt sich dabei nicht aus der Maximierung eines
einzelnen Ergebnisses, sondern aus der Wahl einer Strategie, die den
bestmöglichen Ausgang unter der Annahme eines optimal handelnden Gegners
garantiert.

Zudem wird oft zwischen Spielen mit vollständiger und unvollständiger
Information unterschieden. Bei \textbf{Spielen mit vollständiger Information}
ist allen beteiligten Akteuren der aktuelle Zustand des Systems zu jedem
Zeitpunkt bekannt. Jeder Spieler kann somit seine Entscheidungen unter
Berücksichtigung des gesamten bisherigen Spielverlaufs treffen. So können die
Spieler auch die möglichen Reaktionen des Gegenspieler analysieren, was die
Wahl der Handlung unterstützt, die den bestmöglichen Ausgang für den eigenen
Spieler verspricht. \textbf{Spiele mit unvollständiger Information} hingegen
zeichnen sich dadurch aus, dass einzelne Spieler über private Informationen
verfügen, die für andere nicht zugänglich sind. Eine weitere Variante sind
teilweise beobachtbare Spiele, bei denen die Teilnehmer nur begrenzte
Informationen über den aktuellen Stand erhalten, was die strategische
Entscheidungsfindung zusätzlich erschwert.

Diese Arbeit konzentriert sich auf \textbf{Zwei-Akteur-Nullsummenspiele mit
\linebreak vollständiger Information}. Anschauliche Beispiele für diese Art von Spielen
sind Tic-Tac-Toe und Schach. Diese Einschränkung erlaubt es, die zentralen
Konzepte und Algorithmen der adversarialen Suche klar zu analysieren und deren
Funktionsweise nachvollziehbar darzustellen. Grundsätzlich sind viele der
vorgestellten Verfahren auch auf Szenarien mit mehr als zwei Akteuren
skalierbar, auf diese wird jedoch nur am Rande eingegangen. Andere Klassen von
Spielen, insbesondere solche mit unvollständiger Information, erfordern
grundlegend abweichende Ansätze und werden im Rahmen dieser Arbeit daher nicht
behandelt.
