\chapter{Nullsummenspiele}

Im Kontext der adversarialen Suche werden in der Fachliteratur überwiegend
sogenannte \textbf{Nullsummenspiele} betrachtet. Dabei handelt es sich um
Entscheidungsszenarien, in denen der Vorteil eines Akteurs gleichzeitig einen
gleich großen Nachteil für den Gegenspieler bedeutet.

Formal lässt sich ein Spiel durch die folgenden Komponenten beschreiben:

\begin{itemize}
    \item die Menge P der Spieler
    \item die Menge S aller möglichen Zustände
    \item die Menge E der Werte, die zur Bewertung von Zuständen verwendet
        werden
    \item die Funktion $A: S \to \mathcal{P}(S)$, wobei $A(s)$ die Menge aller erreichbaren
        Nachfolgezustände im Zustand $s \in S$ enthält
    \item der Terminal-Test $\text{\scshape is-terminal}: S \to
        \{\text{\ttfamily falsch}, \text{\ttfamily wahr}\}$, der {\ttfamily
        wahr} ist, wenn das Spiel zu Ende ist, und sonst {\ttfamily falsch}.
        Zustände, für die $\text{\scshape is-terminal}(s) = \text{\ttfamily wahr}$ gilt, werden als
        \textbf{terminale Zustände} bezeichnet
    \item die Funktion $\text{\scshape to-move}: S \to P$, die den Spieler
        zurückgibt, der im Zustand $s$ am Zug ist
    \item die Funktion $\text{\scshape eval}: S \to E$, die jedem Zustand
        eine Bewertung zuordnet
\end{itemize}

Weiterhin bezeichnet ein \textbf{Ply} die Aktion eines einzelnen Spielers, die
von einem bestimmten Zustand ausgeführt wird. Dieser Begriff wird verwendet, um
einzelne Spieleraktionen eindeutig zu beschreiben, da der Begriff \textbf{Zug}
in manchen Spielen die Aktionen beider Spieler zusammenfasst.

Werden von einem Ausgangszustand aus alle mögliche Aktione betrachtet,
entstehen für jeden Zustand mehrere mögliche Folgezustände. Dieser Prozess
wiederholt sich rekursiv, sodass sich eine Struktur ergibt, in der jeder
Zustand als Knoten und jeder Übergang als gerichtete Kante interpretiert werden
kann. Eine derartige Darstellung wird als \textbf{Spielbaum} bezeichnet.
