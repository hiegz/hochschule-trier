\chapter{Nullsummenspiele}

Im Kontext des Adversarial Search wird im Allgemeinen überwiegend von
sogenannten Nullsummenspielen gesprochen. Dabei handelt es sich um Spiele, in
denen ein Vorteil für einen Akteur gleichzeitig einen gleich großen Nachteil
für den anderen bedeutet.

In der Fachliteratur wird zudem häufig zwischen sogenannten Spielen mit
unvollständiger Information, bei denen ein Akteur private Informationen erhält,
die anderen nicht zugänglich sind, und Spielen mit vollständiger Information,
bei denen allen Akteuren der gesamte Zustand bekannt ist, unterschieden.
Außerdem existieren teilweise beobachtbare Spiele, in denen die Akteure nur
unvollständige Informationen über den aktuellen Zustand erhalten.

Diese Arbeit konzentriert sich auf perfekte Zwei-Akteur-Nullsummenspiele. Die
im Folgenden vorgestellten Algorithmen sollten prinzipiell auch auf Szenarien
mit mehr als zwei Akteuren skalierbar sein, jedoch werden in dieser Arbeit nur
kurz angesprochen. Andere Spielklassen mit unvollständiger Information
erfordern grundlegend andere Suchansätze und werden in dieser Arbeit daher
nicht behandelt.
