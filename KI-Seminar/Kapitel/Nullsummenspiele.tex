\chapter{Nullsummenspiele}

Im Kontext der adversarialen Suche werden in der Fachliteratur überwiegend
sogenannte \textbf{Nullsummenspiele} betrachtet. Dabei handelt es sich um
Entscheidungsszenarien, in denen der Vorteil eines Akteurs gleichzeitig einen
gleich großen Nachteil für den Gegenspieler bedeutet.

% Das Grundprinzip besteht darin, dass der Gesamtnutzen aller Akteure für jedes
% mögliche Ergebnis konstant bleibt, was zu einem direkten Widerspruch der
% Interessen führt.  Die optimale Strategie bei solchen Spielen ergibt sich dabei
% nicht aus der Maximierung eines einzelnen Ergebnisses, sondern aus der Wahl
% einer Strategie, die den bestmöglichen Ausgang unter der Annahme eines optimal
% handelnden Gegners garantiert.

\section{Informationsverfügbarkeit in Spielen}
\label{sec:information-availability-in-games}

Eine weitere Eigenschaft solcher Spiele betrifft die Frage, welche
Informationen den beteiligten Akteuren während des Spiels zur Verfügung stehen.
Es wird oft zwischen Spielen mit vollständiger und unvollständiger Information
unterschieden.

Bei \textbf{Spielen mit vollständiger Information} ist allen
beteiligten Akteuren der aktuelle Zustand des Systems zu jedem Zeitpunkt
bekannt. Jeder Spieler kann somit seine Entscheidungen unter Berücksichtigung
des gesamten bisherigen Spielverlaufs treffen. So können die Spieler auch die
möglichen Reaktionen des Gegenspieler analysieren, was die Wahl der Handlung
unterstützt, die den bestmöglichen Ausgang für den eigenen Spieler verspricht.

\textbf{Spiele mit unvollständiger Information} hingegen zeichnen sich dadurch
aus, dass einzelne Spieler über private Informationen verfügen, die für andere
nicht zugänglich sind. Eine weitere Variante sind \textbf{teilweise
beobachtbare Spiele}, bei denen die Teilnehmer nur begrenzte Informationen über
den aktuellen Stand erhalten, was die strategische Entscheidungsfindung
zusätzlich erschwert.

\section{Formale Definition eines Spiels}

Ein Spiel hat im Wesentlichen zwei grundlegende Komponenten: einen
\textbf{Zustand} (oder auch \textbf{Position}) und durch die Regeln des
jeweiligen Spiels definierte \textbf{Übergänge} (oder auch \textbf{Aktion})
zwischen Zuständen. Eine Position muss sämtliche Informationen enthalten, die
erforderlich sind, um alle möglichen Übergänge zu weiteren Positionen eindeutig
zu erzeugen.

Ein \textbf{Ply} bezeichnet die Aktion eines einzelnen Spielers, die von einem
bestimmten Zustand ausgeführt wird. Dieser Begriff wird verwendet, um einzelne
Spieleraktionen eindeutig zu beschreiben, da der Begriff \textbf{Zug} in
manchen Spielen die Aktionen beider Spieler zusammenfasst.

Ein Zustand, von dem keine weiteren Übergänge möglich sind, wird üblicherweise
als \textbf{terminaler Zustand} bezeichnet. Terminale Zustände markieren das
Ende eines Spiels und erlauben eine eindeutige Bewertung des Spielergebnisses,
etwa als Gewinn, Verlust oder Unentschieden aus der Perspektive eines
bestimmten Spielers.

\section{Spielbäume}

Werden von einem Ausgangszustand aus alle mögliche Aktione betrachtet,
entstehen für jeden Zustand mehrere mögliche Folgezustände. Dieser Prozess
wiederholt sich rekursiv, sodass sich eine Struktur ergibt, in der jeder
Zustand als Knoten und jeder Übergang als gerichtete Kante interpretiert werden
kann. Eine derartige Darstellung wird als \textbf{Spielbaum} bezeichnet.

\begin{figure}[h]
    \centering
    \includegraphics[width=1\textwidth]{Abbildungen/gametree.png}
    \caption{Spielbaum für Tic-Tac-Toe mit einer Tiefe von 2}
    \label{fig:gametree}
\end{figure}

In vielen Spielen ist es nicht möglich, den Spielbaum vollständig bis zu den
terminalen Zuständen zu erzeugen, da die Anzahl der Zustände nach einigen Plys
stark zunehmen kann. Aus diesem Grund werden häufig in der Praxis
\textbf{Spielbäume mit begrenzter Tiefe} konstruiert. Als anschauliches
Beispiel zeigt Abbildung~\ref{fig:gametree} einen 2-Ply-Spielbaum für
Tic-Tac-Toe.
