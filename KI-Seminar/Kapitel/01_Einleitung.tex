\chapter{Einleitung}

Das Verfassen einer wissenschaftlichen Arbeit stellt für Studierende
oft eine Herausforderung dar, insbesondere, wenn sie erstmalig mit
den formalen und inhaltlichen Anforderungen wissenschaftlichen
Schreibens konfrontiert werden. So sollen Regeln hinsichtlich der
\textbf{Form} von Seminararbeit, Bachelor- und Masterthesis
eingehalten werden, die \textbf{inhaltliche Auseinandersetzung} mit
einem Thema soll zudem wissenschaftlichen Standards genügen. Häufig
wird in beiden Bereichen gleich schlecht oder gut gearbeitet,
Qualität von Form und Inhalt korrelieren oft stark.

Diese Vorlage ist durchgehend automatisiert und entspricht in ihrer
Struktur und in den aufgenommenen Ebenen und Verzeichnissen
Ansprüchen an schriftliche wissenschaftliche Arbeiten.

Sie zielt insbesondere auf \textbf{formale Aspekte} ab. Andere
Formatierungen sind möglich. Die Verzeichnisse beinhalten beliebige
Beispieleinträge. Vorangehende Seiten, die nicht zum eigentlichen
Textteil gehören, wurden mit römischen Zahlen nummeriert. Andere
Nummerierungen sind zulässig.

Die Textelemente in dieser Vorlage liefern allgemeine formale und
inhaltliche Tipps und Tricks (ohne Anspruch auf Vollständigkeit!),
können ansonsten aber als Dummy-Text und Dummy-Überschriften
angesehen werden, die gelöscht und überschrieben werden sollen.

Beachten Sie auch die von Prof. Keilus zusammengetragenen häufigen
Fehler beim wissenschaftlichen Arbeiten, die Sie Anhang 1 entnehmen
können. Der Abgleich mit solchen Fehlern ermöglicht es Ihnen, zu
prüfen, ob Sie auf einem guten Weg sind.
