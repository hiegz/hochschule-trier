\chapter{Spielbaum}

Ein Spiel hat im Wesentlichen zwei grundlegende Komponenten: einen
\textbf{Zustand} (oder auch \textbf{Position}) und durch die Regeln des
jeweiligen Spiels definierte \textbf{Übergänge} (oder auch \textbf{Aktion})
zwischen Zuständen. Eine Position muss sämtliche Informationen enthalten, die
erforderlich sind, um alle möglichen Übergänge zu weiteren Positionen eindeutig
zu erzeugen. Ein \textbf{Ply} bezeichnet dabei die Aktion eines einzelnen
Spielers, die von einem bestimmten Zustand ausgeführt wird, im Gegensatz zum
Begriff \textbf{Zug}, der in manchen Spielen die Aktionen beider Spieler
zusammenfasst. Ein Zustand, von dem keine weiteren Übergänge möglich sind, wird
üblicherweise als \textbf{terminaler Zustand} bezeichnet.

In vielen Spielen ist es jedoch nicht möglich, den Spielbaum vollständig bis zu
den terminalen Zuständen zu erzeugen, da die Anzahl der Zustände nach einigen
Plys stark zunehmen kann. Aus diesem Grund werden häufig in der Praxis
\textbf{Spielbäume mit begrenzter Tiefe} konstruiert. Als anschauliches
Beispiel zeigt Abbildung~\ref{fig:gametree} einen 2-Ply-Spielbaum für
Tic-Tac-Toe.

\begin{figure}[h]
    \centering
    \includegraphics[width=1\textwidth]{Abbildungen/gametree.png}
    \caption{Spielbaum für Tic-Tac-Toe mit einer Tiefe von 2}
    \label{fig:gametree}
\end{figure}
