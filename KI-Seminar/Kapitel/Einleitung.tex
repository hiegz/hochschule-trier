\chapter{Einleitung}

In vielen realen Entscheidungssituationen hängt der Erfolg einer Handlung nicht
allein von den eigenen Entscheidungen ab, sondern wesentlich vom Verhalten
anderer Akteure. Besonders in konkurrierenden Umgebungen ist es häufig
erforderlich, Strategien zu entwickeln, die die Interessen der übrigen
Teilnehmer berücksichtigen, welche sich meist erheblich von den eigenen
unterscheiden und mitunter sogar im Widerspruch dazu stehen. Das Ergebnis
solcher Handlungen ist daher nicht eindeutig vorhersehbar, sondern entsteht aus
dem Zusammenspiel aller beteiligten Akteure.

Im Bereich der KI-Forschung beschäftigen sich Wissenschaftler schon seit langem
mit der formalen Modellierung und algorithmischen Lösung solcher
Entscheidungsprobleme. Ein zentraler Aspekt besteht dabei darin,
Handlungsstrategien zu entwickeln, die auch unter ungünstigen Bedingungen zu
möglichst guten Ergebnissen führen. Besonders problematisch sind dabei
Situationen, in denen ein Akteur einem oder mehreren rational agierenden
Gegenspielern gegenübersteht, die aktiv versuchen, seinen Erfolg zu verhindern.
Eine Handlung ist daher nur dann sinnvoll, wenn sie auch unter Berücksichtigung
des bestmöglichen gegnerischen Verhaltens Bestand hat. Diese Art von Problemen
wird unter dem Begriff \textbf{Adversarial Search} zusammengefasst.

Die vorliegende Arbeit befasst sich mit einem Teilbereich dieser Forschung, der
sich mit Entscheidungsproblemen in konkurrierenden Szenarien befasst,
insbesondere in strategischen Spielen wie Tic-Tac-Toe oder Schach. Dabei liegt
der Schwerpunkt auf der formalen Modellierung solcher Situationen sowie auf
Ansätzen, die eine rationale und systematische Entscheidungsfindung
ermöglichen. Es werden die grundlegenden Problemen diskutiert, mögliche
Lösungsstrategien verglichen und deren Vor- und Nachteile besprochen. Ziel ist
es, ein umfassendes Verständnis dafür zu vermitteln, wie Entscheidungen in
komplexen und interaktiven Umgebungen getroffen werden können.
