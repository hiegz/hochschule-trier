\chapter{Einleitung}

Adversarial Search gehört zu den frühesten Entwicklungen der KI-Forschung in
einer Zeit, in der theoretische Untersuchungen noch nicht über die notwendigen
Rechenressourcen verfügten und somit der Ära moderner, universell einsetzbarer
Computer vorausgingen, obwohl viele der Konzepte bis heute Anwendung finden.
Dabei beschäftigt sich Adversarial Search mit kompetitiven Umgebungen, in denen
zwei oder mehr Spieler gegensätzliche Ziele verfolgen. Schwerpunktmäßig werden
Spiele betrachtet, deren Zustände einfach zu repräsentieren sind und deren
Aktionsmöglichkeiten für einen einzelnen Spieler begrenzt sind. Beispiele
hierfür sind Spiele wie Schach, Go oder Poker.

Diese Arbeit soll als Einführung in das Thema Adversarial Search dienen und
sowohl die grundlegenden Konzepte als auch die zentralen Algorithmen -
insbesondere Minimax und Alpha-Beta-Pruning - erläutern. Dabei wird gezeigt,
wie rationale Entscheidungsfindung in diesen Spielumgebungen modelliert
werden kann, welche Herausforderungen bei der Suche in großen Zustandsräumen
entstehen und wie Optimierungen die Effizienz solcher Verfahren erhöhen.

Aufgrund des begrenzten Umfangs dieser Arbeit werden in erster Linie perfekte
Nullsummenspiele mit zwei Spielern behandelt. Andere Aspekte des Adversarial
Search, wie beispielsweise Spiele mit mehr als zwei Spielern oder stochastische
Spiele, bei denen die verfügbaren Aktionen eines Spielers von
Wahrscheinlichkeiten abhängen, werden nur kurz angesprochen.

Am Ende soll der Leser ein fundiertes Verständnis der Konzepte des Adversarial
Search erlangt haben und über eine solide Grundlage für weiterführende Lektüre
und vertiefende Beschäftigung mit diesem Forschungsfeld verfügen.

% Das Verfassen einer wissenschaftlichen Arbeit stellt für Studierende
% oft eine Herausforderung dar, insbesondere, wenn sie erstmalig mit
% den formalen und inhaltlichen Anforderungen wissenschaftlichen
% Schreibens konfrontiert werden. So sollen Regeln hinsichtlich der
% \textbf{Form} von Seminararbeit, Bachelor- und Masterthesis
% eingehalten werden, die \textbf{inhaltliche Auseinandersetzung} mit
% einem Thema soll zudem wissenschaftlichen Standards genügen. Häufig
% wird in beiden Bereichen gleich schlecht oder gut gearbeitet,
% Qualität von Form und Inhalt korrelieren oft stark.
%
% Diese Vorlage ist durchgehend automatisiert und entspricht in ihrer
% Struktur und in den aufgenommenen Ebenen und Verzeichnissen
% Ansprüchen an schriftliche wissenschaftliche Arbeiten.
%
% Sie zielt insbesondere auf \textbf{formale Aspekte} ab. Andere
% Formatierungen sind möglich. Die Verzeichnisse beinhalten beliebige
% Beispieleinträge. Vorangehende Seiten, die nicht zum eigentlichen
% Textteil gehören, wurden mit römischen Zahlen nummeriert. Andere
% Nummerierungen sind zulässig.
%
% Die Textelemente in dieser Vorlage liefern allgemeine formale und
% inhaltliche Tipps und Tricks (ohne Anspruch auf Vollständigkeit!),
% können ansonsten aber als Dummy-Text und Dummy-Überschriften
% angesehen werden, die gelöscht und überschrieben werden sollen.
%
% Beachten Sie auch die von Prof. Keilus zusammengetragenen häufigen
% Fehler beim wissenschaftlichen Arbeiten, die Sie Anhang 1 entnehmen
% können. Der Abgleich mit solchen Fehlern ermöglicht es Ihnen, zu
% prüfen, ob Sie auf einem guten Weg sind.
