\chapter{Einschränkungen}

In der Praxis werden die im vorherigen Kapitel eingeführten Konzepte und
Algorithmen auf moderner Hardware bereits weitgehend in Spielprogrammen auf
Expertenniveau wie Stockfish eingesetzt – Programme, die so stark sind, dass
kein menschlicher Spieler sie besiegen kann.

Dies bedeutet jedoch nicht, dass diese Algorithmen universell richtige oder
unumstrittene Entscheidungen liefern. Programme wie Stockfish liefern
Ergebnisse, die für eine gewählte Suchtiefe so genau wie möglich sind. Aufgrund
des exponentiellen Wachstums von Spielbäumen hängt die Effektivität dieser
Algorithmen stark von der spezifischen Struktur der Bäume und der Reihenfolge
ab, in der Züge untersucht werden
\cite{a-comparison-of-minimax-tree-search-algorithms}. Außerdem haben die in
Abschnitt 3.5 diskutierten Probleme, wie der Horizon-Effekt, keine formal
anerkannte, universell anwendbare Lösung für jedes Spiel. Folglich ist die
Genauigkeit der Suche grundsätzlich durch die Unsicherheit bei der Bewertung
nicht-terminaler Zustände begrenzt. Eine größere Suchtiefe beim
Minimax-Verfahren verbessert die Entscheidungsgenauigkeit nicht zwangsläufig
und kann sogar Fehler verstärken, wenn die Bewertungsfunktion häufig
irreführende Werte liefert. \cite{the-nature-of-minimax}

Eine weitere wesentliche Einschränkung adversarialer Suche liegt in den
Entwicklungs- und Rechenkosten. In Spielen wie Schach, in denen die Menge der
zulässigen Züge aus einem Zustand begrenzt ist, ist eine vollständige oder
nahezu vollständige Suche noch machbar. In anderen adversarialen Szenarien, in
denen die Menge möglicher Aktionen extrem groß oder praktisch unbegrenzt sein
kann, werden traditionelle Suchverfahren jedoch schnell unpraktikabel, sodass
alternative Ansätze erforderlich sein können.
