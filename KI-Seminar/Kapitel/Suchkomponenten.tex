\chapter{Suchkomponenten}

\section{Spielzustandsrepräsentation}

Die Repräsentation des Spielzustands bildet die Grundlage jeder algorithmischen
Suche. Ein Spielzustand muss alle Informationen enthalten, die notwendig sind,
um den weiteren Spielverlauf eindeutig zu bestimmen. Die Art der Repräsentation
kann die Verarbeitung von Zuständen und Zustandsübergängen erheblich
erleichtern, was wiederum Aspekte wie die Zugerzeugung beeinflussen kann.

Während kein allgemeiner Standard für alle Spiele existiert, haben sich in
bestimmten Bereichen, etwa im Schach, Bitboards etabliert. Dabei kodiert ein
64-Bit-Wort die Positionen aller Figuren eines bestimmten Typs. Diese
Darstellung erlaubt parallele Ausführung von Operationen auf allen 64 Feldern.
Neben der schnelleren Verarbeitung erleichtert diese Darstellung auch die
Speicherung und den Zugriff auf Zustände in bestimmten Datenstrukturen, was
beispielsweise für Lookups oder andere Operationen innerhalb der
Suchalgorithmen von Vorteil ist. Die Abbildung~\ref{fig:bitboard} soll das
Prinzip der Bitboard-Darstellung veranschaulichen.

\begin{figure}[h]
    \centering
    \includegraphics[width=0.85\textwidth]{Abbildungen/bitboard.png}
    \caption{Bitboard-Darstellung}
    \label{fig:bitboard}
\end{figure}

Die Wahl der Spielzustandsrepräsentation sollte sorgfältig auf den jeweiligen
Anwendungsfall abgestimmt werden. Bestimmte Repräsentationen, wie
etwa Bitboards, können die Berechnungen zwar beschleunigen, erhöhen dabei
jedoch den Speicherbedarf, da sie Informationen über sowohl besetzte als auch
leere Felder speichern. So belegt ein Bitboard beispielsweise 8 Bytes, nur um
die Position einer einzelnen Figur (z.B. des Königs) darzustellen, die
theoretisch in einem Byte gespeichert werden könnte. Daher ist es möglich, für
ein und dasselbe Spielmodell mehrere Repräsentationen zu verwenden, um
unterschiedliche Aspekte der Suche optimal zu unterstützen.

\section{Zugerzeugung}

Aus einem gegebenen Spielzustand lassen sich weitere Zustände ableiten, indem
alle möglichen Übergänge oder Züge bestimmt werden, die von diesem
Ausgangspunkt aus ausgeführt werden können. Dieser Prozess wird als
Zugerzeugung bezeichnet.

Ein zentrales Ziel der Zugerzeugung ist es, korrekt (nur regelkonforme Züge)
und vollständig (alle möglichen Züge) zu sein. Gleichzeitig muss sie effizient
arbeiten, da sie während der Suche tausend- bis millionenfach ausgeführt wird.
Unnötig generierte Züge oder aufwendige Regelprüfungen wirken sich direkt
negativ auf die Suchleistung aus.

In vielen Spielen wird zwischen Pseudo-Legalität und tatsächlicher Legalität
unterschieden: Zunächst werden Kandidatenzüge generiert, die anschließend durch
zusätzliche Regelprüfungen gefiltert werden. Diese Zweistufigkeit ist ein
gängiger Kompromiss zwischen Geschwindigkeit und Korrektheit.

\section{Bewertungsfunktion}

Vielleicht der wichtigste Bestandteil eines Schachprogramms ist die
Bewertungsfunktion.

Theoretisch wäre es möglich, den besten Zug zu bestimmen, indem man
ausschließlich Züge auswählt, die Teil einer ‚gewinnbringenden‘ Zugfolge sind –
also einer Sequenz, die zu einem Sieg führt. Praktisch ist dies jedoch nahezu
unmöglich, da die Anzahl möglicher Spielverläufe derart groß ist, dass kein
moderner Computer in der Lage wäre, alle Szenarien zu durchsuchen, selbst wenn
beide Spieler optimal agieren würden.

Aus diesem Grund wird eine Bewertungsfunktion benötigt, die den Wert von
Positionen oder Spielzuständen abschätzt, die nicht vollständig durchsucht
werden können. Im einfachsten Fall berücksichtigt die Bewertungsfunktion im
Schach lediglich die Materialdifferenz. Für ein stärkeres Spiel ist es jedoch
erforderlich, auch zahlreiche positionelle Faktoren zu berücksichtigen, wie
etwa die Struktur der Bauern.

Hier zeigen sich die Unterschiede zwischen guten und schlechten Suchprogrammen.
Außerdem ermöglicht die Bewertungsfunktion nicht nur eine präzisere Auswahl von
Zügen, sondern definiert auch den Charakter, mit dem das Programm das Spiel
spielt.

\section{Spielbäume}

Um solche Umgebungen systematisch analysieren zu können, wird der
Entscheidungsraum in Form eines Spielbaums modelliert, der aus Spielzuständen
und Übergängen besteht und alle möglichen Zugfolgen bis zu den terminalen
Zuständen abbildet, also jenen Zuständen, in denen das Spiel endet und keine
weiteren Übergänge mehr möglich sind. In komplexen Spielen wie Schach wächst
dieser Baum jedoch bereits nach wenigen Zügen beider Spieler derart stark an,
dass kein moderner Computer in vertretbarer Zeit sämtliche Knoten vollständig
durchsuchen könnte. Aus diesem Grund werden Spielbäume in der Praxis nur bis zu
einer festgelegten Tiefe betrachtet.

In Abbildung~\ref{fig:gametree} ist ein Beispiel solches Spielbaums für Tic-Tac-Toe mit einer
Tiefe von 2 dargestellt.

\begin{figure}[h]
    \centering
    \includegraphics[width=1\textwidth]{Abbildungen/gametree.png}
    \caption{Spielbaum für Tic-Tac-Toe mit einer Tiefe von 2}
    \label{fig:gametree}
\end{figure}
