\chapter{Minimax}

\section{Definition der Umgebung}

Betrachten wir ein Nullsummenspiel mit zwei Akteuren $P = \{\text{\scshape
Max}, \text{\scshape Min}\}$, deren Ziele einander widersprechen.

Jedem Spielzustand $s \in S$ entspricht ein Spielwert $\text{\scshape eval}(s)
\in [-1,1]$. Dabei erhalten terminale Zustände den Wert

\begin{itemize}
    \item $\text{\scshape eval}(s) = 1$, wenn \textsc{Max} gewinnt,
    \item $\text{\scshape eval}(s) = -1$, wenn \textsc{Min} gewinnt,
    \item oder $\textsc{\scshape eval}(s) = 0$ bei einem Unentschieden.
\end{itemize}

Für nicht-terminale Zustände liegt der Wert im offenen Intervall $(-1,1)$,
wobei positive Werte \textsc{Max} und negative Werte \textsc{Min} begünstigen.

\section{Optimale Entscheidungsfindung}

Aus einem gegebenen Spielbaum lässt sich eine optimale Strategie ableiten,
indem die sogenannten \textbf{Minimax-Werte} der Knoten berechnet werden.

Blattknoten, die entweder terminale Zustände oder die Endpunkte eines auf eine
bestimmte Tiefe begrenzten Spielbaums darstellen, übernehmen dabei einfach den
Spielwert, den $\text{\scshape eval}(s)$ liefert. In nicht-terminalen Zuständen
wählt \textsc{Max} den größten und \textsc{Min} den kleinsten Wert unter den
Nachfolgern, wodurch die Werte der Blattknoten bis zur Wurzel des Baums
zurückpropagiert werden. Formal lässt sich dies wie folgt ausdrücken:
\begin{equation}
    \text{\textsc{minimax}(s)} =
    \begin{cases}
        \text{\textsc{eval}}(s) & \text{falls} \; s \; \text{ein Blattknoten ist}, \\
        \text{max}_{s' \in A(s)} \text{\textsc{minimax}}(s') & \text{falls \textsc{to-move}}(s) = \text{\textsc{Max}}, \\
        \text{min}_{s' \in A(s)} \text{\textsc{minimax}}(s') & \text{falls \textsc{to-move}}(s) = \text{\textsc{Min}}.
    \end{cases}
    \label{eq:minimax}
\end{equation}

Der beste Zug ist stets derjenige Nachfolger, der für den Spieler den optimalen
Minimax-Wert liefert.

\begin{figure}[h]
    \centering

    \begin{subfigure}[t]{0.49\textwidth}
        \includegraphics[width=\linewidth]{Abbildungen/minimax-a.png}
    \end{subfigure}
    \hfill
    \begin{subfigure}[t]{0.49\textwidth}
        \includegraphics[width=\linewidth]{Abbildungen/minimax-b.png}
    \end{subfigure}

    \begin{subfigure}[t]{0.49\textwidth}
        \includegraphics[width=\linewidth]{Abbildungen/minimax-c.png}
    \end{subfigure}
    \hfill
    \begin{subfigure}[t]{0.49\textwidth}
        \includegraphics[width=\linewidth]{Abbildungen/minimax-d.png}
    \end{subfigure}

    \begin{subfigure}[t]{0.49\textwidth}
        \includegraphics[width=\linewidth]{Abbildungen/minimax-e.png}
    \end{subfigure}
    \hfill
    \begin{subfigure}[t]{0.49\textwidth}
        \includegraphics[width=\linewidth]{Abbildungen/minimax-f.png}
    \end{subfigure}

    \caption{Minimax-Algorithmus auf einem 2-Ply-Spielbaum}
    \label{fig:minimax-on-a-2-ply-game-tree}
\end{figure}

Die Abbildung oben illustriert anschaulich den Minimax-Algorithmus auf einem
2-Ply-Spielbaum mit dem \textsc{Max}-Knoten an der Wurzel. Die optimalste
Zugfolge verläuft dabei über die Knoten $\text{D} \rightarrow \text{A}
\rightarrow 0,7$.
