\chapter{Minimax}

\section{Definition der Umgebung}

Betrachten wir ein Nullsummenspiel mit zwei Akteuren $P = \{\text{\scshape
Max}, \text{\scshape Min}\}$, deren Ziele einander widersprechen.

Jedem Spielzustand $s \in S$ entspricht ein Spielwert $\text{\scshape eval}(s)
\in [-1,1]$. Dabei erhalten terminale Zustände den Wert

\begin{itemize}
    \item $\text{\scshape eval}(s) = 1$, wenn \textsc{Max} gewinnt,
    \item $\text{\scshape eval}(s) = -1$, wenn \textsc{Min} gewinnt,
    \item oder $\textsc{\scshape eval}(s) = 0$ bei einem Unentschieden.
\end{itemize}

Für nicht-terminale Zustände liegt der Wert im offenen Intervall $(-1,1)$,
wobei positive Werte \textsc{Max} und negative Werte \textsc{Min} begünstigen.

\section{Optimale Entscheidungsfindung}

Aus einem gegebenen Spielbaum lässt sich eine optimale Strategie ableiten,
indem die sogenannten \textbf{Minimax-Werte} der Knoten berechnet werden.

Blattknoten, die entweder terminale Zustände oder die Endpunkte eines auf eine
bestimmte Tiefe begrenzten Spielbaums darstellen, übernehmen dabei einfach den
Spielwert, den $\text{\scshape eval}(s)$ liefert. In nicht-terminalen Zuständen
wählt \textsc{Max} den größten und \textsc{Min} den kleinsten Wert unter den
Nachfolgern, wodurch die Werte der Blattknoten bis zur Wurzel des Baums
zurückpropagiert werden. Formal lässt sich dies wie folgt ausdrücken:
\begin{equation}
    \text{\textsc{minimax}(s)} =
    \begin{cases}
        \text{\textsc{eval}}(s) & \text{falls} \; s \; \text{ein Blattknoten ist}, \\
        \text{max}_{s' \in A(s)} \text{\textsc{minimax}}(s') & \text{falls \textsc{to-move}}(s) = \text{\textsc{Max}}, \\
        \text{min}_{s' \in A(s)} \text{\textsc{minimax}}(s') & \text{falls \textsc{to-move}}(s) = \text{\textsc{Min}}.
    \end{cases}
    \label{eq:minimax}
\end{equation}

Der beste Zug ist stets derjenige Nachfolger, der für den Spieler den optimalen
Minimax-Wert liefert.

Der folgende 2-Ply-Spielbaum dient als Beispiel.

\begin{figure}[h]
    \centering
    \includegraphics[width=1\textwidth]{Abbildungen/empty-minimax-game-tree.png}
    \caption{Ein 2-Ply-Spielbaum mit Zustandsbewertungen}
    % \label{fig:empty-minimax-game-tree}
\end{figure}

Die Symbole $\diamond$ und $\circ$ stehen für \textsc{Max}- bzw.
\textsc{Min}-Knoten, an denen der jeweilige Spieler am Zug ist. Blattknoten,
dargestellt durch $\square$, zeigen ihre Spielwerte.

Wendet man Definition~\ref{eq:minimax} an, ergeben sich folgende Minimax-Werte:

\begin{itemize}
    \item $\text{\scshape minimax}(\text{A}) = \text{min}\{0.9, 0.8, 0.7\} = 0.7$
    \item $\text{\scshape minimax}(\text{B}) = \text{min}\{0.0, 1.0, 0.7\} = 0.0$
    \item $\text{\scshape minimax}(\text{C}) = \text{min}\{-0.9, 0.3, 0.0\} = -0.9$
    \item $\text{\scshape minimax}(\text{D}) = \text{max}\{0.7, 0.0, -0.9\} = 0.7$
\end{itemize}

Der so berechnete Spielbaum sieht dann wie in
Abbildung~\ref{fig:filled-minimax-game-tree} aus.

Die optimalste Zugfolge für beide Spieler verläuft dabei über die Knoten
$\text{D} \rightarrow \text{A} \rightarrow 0.7$.

\begin{figure}[h]
    \centering
    \includegraphics[width=1\textwidth]{Abbildungen/filled-minimax-game-tree.png}
    \caption{Ein 2-Ply-Spielbaum mit Zustandsbewertungen und mit den berechneten Minimax-Werten}
    \label{fig:filled-minimax-game-tree}
\end{figure}
