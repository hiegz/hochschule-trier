\chapter{Minimax}

\section{Definition der Umgebung}

Betrachten wir ein Nullsummenspiel mit zwei Akteuren $P = \{\text{\scshape
Max}, \text{\scshape Min}\}$, deren Ziele einander widersprechen.

Jedem Spielzustand $s \in S$ entspricht ein Spielwert $\text{\scshape eval}(s)
\in [-1,1]$. Dabei erhalten terminale Zustände den Wert

\begin{itemize}
    \item $\text{\scshape eval}(s) = 1$, wenn \textsc{Max} gewinnt,
    \item $\text{\scshape eval}(s) = -1$, wenn \textsc{Min} gewinnt,
    \item oder $\textsc{\scshape eval}(s) = 0$ bei einem Unentschieden.
\end{itemize}

Für nicht-terminale Zustände liegt der Wert im offenen Intervall $(-1,1)$,
wobei positive Werte \textsc{Max} und negative Werte \textsc{Min} begünstigen.

\section{Optimale Entscheidungsfindung}

Aus einem gegebenen Spielbaum lässt sich eine optimale Strategie ableiten,
indem die sogenannten \textbf{Minimax-Werte} der Knoten berechnet werden.

Blattknoten, die entweder terminale Zustände oder die Endpunkte eines auf eine
bestimmte Tiefe begrenzten Spielbaums darstellen, übernehmen dabei einfach den
Spielwert, den $\text{\scshape eval}(s)$ liefert. In nicht-terminalen Zuständen
wählt \textsc{Max} den größten und \textsc{Min} den kleinsten Wert unter den
Nachfolgern, wodurch die Werte der Blattknoten bis zur Wurzel des Baums
zurückpropagiert werden. Formal lässt sich dies wie folgt ausdrücken:
\begin{equation}
    \text{\textsc{minimax}(s)} =
    \begin{cases}
        \text{\textsc{eval}}(s) & \text{falls} \; s \; \text{ein Blattknoten ist}, \\
        \text{max}_{s' \in A(s)} \text{\textsc{minimax}}(s') & \text{falls \textsc{to-move}}(s) = \text{\textsc{Max}}, \\
        \text{min}_{s' \in A(s)} \text{\textsc{minimax}}(s') & \text{falls \textsc{to-move}}(s) = \text{\textsc{Min}}.
    \end{cases}
    \label{eq:minimax}
\end{equation}

Der beste Zug ist stets derjenige Nachfolger, der für den Spieler den optimalen
Minimax-Wert liefert.

\begin{figure}[h]
    \centering

    \begin{subfigure}[t]{0.49\textwidth}
        \includegraphics[width=\linewidth]{Abbildungen/minimax-a.png}
    \end{subfigure}
    \hfill
    \begin{subfigure}[t]{0.49\textwidth}
        \includegraphics[width=\linewidth]{Abbildungen/minimax-b.png}
    \end{subfigure}

    \begin{subfigure}[t]{0.49\textwidth}
        \includegraphics[width=\linewidth]{Abbildungen/minimax-c.png}
    \end{subfigure}
    \hfill
    \begin{subfigure}[t]{0.49\textwidth}
        \includegraphics[width=\linewidth]{Abbildungen/minimax-d.png}
    \end{subfigure}

    \begin{subfigure}[t]{0.49\textwidth}
        \includegraphics[width=\linewidth]{Abbildungen/minimax-e.png}
    \end{subfigure}
    \hfill
    \begin{subfigure}[t]{0.49\textwidth}
        \includegraphics[width=\linewidth]{Abbildungen/minimax-f.png}
    \end{subfigure}

    \caption{Minimax-Algorithmus auf einem 2-Ply-Spielbaum}
    \label{fig:minimax-on-a-2-ply-game-tree}
\end{figure}

Die Abbildung oben illustriert anschaulich den Minimax-Algorithmus auf einem
2-Ply-Spielbaum mit dem \textsc{Max}-Knoten an der Wurzel. Die optimalste
Zugfolge verläuft dabei über die Knoten $\text{D} \rightarrow \text{A}
\rightarrow 0,7$.

\section{Alpha-Beta Pruning}

Die Berechnung der Minimax-Werte erfordert im Worst-Case die vollständige
Auswertung des Spielbaums bis zur gewählten Suchtiefe.

\textbf{Alpha-Beta Pruning} ist eine Optimierung des Minimax-Verfahrens, bei
der Teile des Spielbaums verworfen werden, ohne das Ergebnis der Berechnung zu
verändern. Die Grundidee besteht darin, Knoten nicht weiter zu untersuchen,
wenn bereits feststeht, dass sie keinen Einfluss auf die endgültige
Entscheidung haben können.

Die Optimierung hat ihren Namen von den beiden Parametern, die sie einführt:

\begin{itemize}
    \item Der Wert $\alpha$ der besten Entscheidung, die wir bisher an jedem
        Entscheidungspunkt entlang des Pfades für \textsc{Max} gefunden haben

    \item Der Wert $\beta$ der besten Entscheidung, die wir bisher an jedem
        Entscheidungspunkt entlang des Pfades für \textsc{Min} gefunden haben
\end{itemize}

Das Verfahren aktualisiert während der Baumdurchsuchung die Werte von $\alpha$
und $\beta$ und schneidet die verbleibenden Zweige ab, sobald der Wert am
aktuellen Knoten für \textsc{Max} bzw. \textsc{Min} schlechter ist als der
jeweils aktuelle $\alpha$ bzw. $\beta$-Wert.

Die Abbildung~\ref{fig:alpha-beta-on-a-2-ply-game-tree} veranschaulicht die
Vorgehensweise anhand eines Spielbaums aus dem vorherigen Abschnitt. In diesem
Beispiel handelte es sich lediglich um Blattknoten, normalerweise werden jedoch
ganze Teilbäume abgeschnitten.

\begin{figure}[h]
    \centering

    \begin{subfigure}[t]{0.49\textwidth}
        \includegraphics[width=\linewidth]{Abbildungen/alpha-beta-a.png}
    \end{subfigure}
    \hfill
    \begin{subfigure}[t]{0.49\textwidth}
        \includegraphics[width=\linewidth]{Abbildungen/alpha-beta-b.png}
    \end{subfigure}

    \begin{subfigure}[t]{0.49\textwidth}
        \includegraphics[width=\linewidth]{Abbildungen/alpha-beta-c.png}
    \end{subfigure}
    \hfill
    \begin{subfigure}[t]{0.49\textwidth}
        \includegraphics[width=\linewidth]{Abbildungen/alpha-beta-d.png}
    \end{subfigure}

    \begin{subfigure}[t]{0.49\textwidth}
        \includegraphics[width=\linewidth]{Abbildungen/alpha-beta-e.png}
    \end{subfigure}
    \hfill
    \begin{subfigure}[t]{0.49\textwidth}
        \includegraphics[width=\linewidth]{Abbildungen/alpha-beta-f.png}
    \end{subfigure}

    \caption{Alpha-Beta Pruning auf einem 2-Ply-Spielbaum}
    \label{fig:alpha-beta-on-a-2-ply-game-tree}
\end{figure}

\section{Forward Pruning}
\label{sec:forward-pruning}

Eine weitere Form des Prunings im Spielbaum ist das sogenannte \textbf{Forward
Pruning}. Dabei werden Züge, die auf den ersten Blick wenig erfolgversprechend
erscheinen, vorzeitig abgeschnitten, um Rechenzeit zu sparen. Dies geschieht
jedoch auf Kosten der Genauigkeit, da dadurch potenziell gute Züge übersehen
werden können.

Ein spezieller Ansatz des Forward Prunings ist die \textbf{Late Move
Reduction}. Dabei wird angenommen, dass die Züge in sinnvoller Reihenfolge
untersucht werden, sodass die später betrachteten Züge weniger wahrscheinlich
gute Züge sind. Anstatt diese Züge komplett zu verwerfen, reduziert man
lediglich die Tiefe, bis zu der sie untersucht werden, und spart dadurch
Rechenzeit.

Falls die verkürzte Suche jedoch einen Wert liefert, der über dem aktuellen
$\alpha$-Wert liegt, kann die Suche für diesen Zug erneut mit voller Tiefe
durchgeführt werden.

\section{Zugreihenfolge}

Die Effektivität des Spielbaum-Pruning hängt stark von der Reihenfolge ab, in
der die Zustände untersucht werden. Wie in Abschnitt~\ref{sec:forward-pruning}
erläutert, kann eine ungünstige Zugreihenfolge dazu führen, dass
Forward-Pruning-Verfahren auch gute Züge verwerfen und damit die Genauigkeit
der Suche beeinträchtigen. Beim Alpha-Beta-Verfahren führt eine schlechte
Zugordnung zwar nicht zu falschen Ergebnissen, kann die Suche jedoch deutlich
verlangsamen, da ungünstige Züge früh untersucht werden können und dadurch
weniger Möglichkeiten zum Abschneiden entstehen.

Könnte die Zugreihenfolge perfekt bestimmt werden, müsste Alpha-Beta-Pruning
zur Bestimmung des besten Zuges nur $O(b^{m/2})$ Knoten untersuchen, anstatt
$O(b^m)$ beim klassischen Minimax-Verfahren. Dies entspricht einem effektiven
Verzweigungsfaktor von $\sqrt{b}$ statt $b$ — im Schach also etwa $6$ statt
$35$.

Aus naheliegenden Gründen ist eine perfekte Zugordnung in den meisten Spielen
jedoch nicht erreichbar, da eine solche Ordnungsfunktion andernfalls direkt zur
Bestimmung einer optimalen Spielstrategie verwendet werden könnte. Trotzdem
kann man der idealen Ordnung relativ nahekommen.

Im Schach liefert bereits eine vergleichsweise einfache Zugordnungsheuristik,
die zunächst Schlagzüge, anschließend Drohungen, danach Vor- und zuletzt
Rückwärtszüge betrachtet, eine Annäherung an das Best-Case-Verhalten von
$O(b^{m/2})$.

Durch den Einsatz dynamischer Zugordnungsverfahren, etwa indem zuerst Züge
untersucht werden, die sich in früheren Suchen als besonders gut erwiesen
haben, lässt sich die Leistung weiter steigern.
