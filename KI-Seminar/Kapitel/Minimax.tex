\chapter{Minimax}

\section{Grundidee}

Betrachten wir ein Zwei-Akteur-Nullsummenspiel mit vollständiger Information
(siehe Abschnitt~\ref{sec:information-availability-in-games}). Die beiden
Spieler werden als \textsc{max} und \textsc{min} bezeichnet, wobei \textsc{max}
den Spielwert maximiert, während \textsc{min} ihn minimiert. Dabei kann ein
\textbf{Spielwert} jeden sinnvollen Wert zur Bewertung einer Position in einem
Nullsummenspiel darstellen, um ihre Vorteilhaftigkeit für beide Spieler
auszudrücken.

\begin{figure}[h]
    \centering
    \includegraphics[width=0.7\textwidth]{Abbildungen/minimax-game-tree.png}
    \caption{Ein 2-Ply-Spielbaum mit Zustandsbewertungen und den berechneten Minimax-Werten}
    \label{fig:minimax-game-tree}
\end{figure}

In der Abbildung~\ref{fig:minimax-game-tree} stehen $\diamond$ und $\circ$
jeweils für \textsc{max}- und \textsc{min}-Knoten, an denen der entsprechende
Spieler am Zug ist. Terminale Knoten, dargestellt durch $\square$, zeigen die
Endbewertung des Spiels, während alle anderen Knoten die daraus berechneten
Minimax-Werte enthalten.

Beide Spieler durchsuchen die verfügbaren Züge und treffen die jeweils für sie
vorteilhafteste Entscheidung. So verfügt der erste \textsc{min}-Knoten in
Abbildung~\ref{fig:minimax-game-tree} über drei mögliche Aktionen, die zu Positionen
mit den Spielwerten 7, 30 und 2 führen; sein Minimax-Wert beträgt daher 2. Die
beiden weiteren \textsc{min}-Knoten erhalten entsprechend die Werte 5 und 13.
Damit bestimmt sich der Wert des \textsc{max}-Knotens als Maximum der Werte
seiner Nachfolger, in diesem Fall 13.
