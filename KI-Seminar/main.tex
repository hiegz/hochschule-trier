%------------------ vorlage.tex ------------------------------------------------
%
%
%-------------------------------------------------------------------------------


%------------------ Präambel ---------------------------------------------------
\documentclass[envcountsame, envcountchap, deutsch, bibtotoc]{i-studis}
\usepackage{parskip}
\usepackage[utf8]{inputenc}

\usepackage[a4paper]{geometry}
\usepackage[english, ngerman]{babel}
\usepackage{float}
\usepackage{nameref}

\usepackage[pdftex]{graphicx}
\usepackage{epstopdf}

\usepackage{listings}
\usepackage{makecell}

\usepackage[german, ruled, vlined]{algorithm2e}
\usepackage{amssymb, amsfonts, amstext, amsmath}
\usepackage{array}
%\usepackage[skip=10pt]{caption}
\usepackage[usenames, dvipsnames]{color}
\usepackage{xcolor}
%\usepackage{sectsty}
\usepackage{textcomp}
\usepackage{booktabs}
\usepackage{wrapfig}
\usepackage{titlesec}
\usepackage{blindtext}
\usepackage[shortcuts]{extdash}
% \usepackage{enumerate}
\usepackage{enumitem}

% Zitierpakete
\usepackage[style=apa,backend=biber,style=alphabetic, maxcitenames=2]{biblatex}
% \DeclareLanguageMapping{german}{english}
\usepackage[pdftex, plainpages=false, breaklinks=true]{hyperref}
\usepackage[babel, german=quotes]{csquotes}
\addbibresource{Literatur.bib}

\usepackage{makeidx}
\usepackage{multicol}
\setlength{\tabcolsep}{0.5em}                  % for horizontal padding in tables

% Einstellungen für Code-Listings
\lstset{
        basicstyle=\ttfamily\scriptsize,       % print whole listing small and in monospace
        keywordstyle=\color{blue}\bfseries,    % underlined bold black keywords
        identifierstyle=,                      % nothing happens
        commentstyle=\color{red},              % white comments
        stringstyle=\ttfamily,                 % typewriter type for strings
        showstringspaces=false,                % no special string spaces
        framexleftmargin=7mm,
        tabsize=3,
        showtabs=false,
        frame=single,
        rulesepcolor=\color{blue},
        numbers=left,
        linewidth=146mm,
        xleftmargin=8mm,
        captionpos=b
}

\graphicspath{ {./Abbildungen/} }

\hypersetup{
    colorlinks,
    linkcolor={black},
    citecolor={blue!50!black},
    urlcolor={blue!80!black}
}

\makeindex

\pagestyle{myheadings}
\setlength{\textheight}{1.1\textheight}

%------------------ Titelseite -------------------------------------------------
\begin{document}
\setcounter{secnumdepth}{2}
\title{Adversarial Search}
\project{Seminararbeit}
% \degree{Bachelor of Science (B.Sc.)}                    % Nur bei Abschlussarbeit!
\address{im Studiengang Künstliche Intelligenz und Data Science an der\\Hochschule Trier}

\author{Volodymyr But}
\supervisorFirst{Prof. Dr. Claudia Schon}
\matrikelnummer{982324}
\fachbereich{Informatik}
\studiengang{Künstliche Intelligenz und Data Science}
\submitdate{26.02.2026}

\mytitlepage

%------------------ Vorwort, Kurzfassung, Verzeichnisse ------------------------
\frontmatter
\preface

An dieser Stelle möchte ich mich gerne bei allen Personen bedanken
die mich bei dieser Bachelorarbeit unterstützt haben.

Mein Dank gilt insbesondere ...
                                % Vorwort (optional)
\kurzfassung

Thema dieser Seminararbeit ist der Minimax-Algorithmus als grundlegendes Verfahren zur
optimalen Entscheidungsfindung in Nullsummenspielen mit vollständiger
Information. Es wurde erläutert, wie Spielbäume aufgebaut und bewertet werden,
wie Minimax-Werte berechnet werden und welche Optimierungen, wie
Alpha-Beta-Pruning, die Effizienz der Suche verbessern. Dabei wurde auch auf
weiterführende Techniken wie Forward Pruning, Late Move Reduction, Quiescence
Search und singuläre Erweiterungen eingegangen. Dabei wurden auch die
wesentlichen Grenzen des Ansatzes erläutert.
    						% Kurzfassung/Abstract (nur bei Abschlussarbeiten)
\tableofcontents										% Inhaltsverzeichnis
\listoffigures                        % Abbildungsverzeichnis
\listoftables                           % Tabellenverzeichnis
%------------------ Kapitel ----------------------------------------------------
\mainmatter
\chapter{Einleitung}

In vielen realen Entscheidungssituationen hängt der Erfolg einer Handlung nicht
allein von den eigenen Entscheidungen ab, sondern wesentlich vom Verhalten
anderer Akteure. Besonders in konkurrierenden Umgebungen ist es häufig
erforderlich, Strategien zu entwickeln, die die Interessen der übrigen
Teilnehmer berücksichtigen, welche sich meist erheblich von den eigenen
unterscheiden und mitunter sogar im Widerspruch dazu stehen. Das Ergebnis
solcher Handlungen ist daher nicht eindeutig vorhersehbar, sondern entsteht aus
dem Zusammenspiel aller beteiligten Akteure.

Im Bereich der KI-Forschung beschäftigen sich Wissenschaftler schon seit langem
mit der formalen Modellierung und algorithmischen Lösung solcher
Entscheidungsprobleme. Ein zentraler Aspekt besteht dabei darin,
Handlungsstrategien zu entwickeln, die auch unter ungünstigen Bedingungen zu
möglichst guten Ergebnissen führen. Besonders problematisch sind dabei
Situationen, in denen ein Akteur einem oder mehreren rational agierenden
Gegenspielern gegenübersteht, die aktiv versuchen, seinen Erfolg zu verhindern.
Eine Handlung ist daher nur dann sinnvoll, wenn sie auch unter Berücksichtigung
des bestmöglichen gegnerischen Verhaltens Bestand hat. Diese Art von Problemen
wird unter dem Begriff \textbf{Adversarial Search} zusammengefasst.

Die vorliegende Arbeit befasst sich mit einem Teilbereich dieser Forschung, der
sich mit Entscheidungsproblemen in konkurrierenden Szenarien befasst,
insbesondere in strategischen Spielen wie Tic-Tac-Toe oder Schach. Dabei liegt
der Schwerpunkt auf der formalen Modellierung solcher Situationen sowie auf
Ansätzen, die eine rationale und systematische Entscheidungsfindung
ermöglichen. Es werden die grundlegenden Problemen diskutiert, mögliche
Lösungsstrategien verglichen und deren Vor- und Nachteile besprochen. Ziel ist
es, ein umfassendes Verständnis dafür zu vermitteln, wie Entscheidungen in
komplexen und interaktiven Umgebungen getroffen werden können.

\chapter{Nullsummenspiele}

Im Kontext der adversarialen Suche werden in der Fachliteratur überwiegend
sogenannte \textbf{Nullsummenspiele} betrachtet. Dabei handelt es sich um
Entscheidungsszenarien, in denen der Vorteil eines Akteurs gleichzeitig einen
gleich großen Nachteil für den Gegenspieler bedeutet. Das Grundprinzip besteht
darin, dass der Gesamtnutzen aller Akteure für jedes mögliche Ergebnis konstant
bleibt, was zu einem direkten Widerspruch der Interessen führt.  Die optimale
Strategie bei solchen Spielen ergibt sich dabei nicht aus der Maximierung eines
einzelnen Ergebnisses, sondern aus der Wahl einer Strategie, die den
bestmöglichen Ausgang unter der Annahme eines optimal handelnden Gegners
garantiert.

Zudem wird oft zwischen Spielen mit vollständiger und unvollständiger
Information unterschieden. Bei \textbf{Spielen mit vollständiger Information}
ist allen beteiligten Akteuren der aktuelle Zustand des Systems zu jedem
Zeitpunkt bekannt. Jeder Spieler kann somit seine Entscheidungen unter
Berücksichtigung des gesamten bisherigen Spielverlaufs treffen. So können die
Spieler auch die möglichen Reaktionen des Gegenspieler analysieren, was die
Wahl der Handlung unterstützt, die den bestmöglichen Ausgang für den eigenen
Spieler verspricht. \textbf{Spiele mit unvollständiger Information} hingegen
zeichnen sich dadurch aus, dass einzelne Spieler über private Informationen
verfügen, die für andere nicht zugänglich sind. Eine weitere Variante sind
teilweise beobachtbare Spiele, bei denen die Teilnehmer nur begrenzte
Informationen über den aktuellen Stand erhalten, was die strategische
Entscheidungsfindung zusätzlich erschwert.

Diese Arbeit konzentriert sich auf \textbf{Zwei-Akteur-Nullsummenspiele mit
\linebreak vollständiger Information}. Anschauliche Beispiele für diese Art von Spielen
sind Tic-Tac-Toe und Schach. Diese Einschränkung erlaubt es, die zentralen
Konzepte und Algorithmen der adversarialen Suche klar zu analysieren und deren
Funktionsweise nachvollziehbar darzustellen. Grundsätzlich sind viele der
vorgestellten Verfahren auch auf Szenarien mit mehr als zwei Akteuren
skalierbar, auf diese wird jedoch nur am Rande eingegangen. Andere Klassen von
Spielen, insbesondere solche mit unvollständiger Information, erfordern
grundlegend abweichende Ansätze und werden im Rahmen dieser Arbeit daher nicht
behandelt.

\chapter{Suchkomponenten}

\section{Spielzustandsrepräsentation}

Die Repräsentation des Spielzustands bildet die Grundlage jeder algorithmischen
Suche. Ein Spielzustand muss alle Informationen enthalten, die notwendig sind,
um den weiteren Spielverlauf eindeutig zu bestimmen. Die Art der Repräsentation
kann die Verarbeitung von Zuständen und Zustandsübergängen erheblich
erleichtern, was wiederum Aspekte wie die Zugerzeugung beeinflussen kann.

Während kein allgemeiner Standard für alle Spiele existiert, haben sich in
bestimmten Bereichen, etwa im Schach, Bitboards etabliert. Dabei kodiert ein
64-Bit-Wort die Positionen aller Figuren eines bestimmten Typs. Diese
Darstellung erlaubt parallele Ausführung von Operationen auf allen 64 Feldern.
Neben der schnelleren Verarbeitung erleichtert diese Darstellung auch die
Speicherung und den Zugriff auf Zustände in bestimmten Datenstrukturen, was
beispielsweise für Lookups oder andere Operationen innerhalb der
Suchalgorithmen von Vorteil ist. Die Abbildung~\ref{fig:bitboard} soll das
Prinzip der Bitboard-Darstellung veranschaulichen.

\begin{figure}[h]
    \centering
    \includegraphics[width=0.85\textwidth]{Abbildungen/bitboard.png}
    \caption{Bitboard-Darstellung}
    \label{fig:bitboard}
\end{figure}

Die Wahl der Spielzustandsrepräsentation sollte sorgfältig auf den jeweiligen
Anwendungsfall abgestimmt werden. Bestimmte Repräsentationen, wie
etwa Bitboards, können die Berechnungen zwar beschleunigen, erhöhen dabei
jedoch den Speicherbedarf, da sie Informationen über sowohl besetzte als auch
leere Felder speichern. So belegt ein Bitboard beispielsweise 8 Bytes, nur um
die Position einer einzelnen Figur (z.B. des Königs) darzustellen, die
theoretisch in einem Byte gespeichert werden könnte. Daher ist es möglich, für
ein und dasselbe Spielmodell mehrere Repräsentationen zu verwenden, um
unterschiedliche Aspekte der Suche optimal zu unterstützen.

\section{Zugerzeugung}

Aus einem gegebenen Spielzustand lassen sich weitere Zustände ableiten, indem
alle möglichen Übergänge oder Züge bestimmt werden, die von diesem
Ausgangspunkt aus ausgeführt werden können. Dieser Prozess wird als
Zugerzeugung bezeichnet.

Ein zentrales Ziel der Zugerzeugung ist es, korrekt (nur regelkonforme Züge)
und vollständig (alle möglichen Züge) zu sein. Gleichzeitig muss sie effizient
arbeiten, da sie während der Suche tausend- bis millionenfach ausgeführt wird.
Unnötig generierte Züge oder aufwendige Regelprüfungen wirken sich direkt
negativ auf die Suchleistung aus.

In vielen Spielen wird zwischen Pseudo-Legalität und tatsächlicher Legalität
unterschieden: Zunächst werden Kandidatenzüge generiert, die anschließend durch
zusätzliche Regelprüfungen gefiltert werden. Diese Zweistufigkeit ist ein
gängiger Kompromiss zwischen Geschwindigkeit und Korrektheit.

\section{Bewertungsfunktion}

Vielleicht der wichtigste Bestandteil eines Schachprogramms ist die
Bewertungsfunktion.

Theoretisch wäre es möglich, den besten Zug zu bestimmen, indem man
ausschließlich Züge auswählt, die Teil einer ‚gewinnbringenden‘ Zugfolge sind –
also einer Sequenz, die zu einem Sieg führt. Praktisch ist dies jedoch nahezu
unmöglich, da die Anzahl möglicher Spielverläufe derart groß ist, dass kein
moderner Computer in der Lage wäre, alle Szenarien zu durchsuchen, selbst wenn
beide Spieler optimal agieren würden.

Aus diesem Grund wird eine Bewertungsfunktion benötigt, die den Wert von
Positionen oder Spielzuständen abschätzt, die nicht vollständig durchsucht
werden können. Im einfachsten Fall berücksichtigt die Bewertungsfunktion im
Schach lediglich die Materialdifferenz. Für ein stärkeres Spiel ist es jedoch
erforderlich, auch zahlreiche positionelle Faktoren zu berücksichtigen, wie
etwa die Struktur der Bauern.

Hier zeigen sich die Unterschiede zwischen guten und schlechten Suchprogrammen.
Außerdem ermöglicht die Bewertungsfunktion nicht nur eine präzisere Auswahl von
Zügen, sondern definiert auch den Charakter, mit dem das Programm das Spiel
spielt.

\section{Spielbäume}

Um solche Umgebungen systematisch analysieren zu können, wird der
Entscheidungsraum in Form eines Spielbaums modelliert, der aus Spielzuständen
und Übergängen besteht und alle möglichen Zugfolgen bis zu den terminalen
Zuständen abbildet, also jenen Zuständen, in denen das Spiel endet und keine
weiteren Übergänge mehr möglich sind. In komplexen Spielen wie Schach wächst
dieser Baum jedoch bereits nach wenigen Zügen beider Spieler derart stark an,
dass kein moderner Computer in vertretbarer Zeit sämtliche Knoten vollständig
durchsuchen könnte. Aus diesem Grund werden Spielbäume in der Praxis nur bis zu
einer festgelegten Tiefe betrachtet.

In Abbildung~\ref{fig:gametree} ist ein Beispiel solches Spielbaums für Tic-Tac-Toe mit einer
Tiefe von 2 dargestellt.

\begin{figure}[h]
    \centering
    \includegraphics[width=1\textwidth]{Abbildungen/gametree.png}
    \caption{Spielbaum für Tic-Tac-Toe mit einer Tiefe von 2}
    \label{fig:gametree}
\end{figure}

\input{Kapitel/Klassische_Suchverfahren.tex}
\chapter{Minimax}

\section{Grundidee}

Betrachten wir ein Zwei-Akteur-Nullsummenspiel mit vollständiger Information
(siehe Abschnitt~\ref{sec:information-availability-in-games}). Die beiden
Spieler werden als \textsc{max} und \textsc{min} bezeichnet, wobei \textsc{max}
den Spielwert maximiert, während \textsc{min} ihn minimiert. Dabei kann ein
\textbf{Spielwert} jeden sinnvollen Wert zur Bewertung einer Position in einem
Nullsummenspiel darstellen, um ihre Vorteilhaftigkeit für beide Spieler
auszudrücken.

\begin{figure}[h]
    \centering
    \includegraphics[width=0.7\textwidth]{Abbildungen/minimax-game-tree.png}
    \caption{Ein 2-Ply-Spielbaum mit Zustandsbewertungen und den berechneten Minimax-Werten}
    \label{fig:minimax-game-tree}
\end{figure}

In der Abbildung~\ref{fig:minimax-game-tree} stehen $\diamond$ und $\circ$
jeweils für \textsc{max}- und \textsc{min}-Knoten, an denen der entsprechende
Spieler am Zug ist. Terminale Knoten, dargestellt durch $\square$, zeigen die
Endbewertung des Spiels, während alle anderen Knoten die daraus berechneten
Minimax-Werte enthalten.

Beide Spieler durchsuchen die verfügbaren Züge und treffen die jeweils für sie
vorteilhafteste Entscheidung. So verfügt der erste \textsc{min}-Knoten in
Abbildung~\ref{fig:minimax-game-tree} über drei mögliche Aktionen, die zu Positionen
mit den Spielwerten 7, 30 und 2 führen; sein Minimax-Wert beträgt daher 2. Die
beiden weiteren \textsc{min}-Knoten erhalten entsprechend die Werte 5 und 13.
Damit bestimmt sich der Wert des \textsc{max}-Knotens als Maximum der Werte
seiner Nachfolger, in diesem Fall 13.

\input{Kapitel/Alpha_Beta_Pruning.tex}
\chapter{Schlussbetrachtung}

Diese Seminararbeit hatte zum Ziel, den Minimax-Algorithmus und seine
praktischen Einsatzmöglichkeiten in der adversarialen Suche zu erläutern. Dazu
wurden die theoretischen Grundlagen, Optimierungsverfahren und typische
Probleme in der Praxis beschrieben. Es kann festgehalten werden, dass Minimax
eine solide Basis für die Entscheidungsfindung in deterministischen
Nullsummenspielen darstellt und in Verbindung mit Optimierungen wie
Alpha-Beta-Pruning in modernen Expertensystemen wie Stockfish erfolgreich
angewendet wird. Gleichzeitig zeigt sich, dass die Genauigkeit und Effizienz
der Suche durch Faktoren wie die Bewertungsfunktion, die Zugreihenfolge und die
Größe des Aktionsraums begrenzt sind. Für sehr komplexe Spiele oder Szenarien
mit unüberschaubarem Aktionsraum sind daher alternative Strategien notwendig.

%------------------ Literaturverzeichnis & Index -------------------------------
\backmatter
\printbibliography
\printindex												% Index (optional)
%------------------ Anhänge ----------------------------------------------------
\begin{appendix}
	\chapter{Glossar}

\abbreviation{$\text{BK}_{i,t}$}    {Börsenkurs der Aktie $i$ zum Zeitpunkt $t$}
\abbreviation{$\text{D}_{i,t}$}	    {Dividendenzahlung der Aktie $i$ zum Zeitpunkt $t$}
\abbreviation{$\text{r}_i$}         {Rendite der Aktie $i$}
\abbreviation{...}      		    {...}
						% Glossar (optional)
    \chapter{Häufige Fehler in Seminar- und Abschlussarbeiten}

An dieser Stelle wird ein Überblick darüber gegeben, welche Fehler in
schriftlichen wissenschaftlichen Arbeiten häufig auftreten. Die
genannten Punkte können zur Reflektion und zur Prüfung in allen
Phasen des Verfassens einer Seminar- oder Abschlussarbeit
herangezogen werden.

\section{Inhalt}

\emph{Erfassung der Aufgabenstellung}
\begin{itemize}
    \item Das Thema wird nicht vollständig erfasst, einzelne Themenbestandteile werden nicht berücksichtigt, die Beziehungen zwischen den Themenbestandteilen werden nicht herausgearbeitet.
    \item Es wird ein Thema bearbeitet, welches sich nicht aus der Aufgabenstellung erschließt.
\end{itemize}

\emph{Aufbau der Arbeit und Inhalt}
\begin{itemize}
    \item Technische Gliederungsfehler
    \item Der Fluss der Gliederung ist nicht ersichtlich.
    \item Die Gliederungsteile bauen nicht aufeinander auf, sie stehen vielmehr isoliert nebeneinander.
    \item Die Überschriften sind nicht aussagekräftig.
    \item Eine Überschrift deckt sich vollständig mit dem Thema der Arbeit.
\end{itemize}

\emph{Anmerkungen zum Inhalt}
\begin{itemize}
    \item Problemstellung und Zielsetzung der Arbeit werden nur oberflächlich angerissen.
    \item Allgemeinplätze: „Globalisierung und zunehmende Dynamik ...“
\end{itemize}

\newpage
\emph{Eigenständigkeit der erbrachten Leistung}
\begin{itemize}
    \item Mangelnde Eigenständigkeit der erbrachten Leistung
    \item Zu viele direkte Zitate, mangelnde Reflektion in eigenen Worten
    \item Abbildungen und Tabellen werden 1:1 übernommen ohne kontextbezogene Eigenleistung und Darstellung.
\end{itemize}


    \chapter{Wissenschaftliche Methode}

\emph{Begriffsbildung, Definition, Abgrenzung}
\begin{itemize}
    \item Begriffe werden nicht eingeführt, gar nicht oder aber erst später definiert.
    \item Falsche oder ungenaue Begriffsbildungen und -verwendungen
    \item Begriffe werden nicht überschneidungsfrei voneinander abgegrenzt.
    \item Gleiche Tatbestände werden mit unterschiedlichen Begriffen belegt.
    \item Mit der verwendeten Quelle wechselt die Notation, d. h. Begriffe werden mit unterschiedlichen Symbolen belegt.
\end{itemize}

\emph{Gedankenführung und Aufbau}
\begin{itemize}
    \item Die Kapitelüberschriften behandeln nicht oder nicht exakt das Thema der Arbeit.
    \item Sämtliche Überschriften einer tieferen Gliederungsebene behandeln nicht oder nicht exakt das Thema der Überschrift auf der nächst höheren Gliederungsebene.
    \item Es erfolgt die Untergliederung eines Absatzes, wobei nur ein Unterpunkt ausgewiesen wird: z. B. Absatz 3 soll untergliedert werden. Es wird nur ein Absatz 3.1 aufgeführt, 3.2, 3.3 usw. fehlen.
    \item Die Ausführungen innerhalb eines Absatzes betreffen nicht die Überschrift des Absatzes.
\item Der Fluss der Gedanken muss auf jeder Ebene deutlich sein; dies gilt sowohl für die gesamte Gliederung als auch für einzelne Absätze (auch die müssen strukturiert sein und einem „roten Faden“ folgen).
    \item Oft fehlen die Überleitungen zwischen einzelnen Kapiteln, der Gang der Arbeit wird nicht inhaltlich begründet (sowohl in der Einleitung als auch in den folgenden Ausführungen).
\end{itemize}

\newpage
\emph{Ergebnisbildung}
\begin{itemize}
    \item Angesichts einer in der Einleitung beschriebenen Problemstellung fehlt ein Diskussionsergebnis bzw. es ist unklar.
    \item Das Ergebnis hat wenig mit der in der Einleitung aufgeworfenen Problemstellung sowie den Untersuchungszielen zu tun.
    \item Das Ergebnis wird nicht eindeutig formuliert, der Kandidat „drückt“ sich um eine Stellungnahme.
\end{itemize}

\end{appendix}
\eidesstattlicheErklaerung

Ich erkläre hiermit, dass ich die vorliegende Bachelorarbeit mit dem
Titel \enquote{Adversarial Search} selbst verfasst habe und dass ich dazu keine
anderen als die angegebenen Hilfsmittel und Quellen verwendet habe. Alle
Textstellen, die wörtlich oder sinngemäß aus anderen Werken übernommen wurden,
sind als solche gekennzeichnet. Die Arbeit hat in gleicher oder ähnlicher Form
noch keiner anderen Prüfungsstelle vorgelegen.

\vspace{3cm}

\begin{minipage}[t]{3cm}
	\rule{3cm}{0.5pt}
	Ort, Datum
\end{minipage}
\hfill
\begin{minipage}[t]{9cm}
	\rule{9cm}{0.5pt}
	Unterschrift der Kandidatin\slash des Kandidaten
\end{minipage}
                  % Eidesstaatliche Erklärung


\end{document}
