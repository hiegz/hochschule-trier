\documentclass[10pt, oneside]{article}
\usepackage[a4paper, total={5.5in, 9in}]{geometry}
\usepackage[ngerman]{babel}
\usepackage{import}

\import{../../.texit/include}{preamble}

\title{Einführung in die Künstliche Intelligenz\\[15pt]\Large{Übungsblatt 3}\\[10pt]\Large{SoSe 2025}}
\author{Volodymyr But\\[10pt]Hochschule Trier}
\date{}

% - - - - - - - - - - - - - - - - - - - - - - - - - - - - - - - - - - - - - - %

\begin{document}

\maketitle
\vspace{25px}

\section{Klassifikatoren vergleichen}

\begin{enumerate}[1.]
    \item Erstellen Sie f"ur beide Klassifikatoren jeweils die Confusion Matrix
        \begin{table}[h]
            \centering
            \begin{minipage}[t]{0.4\linewidth}
                \centering
                \begin{tabular}{|c|c|c|}
                    \hline
                    \  & GP & GN \\
                    \hline
                    SP & 3  & 2  \\
                    \hline
                    SN & 1  & 2  \\
                    \hline
                \end{tabular}
                \caption{Confusion Matrix f"ur Klassifikator $A$}
            \end{minipage}
            \hspace{20px}
            \begin{minipage}[t]{0.4\linewidth}
                \centering
                \begin{tabular}{|c|c|c|}
                    \hline
                    \  & GP & GN \\
                    \hline
                    SP & 2  & 2  \\
                    \hline
                    SN & 2  & 2  \\
                    \hline
                \end{tabular}
                \caption{Confusion Matrix f"ur Klassifikator $B$}
            \end{minipage}
        \end{table}

    \item Berechnen Sie für beide Klassifikatoren \verb|Precision|, \verb|Recall| und \verb|Accuracy|.
        \begin{equation*}
            \begin{array}{rcl}
                \text{Precision}_A = \dfrac{3}{3 + 2}             = \dfrac{3}{5} & & \text{Precision}_B   = \dfrac{2}{2 + 2}     = \dfrac{1}{2} \\[15pt]
                \text{Recall}_A    = \dfrac{3}{3 + 1}             = \dfrac{3}{4} & & \text{Recall}_B      = \dfrac{2}{2 + 2}     = \dfrac{1}{2} \\[15pt]
                \text{Accuracy}_A  = \dfrac{3 + 2}{3 + 2 + 2 + 1} = \dfrac{5}{8} & & \text{Accuracy}_B    = \dfrac{2 + 2}{4 * 2} = \dfrac{1}{2}
            \end{array}
        \end{equation*}

    \item Berechnen Sie $\text{Comp}(A, B)$ und $\text{Comp}(B, A)$

        $\text{Comp}(A, B) = (1 - \dfrac{|A_{error} \cap B_{error}|}{|A_{error}|}) \cdot 100 = (1 - \dfrac{1}{3}) \cdot 100 = 66.6\%$ \\[10pt]
        $\text{Comp}(B, A) = (1 - \dfrac{|A_{error} \cap B_{error}|}{|B_{error}|}) \cdot 100 = (1 - \dfrac{1}{4}) \cdot 100 = 75\%$

    \item Erkl"aren Sie mit eigenen Worten, was der Wert $\text{Comp}(A, B)$ aussagt.

        $\text{Comp}(A, B)$ gibt den Anteil der Fehler von Klassifikator A
        zurück, die Klassifikator B nicht gemacht hat.
\end{enumerate}

\end{document}
