\documentclass[10pt, oneside]{article}
\usepackage[a4paper, total={5.5in, 9in}]{geometry}
\usepackage[ngerman]{babel}
\usepackage{import}

\import{../../.texit/include}{preamble}

\title{Einführung in die Künstliche Intelligenz\\[15pt]\Large{Übungsblatt 10}\\[10pt]\Large{SoSe 2025}}
\author{Volodymyr But\\[10pt]Hochschule Trier}
\date{}

% - - - - - - - - - - - - - - - - - - - - - - - - - - - - - - - - - - - - - - %

\begin{document}

\maketitle
\vspace{25px}

\section{Ritter und Schurken}

Auf der Insel der Ritter und Schurken ist jeder Bewohner entweder ein Ritter oder ein
Schurke. Dabei sagen die Ritter immer die Wahrheit und die Schurken lügen immer.

Wir begegnen drei Inselbewohnern, die am Straßenrand stehen.
\begin{itemize}
    \item Person A sagt: „Wir drei sind alle Schurken.“
    \item Person B sagt: „Genau einer von uns ist ein Ritter.“
\end{itemize}
Wisenbasis:
\begin{align*}
    F &= A \Leftrightarrow (\lnot A \land \lnot B \land \lnot C) \\[5pt]
    G &= B \Leftrightarrow (A \land \lnot B \land \lnot C) \lor (\lnot A \land B \land \lnot C) \lor (\lnot A \land \lnot B \land C) \\[5pt]
    W &= F \land G
\end{align*}
\begin{table}[h]
    \centering
    \begin{tabular}{c|c|c|c}
        $A$    & $B$    & $C$    & $W$    \\
        \hline
        0 & 0 & 0 & 0 \\
        0 & 0 & 1 & 0 \\
        0 & 1 & 0 & 0 \\
        0 & 1 & 1 & 0 \\
        1 & 0 & 0 & 0 \\
        1 & 0 & 1 & 0 \\
        1 & 1 & 0 & 0 \\
        1 & 1 & 1 & 0 \\
    \end{tabular}
    \caption{Wahrheitstabelle}
\end{table}

\end{document}
