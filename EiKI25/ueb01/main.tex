\documentclass[10pt, oneside]{article}
\usepackage[a4paper, total={5.5in, 9in}]{geometry}
\usepackage[ngerman]{babel}
\usepackage{import}

\import{../../.texit/include}{preamble}

\title{Einführung in die Künstliche Intelligenz\\[15pt]\Large{Übungsblatt 1}\\[10pt]\Large{SoSe 2025}}
\author{Volodymyr But\\[10pt]Hochschule Trier}
\date{}

% - - - - - - - - - - - - - - - - - - - - - - - - - - - - - - - - - - - - - - %

\begin{document}

\maketitle
\vspace{25px}

\section{Geometrische Analogien und KI}

\begin{enumerate}[(a)]
    \item \begin{enumerate}[a)]
        \item C-3. Das Bild wird durch Hinzuf"ugen einer weiteren Schicht von
            derselben geometrischen Figur erweitert.
        \item C-3. Die innere Figur wird entfernt.
        \item C-1. Das Bild wird vertikal gespiegelt.
        \item C-2. Der innere Kreis wird in einen Halbkreis umgewandelt.
    \end{enumerate}
    \item Menschen müssen keine tausenden Übungsaufgaben lösen, um Muster
        erkennen zu k"onnen. Unser Gehirn scheint intuitiv zu generalisieren
        und zu abstrahieren, oft aus nur wenigen Beispielen. Im Gegensatz dazu
        müssen Maschinen meist mit enormen Datenmengen trainiert werden, um
        ähnliche Fähigkeiten zu entwickeln. Selbst dann gelingt es ihnen nicht
        immer, zuverlässig zu unterscheiden – insbesondere, wenn es um
        komplexe, kontextabhängige Muster geht. Daher sind Aufgaben zur
        Mustererkennung für Computersysteme noch immer schwer zu lösen, während
        sie für die meisten Menschen vergleichsweise einfach erscheinen.
        Bedeutet das, dass wir intelligenter sind als Maschinen? Wahrscheinlich
        schon. Doch würde man eine Maschine, die solche Aufgaben zuverlässig
        bewältigen kann, als intelligent bezeichnen? Vielleicht. Vielleicht
        auch nicht. Um eine sinnvolle Antwort auf diese Frage zu formulieren,
        müssen wir zunächst eine solche Maschine bauen.
\end{enumerate}

\end{document}
