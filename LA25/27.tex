\section{Aufgabe 27}

Es sei $v, u_1, u_2 \in \mathbb{R}^n$, $c \in \mathbb{R}$. Es gelte
\begin{enumerate}[(1)]
    \item $\langle v, u_1 \rangle < c$,
    \item $\langle v, u_2 \rangle < c$,
    \item $w = \lambda u_1 + (1 - \lambda) u_2$.
\end{enumerate}
Aus (3) ergibt sich mittels Linearit"at des Skalarprodukts
\begin{equation*}
    \langle v, w \rangle = \langle v, \lambda u_1 + (1 - \lambda) u_2 \rangle = \langle v, \lambda u_1 \rangle + \langle v, (1 - \lambda) u_2 \rangle = \lambda \langle v, u_1 \rangle + (1 - \lambda) \langle v, u_2 \rangle
\end{equation*}
Und aus (1) und (2) folgt
\begin{equation*}
    \lambda \langle v, u_1 \rangle + (1 - \lambda) \langle v, u_2 \rangle < \lambda c + (1 - \lambda) c = \lambda c + c - \lambda c = c
\end{equation*}
Also gilt
\begin{equation*}
    \langle v, w \rangle < c \ \ \ \square
\end{equation*}
