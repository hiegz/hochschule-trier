\setcounter{section}{4}
\setcounter{subsection}{5} % 4.6
\subsection{IP-Adressen und Subnetze der Hochschule Trier}

Das IP-Netzwerk der Hochschule Trier verwendet die IP-Adressen
\verb|143.93.32.0| bis \verb|143.93.63.255|.

\begin{enumerate}[(a)]
    \item Geben Sie zunächst die beiden begrenzenden IP-Adressen in binärer
        Schreibweise an.

\begin{verbatim}
143.93.32.0    = 1000 1111.0101 1101.0010 0000.0000 0000
143.93.63.255  = 1000 1111.0101 1101.0011 1111.1111 1111
\end{verbatim}

    \item Welche CIDR-Netzmaske gehört zum Hochschule-Netzwerk?

        \verb|143.93.32.0/19|

    \item Sie wollen das Netzwerk auf acht Fachbereiche aufteilen. Welche
        Subnetzmaske verwenden Sie?

        F"ur 8 Fachbereiche, brauchen wir 3 ($8 = 2^3$) zus"atzliche Netz-Bits. Das ergibt:

        \begin{enumerate}[1.]
            \item \verb|143.93.32.0/22|
            \item \verb|143.93.36.0/22|
            \item \verb|143.93.40.0/22|
            \item \verb|143.93.44.0/22|
            \item \verb|143.93.48.0/22|
            \item \verb|143.93.52.0/22|
            \item \verb|143.93.56.0/22|
            \item \verb|143.93.60.0/22|
        \end{enumerate}

    \item Der Fachbereich Informatik ist der siebte / vorletzte Fachbereich. Welche IP-
        Adressen kann er für seine Hosts verwenden?

        \verb|143.93.56.1 - 143.93.59.254| (1022 Hosts)
\end{enumerate}
