\setcounter{section}{4}
\setcounter{subsection}{24} % 4.25
\subsection{Entfernungsvektor-Routing (2)}

F"ullen sie das angegebene Routing-Tabelle f"ur den Knoten A unter der Annahme,
dass $\text{Entfernung}(\text{A}, \text{B}) = 2$, $\text{Entfernung}(\text{A},
\text{C}) = 7$, $\text{Entfernung}(\text{A}, \text{E}) = 2$.

\begin{table}[h]
    \centering
    \begin{tabular}{|c|c|c|c|}
        \hline
          & Entfernung & Next Hop & Begründung\\
        \hline
        A & 0          & .        & . \\
        \hline
        B & 2          & B        & Direktverbindung ist am k"urzesten \\
        \hline
        C & 3          & B        & "Uber B ist am k"urzesten (2 + 1 = 3) \\
        \hline
        D & 5          & B        & "Uber B ist am k"urzesten (2 + 3 = 5) \\
        \hline
        E & 2          & E        & Direktverbindung ist am k"urzesten \\
        \hline
    \end{tabular}
    \caption{}
\end{table}
