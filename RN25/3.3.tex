\setcounter{section}{3}
\setcounter{subsection}{2} % 3.3
\subsection{Hamming-Code}

\newcommand{\cs}[1]{\texttt{\symbol{}#1}}

\begin{enumerate}[(a)]
    \item Gegeben sind die folgenden Daten-Bits einer Nachricht: \verb|1101001100110101|
        \begin{itemize}
            \item F"ugen Sie die ben"otigten Parity-Bits an der passenden
                Stelle ein und zeigen Sie, wie Sie die Parity-Bits berechnet
                haben. Vervollst"andigen Sie dazu die abgebildete Tabelle aus.

                Siehe Tabelle 3.1; die Parity-Bits wurden nach der in der
                Vorlesung vorgestellten Methode berechnet ("uber alle Stellen
                $c_j$, in denen an der $i$-ten Stelle der Bin"arkodierung des
                Index $j$ eine Eins steht)
            \item Wie lauten die kompletten Code-Bits?

                \verb+011110110011001110101+

            \item Angenommen das 10. Code-Bit kippt im "ubertragenen Bitmuster.
                Zeigen Sie, wie der Fehler korrigiert werden kann.

                Wir überprüfen alle Parity-Bits und addieren anschließend die
                Indizes derjenigen, die die Validierung nicht bestanden haben.

                In unserem Beispiel erhalten wir: $2 + 8 = 10\ \ \ (\text{also } d_6 \text{ oder } c_{10})$
        \end{itemize}

        \item Wie ist das Verhältnis im Allgemeinen beim Hamming-Code zwischen
            Daten-Bits und Parity-Bits?
            
            \begin{equation*}
                n : \text{log}_2(n) + 1
            \end{equation*}

        \item Durch Anwendung des Hamming-Codes werden jeweils 4 Bits mit 7 Bits
            codiert. Drei solcher 7-Bit-Codewörter werden wie folgt übertragen: zuerst
            jeweils das erste Bit der drei Codewörter, dann jeweils das zweite Bit der drei
            Codewörter usw. Auf Empfängerseite wird die Matrix von links nach rechts
            befüllt. Empfangen wurde folgende Bitfolge:

\begin{verbatim}
  1 0 1
  1 0 0
  1 0 1
  0 1 0
  0 0 0
  1 1 1
  0 1 0
\end{verbatim}

            Korrigieren Sie die empfangenen Bits. Wie lautet die Ausgangsnachricht
            (korrigiert und ohne Parity-Bits)?

\begin{verbatim}
  1 0 1
  1 0 0
  1 0 1   1 0 1
  0 1 1
  0 1 0   0 1 0
  0 1 1   0 1 1
  0 1 0   0 1 0
\end{verbatim}

            Bündelfehler bis zu welcher Länge können mit diesem Verfahren korrigiert
            werden?

            Wir haben drei individuelle Hamming-Codes, also drei.

\end{enumerate}

\begin{sidewaystable}
    \centering
    \begin{tabular}{|p{2.3cm}|c|c|c|c|c|c|c|c|c|c|c|c|c|c|c|c|c|c|c|c|c|} 
        \hline
        Code-Bits             & $c_1$                                      & $c_2$                                      & $c_3$                                      & $c_4$                                      & $c_5$                                      & $c_6$                                      & $c_7$                                      & $c_8$                                      & $c_9$                                      & $c_{10}$                                   & $c_{11}$                                   & $c_{12}$                                   & $c_{13}$                                   & $c_{14}$                                   & $c_{15}$                                   & $c_{16}$                                   & $c_{17}$                                   & $c_{18}$                                   & $c_{19}$                                   & $c_{20}$                                   & $c_{21}$                                   \\ 
        \hline
        Index                 & \rotatebox[origin=c]{90}{\texttt{ 00001 }} & \rotatebox[origin=c]{90}{\texttt{ 00010 }} & \rotatebox[origin=c]{90}{\texttt{ 00011 }} & \rotatebox[origin=c]{90}{\texttt{ 00100 }} & \rotatebox[origin=c]{90}{\texttt{ 00101 }} & \rotatebox[origin=c]{90}{\texttt{ 00110 }} & \rotatebox[origin=c]{90}{\texttt{ 00111 }} & \rotatebox[origin=c]{90}{\texttt{ 01000 }} & \rotatebox[origin=c]{90}{\texttt{ 01001 }} & \rotatebox[origin=c]{90}{\texttt{ 01010 }} & \rotatebox[origin=c]{90}{\texttt{ 01011 }} & \rotatebox[origin=c]{90}{\texttt{ 01100 }} & \rotatebox[origin=c]{90}{\texttt{ 01101 }} & \rotatebox[origin=c]{90}{\texttt{ 01110 }} & \rotatebox[origin=c]{90}{\texttt{ 01111 }} & \rotatebox[origin=c]{90}{\texttt{ 10000 }} & \rotatebox[origin=c]{90}{\texttt{ 10001 }} & \rotatebox[origin=c]{90}{\texttt{ 10010 }} & \rotatebox[origin=c]{90}{\texttt{ 10011 }} & \rotatebox[origin=c]{90}{\texttt{ 10100 }} & \rotatebox[origin=c]{90}{\texttt{ 10101 }} \\ 
        \hline
        Parity- und Datenbits & $p_{1}$                                    & $p_{2}$                                    & $d_{1}$                                    & $p_{3}$                                    & $d_{2}$                                    & $d_{3}$                                    & $d_{4}$                                    & $p_{4}$                                    & $d_{5}$                                    & $d_{6}$                                    & $d_{7}$                                    & $d_{8}$                                    & $d_{9}$                                    & $d_{10}$                                   & $d_{11}$                                   & $p_{5}$                                    & $d_{12}$                                   & $d_{13}$                                   & $d_{14}$                                   & $d_{15}$                                   & $d_{16}$                                   \\
        \hline
        Wert                  & 0                                          & 1                                          & 1                                          & 1                                          & 1                                          & 0                                          & 1                                          & 1                                          & 0                                          & 0                                          & 1                                          & 1                                          & 0                                          & 0                                          & 1                                          & 1                                          & 1                                          & 0                                          & 1                                          & 0                                          & 1                                          \\
        \hline
        Berechnung $p_1$      & x                                          &                                            & x                                          &                                            & x                                          &                                            & x                                          &                                            & x                                          &                                            & x                                          &                                            & x                                          &                                            & x                                          &                                            & x                                          &                                            & x                                          &                                            & x                                          \\
        \hline 
        Berechnung $p_2$      &                                            & x                                          & x                                          &                                            &                                            & x                                          & x                                          &                                            &                                            & x                                          & x                                          &                                            &                                            & x                                          & x                                          &                                            &                                            & x                                          & x                                          &                                            &                                            \\
        \hline
        Berechnung $p_3$      &                                            &                                            &                                            & x                                          & x                                          & x                                          & x                                          &                                            &                                            &                                            &                                            & x                                          & x                                          & x                                          & x                                          &                                            &                                            &                                            &                                            & x                                          & x                                          \\
        \hline
        Berechnung $p_4$      &                                            &                                            &                                            &                                            &                                            &                                            &                                            & x                                          & x                                          & x                                          & x                                          & x                                          & x                                          & x                                          & x                                          &                                            &                                            &                                            &                                            &                                            &                                            \\
        \hline
        Berechnung $p_5$      &                                            &                                            &                                            &                                            &                                            &                                            &                                            &                                            &                                            &                                            &                                            &                                            &                                            &                                            &                                            & x                                          & x                                          & x                                          & x                                          & x                                          & x                                          \\
        \hline
    \end{tabular}
    \caption{}
\end{sidewaystable}

\FloatBarrier
