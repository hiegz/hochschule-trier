\setcounter{section}{5}
\setcounter{subsection}{14} % 5.15
\subsection{NAT}

\begin{enumerate}[(a)]
    \item Skizzieren Sie das Netzwerk-Szenario

        --

    \item Wie viele interne Clients k"onnen in diesem Szenario maximal
        gleichzeitig im Internet surfen.

        Wenn ein Client über NAT auf das öffentliche Netz zugreifen möchte,
        weist ihm der NAT-Server einen öffentlichen Port zu. Pro IP-Adresse
        stehen maximal 65.536 Ports zur Verfügung, daher können gleichzeitig
        höchstens 65.536 Geräte über diese Adresse auf das Internet zugreifen.

    \item Geben Sie an, welche Eintr"age das NAT-Gateway verwalten muss, wenn
        \begin{itemize}
            \item der interne Client den externen DNS-Server aufruft.
                \begin{table}[h]
                    \begin{minipage}{0.49\linewidth}
                        \centering
                        \begin{tabular}{ccc}
                            Quelle          & Ziel       & Prot. \\
                            \hline
                            10.0.0.10:13001 & 1.2.3.7:53 & UDP \\
                        \end{tabular}
                        \caption{Internes Netz}
                    \end{minipage}
                    \begin{minipage}{0.49\linewidth}
                        \centering
                        \begin{tabular}{ccc}
                            Quelle          & Ziel       & Prot. \\
                            \hline
                            1.2.3.4:25565   & 1.2.3.7:53 & UDP \\
                        \end{tabular}
                        \caption{Externes Netz}
                    \end{minipage}
                \end{table}

            \item der externe Client den internen Web-Server aufruft.
                \begin{table}[h]
                    \begin{minipage}{0.49\linewidth}
                        \centering
                        \begin{tabular}{ccc}
                            Quelle          & Ziel       & Prot. \\
                            \hline
                            1.2.3.8:25567   & 1.2.3.6:80 & TCP \\
                        \end{tabular}
                        \caption{Externes Netz}
                    \end{minipage}
                    \begin{minipage}{0.49\linewidth}
                        \centering
                        \begin{tabular}{ccc}
                            Quelle          & Ziel        & Prot. \\
                            \hline
                            1.2.3.8:25567   & 10.0.0.1:80 & TCP \\
                        \end{tabular}
                        \caption{Internes Netz}
                    \end{minipage}
                \end{table}
        \end{itemize}
\end{enumerate}
