\setcounter{section}{1}
\setcounter{subsection}{1} % 1.2
\subsection{Netztopologien}

Vier paketvermittelte Netze enthalten je n Knoten. Das erste Netz hat eine
Sterntopologie mit einem zentralen Vermittler. Das zweite ist ein
bidirektionaler Ring, beim dritten ist jeder Knoten mit jedem anderen Knoten
durch ein Kabel verbunden. Im vierten Netz sind alle Knoten in einer Linie miteinander
verbunden (Linien-Topologie).

\begin{enumerate}[(a)]
        \item Erstellen Sie eine Tabelle mit den im Übungsblatt angegebenen Gr"o{\ss}en.

            \begin{table*}[h]
                \centering
                \bgroup
                \def\arraystretch{1.5}%  1 is the default, change whatever you need
                \begin{tabularx}{\textwidth}{|V{4cm}|*4{f|}}
                    \hline
                                            & Stern-Topologie & bidirekt. Ring & vollst"andig verkn"upft & Linien-Topologie \\ \hline
                    Anzahl der Kanten       & $n - 1$         & $n$            & $\binom{n}{2}$          & $n - 1$          \\ \hline
                    Anzahl der Teilstrecken & 2               & $n - 1$        & $1$                     & $n - 1$          \\ \hline
                    Bisektionsweite         & --              & 2              & $\binom{n}{2} - n$      & 1                \\ \hline
                    Grad der Knoten         & $(n - 1)$, $1$  & 2              & $n - 1$                 & $1-2$            \\ \hline
                    Kanten-Konnektivit"at   & 1               & 2              & $n - 1$                 & 1                \\ \hline
                    Symmetrie               & nein            & ja             & ja                      & nein             \\ \hline
                \end{tabularx}
                \egroup
            \end{table*}

        \item Welche dieser Gr"o{\ss}en sagen etwas "uber
            \begin{itemize}
                \item die Kosten (+ Effizienz?)
                    \begin{itemize}
                        \item Der Grad einer Topologie gibt die Anzahl der
                            Links pro Knoten an. Je höher der Grad, desto mehr
                            Kabel müssen verlegt werden, und desto teurer wird
                            das Netz.
                        \item Der Durchmesser eines Netzwerks bezeichnet die
                            maximale Anzahl an Hops, die ein Paket benötigt, um
                            von einem beliebigen Knoten zu einem anderen Knoten
                            im Netzwerk zu gelangen. Das hei{\ss}t, je größer
                            der Durchmesser, desto größer die Transferzeit im
                            ungünstigsten Fall.
                        \item Die Bisektionsweite bezeichnet die kleinste
                            Anzahl an Links, die durchtrennt werden müssen, um
                            ein Netzwerk mit N Knoten in zwei Teilnetzwerke mit
                            jeweils N/2 Knoten zu unterteilen. Eine geringere
                            Bisektionsweite hat daher negative Auswirkungen auf
                            den Zeitaufwand für den Datenaustausch zwischen den
                            beiden Netzhälften.
                    \end{itemize}
                \item die Ausfallsicherheit / Robustheit
                    \begin{itemize}
                        \item Die Konnektivität beschreibt die minimale Anzahl
                            an Knoten, die entfernt werden müssen, damit das
                            Netzwerk seine Funktionsfähigkeit verliert. Damit
                            gibt sie auch die Ausfallsicherheit des Netzwerks
                            an: Je höher die Konnektivität, desto
                            ausfallsicherer ist das Netzwerk.
                    \end{itemize}
            \end{itemize}
\end{enumerate}
