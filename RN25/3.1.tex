\setcounter{section}{3}
\setcounter{subsection}{0}
\subsection{Rahmenbildung}

\begin{enumerate}[(a)]
    \item In einem Protokoll der Sicherungsschicht wird die folgende Zeichenkodierung
        verwendet:

        \verb|A:    01000111|\\
        \verb|B:    11100011|\\
        \verb|FLAG: 01111110|\\
        \verb|ESC:  11100000|

        Wir "ubertragen die Zeichenfolge: \verb|A, B, ESC, FLAG| innerhalb der Payload.

        \begin{enumerate}[i.]
            \item Byte Stuffing

\begin{verbatim}
FLAG A B ESC ESC ESC FLAG FLAG = 01111110 01000111 11100011 11100000
                                 11100000 11100000 01111110 01111110
\end{verbatim}

            \item Bit Stuffing

\begin{verbatim}
FLAG A B ESC FLAG FLAG         = 01111110 01000111 11010001 11110000
                                 00011111 01001111 110
\end{verbatim}
        \end{enumerate}

    \item Warum wird bei Rahmenbildung nach Byte- und Bit-Stuffing das FLAG-Byte
        zu Beginn und am Ende des Rahmens verwendet? Was spricht dagegen, nur
        den Beginn des Rahmens mit dem FLAG-Byte zu markieren?

        Der Empfänger muss wissen, wann ein Frame endet. Andernfalls könnte er
        den Leerlaufzustand des Senders als Teil des Frames interpretieren und
        die Daten falsch verstehen. Mit einem End-Flag jedoch setzt er einfach
        seinen Zustand zurück und wartet erneut auf das Start-Flag.
\end{enumerate}
