\setcounter{section}{3}
\setcounter{subsection}{4} % 3.5
\subsection{CRC}

Sie wollen (ich? nicht wirklich) die Bits \verb|10011101| CRC-geschützt unter
Verwendung des Generator-Polynoms $C(x) = x^3 + 1$ übertragen.
\begin{enumerate}[(a)]
    \item Welche Bits "ubermitteln Sie? Wie berechnen sich diese?
        \begin{align*}
            \text{M}(x) \cdot x^2       &= \text{D}(x) \cdot \text{C}(x) + \text{R}(x) \\
            x^9 + x^6 + x^5 + x^4 + x^2 &= (x^6 + x^2 + x) \cdot (x^3 + 1) + x \\[10pt]
            \text{P}(x)                 &= \text{M}(x) \cdot x^k - \text{R}(x) \\
            \text{P}(x)                 &= x^9 + x^6 + x^5 + x^4 + x^2 - x \\
            \text{P}(x)                 &= x^9 + x^6 + x^5 + x^4 + x^2 + x
        \end{align*}

        Die übertragenen Bits sind also \verb|1001110110|.

    \item Was berechnet der Empf"anger, wenn die Nachricht, unver"andert bei ihm eintrifft?

        Der Empfänger überprüft die empfangene Nachricht, indem er sie durch
        das Generator-Polynom dividiert. Bleibt kein Rest, so liegt entweder kein
        Übertragungsfehler vor oder der Fehler konnte nicht erkannt werden.
        Gibt es jedoch einen Rest, wurde die Nachricht während der Übertragung
        verfälscht.

    \item Das hochwertigste Bit (ganz links) der Nachricht wird durch Rauschen
        invertiert. Geben Sie das Fehlerpolynom E(x) an. Wie lautet das
        Ergebnis für die CRC-Berechnung des Empfängers? Woher weiß der
        Empfänger, dass ein Fehler aufgetreten ist?

        In diesem Fall gilt $\text{E}(x) = x^9$. Der Empf"anger erh"alt also
        die Bitfolge \verb|0001110110| statt \verb|1001110110|. Daher bleibt
        bei der Division der empfangenen Nachricht durch das Generator-Polynom
        ein Rest. Daran erkennt der Empf"anger, dass w"ahrend der "Ubertragung ein
        Fehler aufgetreten ist.

    \item Welche Fehler erkennt das Generator-Polynom $\text{C}(x)$?
        \begin{itemize}
            \item alle einzelnen Bit-Fehler.
            \item alle Fehler mit ungerader Anzahl fehlerhafter Bits, da
                $\text{C}(x)$ eine gerade Anzahl von Termen besitzt.
            \item alle B"undelfehler bis zur L"ange 3.
            \item viele Fehler mehr als 3 Bits.
        \end{itemize}

    \item Geben Sie ein Beispiel f"ur einen Fehler an, der nicht erkannt wird.
        \begin{equation*}
            \text{E}(x) = x^9 + x^6 + x^3 + 1
        \end{equation*}
\end{enumerate}
