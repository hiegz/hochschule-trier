\documentclass[10pt, oneside]{article}
\usepackage[a4paper, total={5.5in, 9in}]{geometry}
\usepackage[ngerman]{babel}
\usepackage{import}

\import{../../.texit/include}{preamble}

\title{Objektorientierte Programmierung\\[15pt]\Large{Übungsblatt 2}\\[10pt]\Large{SoSe 2025}}
\author{Volodymyr But\\[10pt]Hochschule Trier}
\date{}

% - - - - - - - - - - - - - - - - - - - - - - - - - - - - - - - - - - - - - - %

\begin{document}

\maketitle
\vspace{25px}

\section{Aufgabe 1}

Welche Ausgabe erzeugt das folgende Programmstück?

\begin{verbatim}
int x = 80, y = 60;
if(x < 100)
    if(y > 50)
        System.out.println("eins");
    else
        System.out.println("zwei");
else
    System.out.println("drei");
\end{verbatim}

\begin{verbatim}
> eins
\end{verbatim}

\section{Aufgabe 2}

Welche Ausgabe erzeugen die folgenden Anweisungen?

\begin{verbatim}
int x = 110, y = 40;
if(x < 100) {
    if(y > 50)
        System.out.println("eins");
}
else {
    System.out.println("zwei");
    System.out.println("drei");
}
\end{verbatim}

\begin{verbatim}
> zwei
  drei
\end{verbatim}

\pagebreak
\section{Aufgabe 3}

Führen Sie die folgende bedingte Anweisung „von Hand“ aus. Welche Ausgabe wird
für jeden der Werte von t erzeugt?

\begin{verbatim}
if(t < 15)
    if(t > 7)
        System.out.println("eins");
    else
        System.out.println("zwei");
    else
        if(t < 18)
            System.out.println("drei");

System.out.println("Ende");
\end{verbatim}

\begin{enumerate}[(a)]
    \item \verb|t = 9|
\begin{verbatim}
> eins
  Ende
\end{verbatim}
    \item \verb|t = 20|
\begin{verbatim}
> Ende
\end{verbatim}
    \item \verb|t = 4|
\begin{verbatim}
> zwei
  Ende
\end{verbatim}
    \item \verb|t = 7|
\begin{verbatim}
> zwei
  Ende
\end{verbatim}
    \item \verb|t = 16|
\begin{verbatim}
> drei
  Ende
\end{verbatim}
\end{enumerate}

\section{Aufgabe 4}

Führen Sie die folgende bedingte Anweisung „von Hand“ aus und beantworten Sie die
angegebenen Fragen

\begin{verbatim}
if(a < b)
    if(m > n)
        System.out.println("schwarz");
    else
        System.out.println("rot");
else
    if(m > n)
        System.out.println("gruen");
    else
        System.out.println("blau");
\end{verbatim}

\pagebreak
\begin{enumerate}
    \item Was wird ausgegeben f"ur \verb|a = 3|, \verb|b = 4|, \verb|m = 1| und \verb|n = 5|?
\begin{verbatim}
> rot
\end{verbatim}
    \item Was wird ausgegeben f"ur \verb|a = 5|, \verb|b = 3|, \verb|m = 7| und \verb|n = 4|?
\begin{verbatim}
> gruen
\end{verbatim}
    \item Was wird ausgegeben wenn \verb|a < b| und \verb|m > n| gilt?
\begin{verbatim}
> schwarz
\end{verbatim}
    \item Schreiben Sie eine gleichwertige geschachtelte \verb|if|-Anweisung,
        die mit dem folgenden Vergleich beginnt: \verb|if (m > n) ...|
\begin{verbatim}
if (m > n)
    if (a < b)
        System.out.println("schwarz");
    else
        System.out.println("gruen");
else
    if (a < b)
        System.out.println("rot");
    else
        System.out.println("blau");
\end{verbatim}
\end{enumerate}

\section{Aufgabe 5}

Führen Sie die folgende bedingte Anweisung „von Hand“ aus. Welche Ausgabe wird
für jeden der Werte von n erzeugt?

\begin{verbatim}
if (n <= 10)
    if (n / 3 < 2)
        System.out.println("eins");
    else if (n / 3 >= 3)
        System.out.println("zwei");
    else
        System.out.println("drei");
else if (n > 20)
    System.out.println("vier");
else
    System.out.println("fuenf");
\end{verbatim}

\begin{enumerate}[(a)]
    \item \verb|n = 17|
\begin{verbatim}
> fuenf
\end{verbatim}
    \item \verb|n = 25|
\begin{verbatim}
> vier
\end{verbatim}
    \item \verb|n = 7|
\begin{verbatim}
> drei
\end{verbatim}
    \item \verb|n = 0|
\begin{verbatim}
> eins
\end{verbatim}
    \item \verb|n = 9|
\begin{verbatim}
> zwei
\end{verbatim}
\end{enumerate}

\section{Aufgabe 6}

Schreibena Sie eine switch-Anweisung, die äquivalent zur folgenden if-Anweisung
ist. a (vom Typ int) soll als Auswahlausdruck verwendet werden.

\begin{verbatim}
if(a == 3)
    result = a;
else if(a == 6)
    result = a + 10;
else if(a == 10)
    result = a + 20;
else
    result = a + 30;
\end{verbatim}

\begin{verbatim}
switch (a) {
    case 3:
        result = a;
        break;
    case 6:
        result = a + 10;
        break;
    case 10:
        result = a + 20;
        break;
    default:
        result = a + 30;
        break;
}
\end{verbatim}

\section{Aufgabe 7}

Schreibe Sie eine if-Anweisung, die zur folgenden switch-Anweisung äquivalent
ist.

\begin{verbatim}
switch(n)
{
    case 39:
    case 15: System.out.println("gewonnen");
             break;
    default: System.out.println("verloren");
             break;
}
\end{verbatim}

\begin{verbatim}
if (n == 39 || n == 15)
    System.out.println("gewonnen");
else
    System.out.println("verloren");
\end{verbatim}

\pagebreak
\section{Aufgabe 8}

Betrachten Sie die folgende Fallunterscheidung:

\begin{verbatim}
switch (n) {
    case 1:
    case 3:
    case 4:
        System.out.println(n);
        break;
    case 2:
    case 5:
    case 6:
        System.out.println(n * 10);
        break;
    default:
        System.out.println(n * 100);
        break;
}
\end{verbatim}

Was wird ausgegeben f"ur
\begin{enumerate}[(a)]
    \item \verb|n = 5| \\
        \verb|> 50|
    \item \verb|n = 4| \\
        \verb|> 4|
    \item \verb|n = 8| \\
        \verb|> 800|
\end{enumerate}

Nehmen Sie an, dass die \verb|break|-Anweisungen aus der
\verb|switch|-Anweisung entfernt wurden. Was würde ausgegeben für
\begin{enumerate}[(a)]
    \item \verb|n = 5| \\
        \verb|> 50| \\
        \verb|  500|
    \item \verb|n = 4| \\
        \verb|> 4| \\
        \verb|  40| \\
        \verb|  400|
    \item \verb|n = 8| \\
        \verb|> 800|
\end{enumerate}

\pagebreak
\section{Aufgabe 9}

Geben Sie zur folgenden if-Anweisung eine gleichwertige switch-Anweisung an.

\begin{verbatim}
if (ch == 'a' || ch == 'b' || ch == 'c')
    System.out.println("im Bereich \"abc\"");
else if (ch >= 'g' && ch < 'i')
    System.out.println("in der Mitte des Alphabets");
else if (ch == 'z')
    System.out.println("ein z, ein z, ...");
else
    System.out.println("andere Buchstaben");
\end{verbatim}

\begin{verbatim}
switch (ch) {
    case 'a':
    case 'b':
    case 'c':
        System.out.println("im Bereich \"abc\"");
        break;
    case 'g':
    case 'h':
        System.out.println("in der Mitte des Alphabets");
        break;
    case 'z':
        System.out.println("ein z, ein z, ...");
        break;
    default:
        System.out.println("andere Buchstaben");
}
\end{verbatim}

\section{Aufgabe 10}

Schreiben Sie eine switch-Anweisung, die äquivalent zur folgenden if-Anweisung
ist. Wie beurteilen Sie Ihre Lösung im Vergleich zur vorliegenden if-Anweisung?

\begin{verbatim}
if (n < 5 || n > 8)
    System.out.println("Bereich 1");
else if (n >= 5 && n <= 7)
    System.out.println("Bereich 2");
else
    System.out.println("Bereich 3");
\end{verbatim}

\begin{verbatim}
switch (n) {
    default:
        System.out.println("Bereich 1");
        break;
    case 5:
    case 6:
    case 7:
        System.out.println("Bereich 2");
        break;
    case 8:
        System.out.println("Bereich 3");
        break;
}
\end{verbatim}

Im Allgemeinen wurde die \verb|switch|-Anweisung nicht daf"ur entwickelt,
Wertebereiche zu verarbeiten. \verb|switch| eignet sich hervorragend f"ur
festgelegte, diskrete Werte - wie Konstanten oder Enums. Wenn es darum geht,
Bedingungen basierend auf Wertebereichen auszuwerten, sind
\verb|if|-Anweisungen die bessere Wahl, da sie nicht nur die Absicht klarer
widerspiegeln, sondern auch eine bessere Lesbarkeit und Flexibilit"at bieten.

In unserem Fall ist die \verb|if|-Anweisung nicht nur flexibler, sondern auch
deutlich lesbarer und verst"andlicher, als zu versuchen, eine
\verb|switch|-Anweisung f"ur Wertebereiche zu verwenden.

\section{Aufgabe 11}

Nehmen Sie an, dass x, y, z und val vom Typ double sind. Geben Sie für die
folgende Wertzuweisung eine gleichwertige if-Anweisung an.

\verb|val = x >= y && x <= z ? z-y : x < y ? 0 : 1.0E25;|

\begin{verbatim}
double val;
if (x >= y && x <= z)
    val = z - y;
else if (x < y)
    val = 0;
else
    val = 1.0E25;
\end{verbatim}

\end{document}
