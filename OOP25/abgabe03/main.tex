\documentclass[10pt, oneside]{article}
\usepackage[a4paper, total={5.5in, 9in}]{geometry}
\usepackage[ngerman]{babel}
\usepackage{import}

\import{../../.texit/include}{preamble}

\title{Objektorientierte Programmierung\\[15pt]\Large{Übungsblatt 3}\\[10pt]\Large{SoSe 2025}}
\author{Volodymyr But\\[10pt]Hochschule Trier}
\date{}

% - - - - - - - - - - - - - - - - - - - - - - - - - - - - - - - - - - - - - - %

\begin{document}

\maketitle
\vspace{25px}

\section{Aufgabe 1}

Die folgende \verb|while|-Schleife soll die Ausgabe \verb|1 4 9 16 25| erzeugen.
Ergänzen Sie den Programmcode entsprechend.

\begin{verbatim}
int i = 1;
while (i <= 5) {
    System.out.print(i * i + " ");
    i++;
}
\end{verbatim}

\section{Aufgabe 2}

Die folgende \verb|while|-Schleife soll die Ausgabe \verb|100 64 36 16 4|
erzeugen. Ergänzen Sie den Programmcode entsprechend.

\begin{verbatim}
int i = 10;
while (i >= 2) {
    System.out.print(i * i + " ");
    i -= 2;
}
\end{verbatim}

\section{Aufgabe 3}

Führen Sie die folgende \verb|while|-Anweisung „von Hand“ aus. Welchen Wert hat die
Variable \verb|n| nach dem Abarbeiten der Schleife?

\begin{verbatim}
int n = 0, i = 1;
while(i < 8)
{
    n += i;
    i += 2;
}
\end{verbatim}

\verb|16|

\section{Aufgabe 4}

Führen Sie die folgende \verb|while|-Anweisung „von Hand“ aus. Welchen Wert hat die
Variable \verb|n| nach dem Abarbeiten der Schleife?

\begin{verbatim}
int n = 0, jahr = 1980;
while(jahr <= 1989)
{
    if(jahr % 4 == 0)
        n++;
    jahr++;
}
\end{verbatim}

\verb|3|

\section{Aufgabe 5}

Führen Sie die folgende do-while-Anweisung „von Hand“ aus. Was wird
ausgegeben?

\begin{verbatim}
int i = 5;
do
{
    System.out.print(i*i + " ");
    i--;
}
while(i > 0);
\end{verbatim}

\verb|25 16 9 4 1|

\section{Aufgabe 6}

Führen Sie die folgende do-while -Anweisung „von Hand“ aus. Was wird
ausgegeben?

\begin{verbatim}
int i = 5;
do
{
    i--;
    System.out.print(i*i + " ");
}
wfhile(i > 0);
\end{verbatim}

\verb|16 9 4 1 0|

\pagebreak
\section{Aufgabe 7}

Schleifen können vertrackte syntaktische und logische Fehler enthalten, die dazu
führen, dass die Schleife nicht übersetzt wird bzw. dass sie nicht korrekt abläuft.
Welche Fehler enthalten die folgenden Schleifen. Erläutern Sie jeweils das Problem.

\begin{enumerate}
\item
\begin{verbatim}
int i = 1;
while(i <= 5);
{
    System.out.println(i);
    i++;
}
\end{verbatim}
Semikolon nach while-Bedingung. Das Semikolon ; direkt nach \verb|while(i <= 5)|
beendet die Definiton der Schleife sofort. Dadurch wird eine leere Schleife
ausgeführt, die nichts tut, solange die Aussage \verb|i <= 5| gilt.

\item
\begin{verbatim}
int i = 9;
do
    System.out.println(i);
    i--;
while(i > 0);
\end{verbatim}
Fehlende geschweifte Klammern bei \verb|do|. Dies ist in dem Fall ein
Syntaxfehler.
\end{enumerate}

\section{Aufgabe 8}

Welche Ausgabe erzeugt der folgende Java-Code?

\begin{verbatim}
int sum = 0, i = 1;
while(i <= 10)
{
    if(i % 2 == 0)
    sum += i;
    i++;
}
System.out.println(sum);
\end{verbatim}

\verb|30|

\section{Aufgabe 9}

Welche Ausgabe erzeugt der folgende Java-Code?

\begin{verbatim}
int fact = 1, i = 1;
do
{
    fact *= i;
    i++;
} while(i < 6);
System.out.println(fact);
\end{verbatim}

\verb|120|

\section{Aufgabe 10}

Schreiben Sie eine \verb|while|-Schleife, welche die ganzen Zahlen von 10 bis
100 ausgibt, jede Zahl auf einer eigenen Zeile.

\begin{verbatim}
int i = 10;
while (i <= 100) {
    System.out.println(i);
    i += 10;
}
\end{verbatim}

\section{Aufgabe 11}

Geben Sie für die folgende \verb|while|-Schleife eine gleichwertige
\verb|do-while|-Schleife an. Die Variablen \verb|wert| und \verb|i| wurden
zuvor als Variablen vom Typ int deklariert und initialisiert.

\begin{verbatim}
while(wert > 0)
{
    wert -= i*i;
    i--;
}
\end{verbatim}

\begin{verbatim}
do {
    if (wert <= 0)
        break;
    wert -= i * i;
    i--;
} while (true);
\end{verbatim}

\section{Aufgabe 12}

Welche Ausgabe erzeugen die folgenden Anweisungen?

\begin{enumerate}
\item
\begin{verbatim}
int i = 1;
while(i <= 8)
{
    System.out.print(i + " ");
    i += 2;
}
\end{verbatim}

\verb|1 3 5 7|

\item
\begin{verbatim}
int i = 4, k;
do
{
    k = 2*i;
    System.out.print(k + " ");
    i--;
} while(i >= 0);
\end{verbatim}

\verb|8 6 4 2 0|
\end{enumerate}

\section{Aufgabe 13}

Führen Sie die folgende while-Anweisung „von Hand“ aus. Welchen Wert hat die
Variable n nach dem Abarbeiten der Schleife?

\begin{verbatim}
int n = 0, i = 5;
while(i > 0)
{
    n += 2 * i;
    i--;
}
\end{verbatim}

\verb|30|

\section{Aufgabe 14}

Führen Sie die folgende do-while-Anweisung „von Hand“ aus. Welchen Wert hat die
Variable n nach dem Abarbeiten der Schleife?

\begin{verbatim}
int n = 0, i = 1;
do
{
    n += 10;
    i++;
} while(i <= 5);
\end{verbatim}

\verb|50|

\section{Aufgabe 15}

Welche Ausgabe erzeugen die folgenden Anweisungen?

\begin{enumerate}[(a)]
\item
\begin{verbatim}
int t = 6;
while(t > 3)
{
    System.out.println(t);
    t--;
}
System.out.println(t);

6
5
4
3
\end{verbatim}

\pagebreak
\item
\begin{verbatim}
int t = 6;
do
{
    System.out.println(t);
    t--;
} while(t > 3);
System.out.println(t);

6
5
4
3
\end{verbatim}
\end{enumerate}

\section{Aufgabe 16}

Schreiben Sie Schleifen, welche die folgenden Algorithmen implementieren.

\begin{enumerate}[(a)]
\item Addieren Sie mit einer \verb|while|-Schleife alle geraden ganzen Zahlen von 2 bis 20.
    Geben Sie die Summe aus.

\begin{verbatim}
int i = 2;
int sum = 0;
while (i <= 20) {
    sum += i;
    i += 2;
}
System.out.println(sum); // 110
\end{verbatim}

\item Addieren Sie mit einer \verb|do-while|-Schleife die ganzen Zahlen 1, 2, 3, ...,
    bis die Summe 1 + 2 + 3 ... größer als 500 geworden ist. Geben Sie die
    Summe aus.

\begin{verbatim}
int i = 1;
int sum = 0;
do {
    sum += i;
    i += 1;
} while (sum <= 500);
System.out.println(sum);
\end{verbatim}

\item Eine \verb|while|-Schleife startet mit \verb|n = 10| und git $10^2, 8^2, ...,  2^2$ aus.

\begin{verbatim}
int n = 10;
while (n >= 2) {
    System.out.println(n * n);
    n -= 2;
}
\end{verbatim}

\end{enumerate}

\pagebreak
\section{Aufgabe 17}

Zwei Varianten einer \verb|while|-Schleife werden verwendet, um die Ausgabe
\verb|14 16 18 20 22| zu erzeugen. Ergänzen Sie den Programmcode entsprechend.

\begin{enumerate}[(a)]
\item
\begin{verbatim}
int i = 14;
while (i <= 22) {
    System.out.print(i + " ");
    i += 2;
}
\end{verbatim}

\item
\begin{verbatim}
int i = 7;
while (i <= 11) {
    System.out.print(2 * i + " ");
    i += 1;
}
\end{verbatim}
\end{enumerate}

\section{Aufgabe 18}

Schleifen können vertrackt sein. Führen Sie den folgenden Java-Code „von Hand“ aus.
Was wird ausgegeben?

\begin{enumerate}[(a)]
\item
\begin{verbatim}
boolean stop = false;
int i;
while(stop = false)
{
    i = TastaturEingabe.readInt("i: ");
    if(i != 7)
        System.out.println(i * i);
    else
        stop = true;
}
\end{verbatim}

nichts. Die Schleife wird nicht ausgef"uhrt.

\item
\begin{verbatim}
int i = 10;
while(i > 0)
{
    System.out.print(i);
    i++;
}

1011121314151617181920212223242526272829303132....
\end{verbatim}

\end{enumerate}

\pagebreak
\section{Aufgabe 19}

Geben Sie für die folgende while-Schleife eine gleichwertige for-Schleife an.

\begin{verbatim}
int i = 10;
while(i >= 1)
{
    System.out.print(i + " ");
    i--;
}


for (int i = 10; i >= 1; i--) {
    System.out.print(i + " ");
}
\end{verbatim}

\section{Aufgabe 20}

Geben Sie für die folgende for-Anweisung eine äquivalente while-Schleife an.

\begin{verbatim}
for(int i = 1; i < 10; i++)
    System.out.println(i+5);

int i = 1;
while (i < 10) {
    System.out.println(i + 5);
    i++;
}
\end{verbatim}

\section{Aufgabe 21}

Geben Sie für die folgende while-Schleife eine gleichwertige for-Schleife an.

\begin{verbatim}
int i = 1, sum = 0;
while(i <= 15)
{
    sum += i;
    i += 2;
}


for (int i = 1, sum = 0; i <= 15; sum += i, i += 2)
    ;
\end{verbatim}

\pagebreak
\section{Aufgabe 22}

Führen Sie die folgenden Anweisungen „von Hand“ aus. Was wird ausgegeben?

\begin{enumerate}[(a)]
\item
\begin{verbatim}
for(int i = 1; i < 5; i++)
    System.out.println(i*10);

10
20
30
40
\end{verbatim}

\item
\begin{verbatim}
for(int i = 5; i > 0; i--)
    System.out.println(i*i);

25
16
9
4
1
\end{verbatim}
\end{enumerate}

\section{Aufgabe 23}

Die Variable \verb|count| führt Buch über die Gesamtzahl der Iterationen (Schleifen-
durchläufe). Welchen Wert hat count nach dem Abarbeiten der folgenden
Anweisungen?

\begin{enumerate}[(a)]
\item
\begin{verbatim}
int count = 0;
for(int i = 1; i <= 5; i++)
    for(int j = 1; j <= 3; j++)
        count++;

15
\end{verbatim}

\item
\begin{verbatim}
int count = 0;
for(int i = 1; i <= 3; i++)
    for(int j = 0; j < i; j++)
        count++;

6
\end{verbatim}
\end{enumerate}

\section{Aufgabe 24}

Welche Ausgabe erzeugt der folgende Programmcode?

\begin{verbatim}
int i, j;
for(i = 1, j = 5; i + 2*j > 9; i++, j--)
    System.out.println(2*i + 3*j);

17
16
\end{verbatim}

\section{Aufgabe 25}

Schleifen können vertrackte syntaktische und logische Fehler enthalten, die dazu
führen, dass die Schleife nicht übersetzt wird bzw. dass sie nicht korrekt abläuft.
Welche Fehler enthalten die folgenden Schleifen. Erläutern Sie jeweils das Problem.

\begin{enumerate}[(a)]
\item
\begin{verbatim}
for(int i = 1, i <= 5, i++)
    System.out.println(i);
\end{verbatim}
Falsche Trennzeichen, Kommas statt Semikolons

\item
\begin{verbatim}
for(int i = 10; i > 0; i++)
    System.out.println(i);
\end{verbatim}
Falsche Schleifenrichtung, die Schleife wird nie beendet.

\end{enumerate}

\section{Aufgabe 26}

Führen Sie die folgende for-Anweisung „von Hand“ aus. Welches Problem tritt hier
auf?

\begin{verbatim}
for(int n = 5; n < 25; n++)
{
    System.out.println(n);
    n = TastaturEingabe.readInt("n: ");
}
\end{verbatim}

Welche Ausgabe wird erzeugt bei den folgenden Eingabewerten:

\verb|10 22 6 23 24|

\begin{verbatim}
5
11
23
7
24
\end{verbatim}

\section{Aufgabe 27}

Welche Ausgabe erzeugt das folgende Programm? Können Sie beschreiben, was das
Programm tut?

\begin{verbatim}
import utilities.TastaturEingabe;
public class loop27
{
    public static void main(String args[])
    {
        long n, p;
        n = TastaturEingabe.readLong(
              "Bitte eine ganze Zahl eingeben: ");
        for(p = 1; p < n; p *= 2)
            System.out.print(p + " ");
        System.out.println();
    }
}
\end{verbatim}

\pagebreak

\begin{enumerate}
    \item \verb|n = 40  : 1 2 4 8 16 32|
    \item \verb|n = 50  : 1 2 4 8 16 32|
    \item \verb|n = 100 : 1 2 4 8 16 32 64|
\end{enumerate}

Das Program gibt alle Zweierpotenzen aus, die kleiner als \verb|n| sind.

\section{Aufgabe 28}

Ergänzen Sie jeweils den Schleifentest derart, dass 12 Sternchen ausgegeben werden.

\begin{enumerate}[(a)]
\item
\begin{verbatim}
for (int i = 1; i <= 3; i++)
    for (int j = 1; j <= 4; j++)
        System.out.print('*');
\end{verbatim}
\item
\begin{verbatim}
for (int i = 3; i < 5; i++)
    for (int j = 6; j > 0; j--)
        System.out.print('*');
\end{verbatim}
\item
\begin{verbatim}
for (int j = 2; j < 38; j += 3)
    System.out.print('*');
\end{verbatim}
\end{enumerate}

\section{Aufgabe 29}

Wie viele Sternchen geben die folgenden geschachtelten Schleifen aus? Die
Deklarationen: int i, j, k; seien gegeben.

\begin{enumerate}[(a)]
\item
\begin{verbatim}
i = 0;
while(i <= 10)
{
    j = 1;
    while(j < i)
    {
        System.out.print('*');
        j++;
    }
    i++;
}

45
\end{verbatim}

\item
\begin{verbatim}
for(i = 0; i < 3; i++)
    for(j = 0; j < 3; j++)
        for(k = 0; k < 2; k++)
            System.out.print('*');

18
\end{verbatim}

\pagebreak
\section{Aufgabe 30}

Die Deklarationen int i, j, k, sum; seien gegeben. Welche Ausgabe erzeugen
die beiden Programmausschnitte?

\begin{enumerate}[(a)]
\item
\begin{verbatim}
for(sum = 0, i = 0, k = 8; i < k; i++, k--)
    sum += 2 * i + k;
System.out.println(sum);

38
\end{verbatim}

\item
\begin{verbatim}
for(i = 0, j = 1; i * j < 100; i++, j *= 10)
    System.out.println(i * j);

0
10
\end{verbatim}
\end{enumerate}

\end{enumerate}

\section{Aufgabe 31}

Betrachten Sie die folgende Schleife. Die Deklarationen int i, n; seien gegeben.

\begin{verbatim}
n = TastaturEingabe.readInt("n: ");
for(i = 1; i < 10; i++)
{
    if(i % 4 == 0)
        continue;
    else if(i == n)
        break;
    else
        System.out.println(i + " ");
}
\end{verbatim}

\begin{enumerate}[(a)]
\item
\begin{verbatim}
n = 7
1
2
3
5
6
\end{verbatim}

\item
\begin{verbatim}
n = 4
1
2
3
5
6
7
9
\end{verbatim}

\item
\begin{verbatim}
n = 9
1
2
3
5
6
7
\end{verbatim}

\end{enumerate}

\section{Aufgabe 32}

Die Deklarationen int i = 0, sum = 0; seien gegeben. Welche Ausgabe
erzeugt der folgende Programmausschnitt?

\begin{verbatim}
while(i < 8)
{
    if(i % 2 == 1)
    {
        i++;
        continue;
    }
    else
    {
        sum += i;
        i++;
    }
}
System.out.println(sum);

12
\end{verbatim}

\section{Aufgabe 33}

Schreiben Sie eine Schleife, die bis zu 10 ganze Zahlen einliest und jede dieser Zahlen
nach den folgenden Regeln verarbeitet.

\begin{enumerate}[(1)]
\item Die Schleife wird beendet, wenn der Eingabewert 6 ist.
\item Wenn der Eingabewert durch 3 teilbar ist, wird der nächste
    Schleifendurchlauf begonnen
\item Wenn der Eingabewert kleiner als 8 ist, wird das Quadrat der Zahl
    ausgegeben; andernfalls wird die Zahl selbst ausgegeben.
\end{enumerate}

\begin{verbatim}
import java.util.Scanner;

public class A33 {
    public static void main(String[] args) {
        Scanner in = new Scanner(System.in);
        for (int i = 0; i < 10; i++) {
            int num = in.nextInt();
            if (num == 6)
                break;
            if (num % 3 == 0)
                continue;
            if (num < 8)
                System.out.println(num * num);
            else
                System.out.println(num);
        }
        in.close();
    }
}
\end{verbatim}

\end{document}
